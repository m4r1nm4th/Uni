\section{Equivariant Cohomology Theories}

Throughout this section fix a compact Lie group $G$.

\begin{definition}
We consider the category of $G$-anima which is defined as the the presheaf category 
\[
\An_G := \Psh(\mathrm{Orb}_G).
\]
For a closed subgroup $H \leq G$ we let $\AffLin^H$ be the category consisting of finite dimensional euclidean vector spaces with an 
$H$-action through linear isometries. A morphism $f \colon V \to W$ is an affine linear morphism that 
decomposes as an equivariant linear isometric embedding $\iota \colon V \hookrightarrow W$ followed by a translation along a vector 
$w \in (\iota(V)^\perp)^H$ out of the $H$ fixed points of the orthogonal complement of $\iota(V)$.
Thus $\AffLin^{\{e\}} \simeq \AffLin$.
The assignment $G/H \mapsto \AffLin^H$ makes $\AffLin^{-}$ into a $G$-category, that is a functor 
$\mathrm{Orb}_G^\op \to \catinfty$.
Let $X \mapsto \AffLin^X$ the unique limit preserving extension $\An_G^\op \to \catinfty$ of $\AffLin^{-}$.
Let 
\[
  \AVB^\vop_G := \int_{X \in \An_G^\op} (\AffLin^{X})^\op   
\]
be the unstraightening of the functor $\AffLin^{-}$.
The category $\AVB^\vop_G$ is the vertical opposite of the \emph{category of (equivariant) affine vector bundles over $G$} 
\[
  \AVB_G := \int_{X \in \An_G} \AffLin^X.
\]
For every $G$-representation $V \in \AffLin^G$ there is a functor 
\[
\iota_V \colon \An_G \to \AVB_G    
\]
that equips a $G$-anima with the trivial $V$ vector bundle over it. One can construct $\iota_V$ 
as in the non-equivariant case via the source of a cartesian lift 
\[\begin{tikzcd}
	\pt && {\Fun(\An_G,\AVB_G)} \\
	{\Delta^1} && {\Fun(\An_G,\An_G)}
	\arrow["{\pr_*}", from=1-3, to=2-3]
	\arrow["{\id \to \mathrm{const}_{G/G}}"', from=2-1, to=2-3]
	\arrow["{\mathrm{const}_{(G/G,V)}}", from=1-1, to=1-3]
	\arrow["{\mathrm{target}}"', from=1-1, to=2-1]
	\arrow[dashed, from=2-1, to=1-3].
\end{tikzcd}\]

As in the case $G = \{e\}$ we have distinguished squares in the category $\AVB^\vop_G$




Let $(\An_G)^\omega_*$ be the category of pointed objects in the category compact $G$-anima.
A \emph{$G$-cohomology theory $E$} is a functor $E \colon ((\An_G)^\omega_*)^\op \times \AffLin_G^\simeq \to \Ab$
together with natural equivalences 
\[
    \sigma_V \colon E(X,W) \to E(S^V \wedge X, V \oplus W)
\]
such that 
$E(-, \R^\bullet_{\mathrm{triv}} \oplus V)$ is a cohomology theory for every $V$ and the following diagrams commute
\[\begin{tikzcd}
	& {E(X,V)} \\
	{E(S^W\wedge X,W\oplus V)} && {E(S^U\wedge X, U \oplus V)} \\
	{E(S^U\wedge S^W \wedge X, U \oplus W \oplus V)} && {E(S^W \wedge S^U \wedge X, W \oplus U \oplus V)}
	\arrow["{\sigma_W}", from=1-2, to=2-1]
	\arrow["{\sigma_U}"', from=1-2, to=2-3]
	\arrow["{\sigma_U}", from=2-1, to=3-1]
	\arrow["{\sigma_W}"', from=2-3, to=3-3]
	\arrow["\simeq", from=3-1, to=3-3]
\end{tikzcd},\]

\[\begin{tikzcd}
	{E(X;W)} & {E(S^V \wedge X; V \oplus W)} \\
	{E(S^{U \oplus V} \wedge X; U \oplus V \oplus W)} & {E(S^U \wedge S^V \wedge X;U \oplus V \oplus W)}
	\arrow["{\sigma_V}", from=1-1, to=1-2]
	\arrow["\simeq"', from=2-1, to=2-2]
	\arrow["{\sigma_U}", from=1-2, to=2-2]
	\arrow["{\sigma_{U\oplus V}}"', from=1-1, to=2-1].
\end{tikzcd}\]

%TODO define genuine cohomology theory 

\end{definition}

\begin{definition}
    Let $E \colon (\AVB^\vop_G|_{\An_G^\omega})^\op \to \Ab$ be a genuine cohomology theory, $Y \to Z$ a map of $G$-anima and fix $V \in \AffLin^G$, then we define 
    the $V$-th $E$-cohomology group of $Z$ relative to $Y$ as
    \[
      E^V(Z,Y) := E(Z,Y;V) := \ker ( E(\iota_V Z \to \iota_V Y)). 
    \]
    For a pointed $G$-anima $X$ we set 
    \[
      \tilde E^V(X) := \tilde E(X;V) := E(X, \pt; V).
    \]
\end{definition}
\begin{lemma}

    The diagram 
    \[\begin{tikzcd}
        {E(\pt, W)} & {E(X,W)} \\
        {E(S^V\vee X,V \oplus W)} & {E(S^V \times X, V \oplus W)} \\
        {E(\pt, V \oplus W)} & {E(S^V \wedge X, V \oplus W)}
        \arrow[from=1-2, to=1-1]
        \arrow[from=1-1, to=2-1]
        \arrow[from=2-2, to=2-1]
        \arrow[from=1-2, to=2-2]
        \arrow[from=3-2, to=2-2]
        \arrow[from=3-1, to=2-1]
        \arrow[from=3-2, to=3-1]
    \end{tikzcd}\]
    induces on horizontal kernels a zigzag 
    \[
    \tilde E(X,W) \to E(S^V \times X, S^V \vee X ; V \oplus W) \leftarrow \tilde E(S^V \wedge X, V \oplus W),
    \]
    where both morphisms are isomorphisms.
\end{lemma}

\begin{proof}
By the snake-lemma the following ladder diagram 
    \[\begin{tikzcd}
        & 0 & {E(S^V\times X;V\oplus W)} & {E(S^V\times X;V\oplus W)} & 0 \\
        0 & {\tilde E(S^V;V\oplus W)} & {E(S^V\vee X;V\oplus W)} & {E(X;V\oplus W)} & 0
        \arrow[from=2-2, to=2-3]
        \arrow[from=2-3, to=2-4]
        \arrow[from=2-4, to=2-5]
        \arrow[from=2-1, to=2-2]
        \arrow[from=1-3, to=2-3]
        \arrow[from=1-2, to=1-3]
        \arrow[from=1-2, to=2-2]
        \arrow[from=1-3, to=1-4]
        \arrow[from=1-4, to=1-5]
        \arrow[two heads, from=1-4, to=2-4]
    \end{tikzcd}\]
induces a long exact sequence
\[
0 \to E(S^V \times X, S^V \vee X; V \oplus W) \to E(X, W) \to \tilde E(S^V; V \oplus W) \to \dots 
\]
The Thom-isomorphism axiom shows that $\tilde E(S^V; V\oplus W) \simeq E(\pt, W)$. 
Hence \[E(S^V \times X, S^V \vee X; V \oplus W) \simeq \tilde E(X;W).\]
A careful examination shows that the left arrow of the span estabishes this isomorphism. The right arrow of the span 
is an isomorphism because of the Mayer-Vietoris axiom.
\end{proof}
\begin{definition}
    We define 
    \[
    \sigma_V \colon \tilde  E^W(X) \to \tilde E^{V\oplus W}(S^V \wedge X)    
    \]
    to be the resulting isomorphism.
\end{definition}

\begin{lemma}
    Let $E \colon (\AVB^\vop_G|_{\An_G^\omega})^\op \to \Ab$ be a genuine cohomology theory.
    Then $\tilde E$ is a $G$-cohomology theory.
\end{lemma}
\begin{proof}
    % TODO Proof
    By definition of the Mayer-Vietoris axiom we have that $\tilde E(-, \R^\bullet \oplus V)$ is an ordinary cohomomology theory 
    for every $V$.
    We will shows that $\sigma_{U\oplus V} \simeq \sigma_U \circ \sigma_V$ holds. Then we are finished since the axiom 
    $\sigma_U \circ \sigma_V \simeq \sigma_V \circ \sigma_U$ then follows from this and the functoriality of 
    $\sigma_{-}$ in $\AffLin^G$.
    The essential part in proving $\sigma_{U\oplus V} \simeq \sigma_U \circ \sigma_V$  comes from the commutativity of the square 
    \[\begin{tikzcd}
        {(X;W)} & {(S^V\times X;V \oplus W)} \\
        {(S^{U\oplus V}\times X;U\oplus V\oplus W)} & {(S^U\times S^V \times X;U\oplus V\oplus W)}
        \arrow["{(\pr,\theta_V)}"', from=1-2, to=1-1]
        \arrow["{(\pr,\theta_{U\oplus V})}", from=2-1, to=1-1]
        \arrow["{(q\times X;\id)}", from=2-2, to=2-1]
        \arrow["{(\pr,\theta_U)}"', from=2-2, to=1-2]
    \end{tikzcd}\]
    where $q\colon S^U \times S^V \to S^{U\oplus V}$ is the quotient map.
\end{proof}


\begin{definition}
Let $\cC$ be a symmetric monoidal category. Let $\cC //^\mathrm{lax} \cC^\simeq$ be the lax 
quotient of the action of the core $\cC^\simeq$ on $\cC$, which is defined as the unstraightening of the 
functor 
\[
B\cC^\simeq \to \catinfty \colon \pt \mapsto \cC.
\]
One can think of an object of $\cC //^\mathrm{lax} \cC^\simeq$ as an object of $\cC$, while a morphism 
\[
  c \to d  
\]
consists of the data of an object $e \in \cC$ and a morphism 
\[
  c \otimes e \to d  
\]
of $\cC$.
\end{definition}
\begin{definition}
Since every pair of objects $c,e \in \cC$ defines a functor 
\[
  \cC^\op(c) \xto{-\otimes \overrightarrow{e}} \cC^\op(c \otimes e)  
\]
and and every morphism $c \to d$ defines a morphism 
\[
  \cC^\op(c) \to \cC^\op(d)  
\]
in a compatible way we find that the assignment
\[
    c \mapsto \cC^\op(c)
\]
defines a functor 
\[
 \cC^\op(-) \colon \cC //^\mathrm{lax} \cC^\simeq \to \catinfty. 
\]
Since the forgetful functors $\cC^\op(c) \to \cC^\op$ are compatible with the structure maps of the diagram $\cC^\op(-)$ we obtain 
a natural transformation 
\[\cC^\op(-) \Rightarrow \mathrm{const_{\cC^\op}}.\]
\end{definition}

\begin{lemma}

\end{lemma}
