\section{Genuine Cohomology Theories}\label{section:Genuinecohomologytheories}

\begin{definition}[Genuine Cohomology Theories]
  
    A \emph{genuine cohomology theory} consists of a collection
  $E^V(X)$ of abelian groups, where $X$ runs trough all compact anima and
  $V$ is a vector bundle over $X$.
  For each map $f \colon X \to Y$ and a vector bundle $W$ over $Y$ we have induced
  maps
  $f^W \colon E^{W}(Y) \to E^{f^*W}(X)$.
  For each map $\phi \colon V \to W$ of vector bundles over $X$ we have
  induced maps
  $\phi_X \colon E^{V}(X) \to E^{W}(X)$.
  The usual functoriality conditions hold, that is
  \begin{itemize}
    \item If $f, \phi$ above are the identity maps then they induce the identity.
    \item We have $(g \circ f)^V = f^{g^*V} \circ g^V$ and
    $(\phi \circ \psi)_X = \phi_X \circ \psi_X$.
  \end{itemize}
  
  We require these collection of abelian groups and induced maps to satisfy some
  properties
  
  \begin{enumerate}
    \item[(i)]\label{homotopyinvariance} (Homotopy Invariance)\\
     If $\phi \simeq \psi$ are homotopic morphisms of vector bundles $V \to W$ over $X$, then 
     the induced morphisms $\phi_X = \psi_X$ are equal.
     If $H \colon f \simeq g$ is a homotopy, then it induces an equivalence 
     $\phi_H \colon f^*V \simeq g^*V$ between vector bundles over $X$.
     We require that $({\phi_H})_X \circ f^V = g^V$, i.e. the following commutes 
     \[\begin{tikzcd}
      && {E^{f^*V}(X)} \\
      {E^V(X)} \\
      && {E^{g^*V}(X)}
      \arrow["{({\phi_H})_X}", from=1-3, to=3-3]
      \arrow["{f^V}", from=2-1, to=1-3]
      \arrow["{g^V}"', from=2-1, to=3-3]
    \end{tikzcd}\]
    \iffalse
    \item[(ii)]\label{additivity} (Additivity)\\
    For each collection $(X_i,V_i)_{i \in I}$
    of anima $X_i$ and vector bundles $V_i$ over $X_i$ the induced homomorphism
    \[
      E^{\coprod_i V_i}(\coprod_i X_i) \to \prod_i E^{V_i}(X_i)
      \]
      is an isomorphism.
    \fi 
      \item[(ii)]\label{beckchavalley} (Beck-Chevalley)\\
      Let $f \colon X \to Y$ be
      a map of anima and $\phi \colon V \to W$ be a map of vector bundles over $Y$.
      Let $f^*\phi$ be the induced map $f^* V \to f^* W$ of vector bundles over $X$.
      Then the following square commutes
      \[\begin{tikzcd}
        {E^{V}(Y)} & {E^{f^*V}(X)} \\
        {E^W(Y)} & {E^{f^*W}(X)}
        \arrow["{\phi_Y}"', from=1-1, to=2-1]
        \arrow["{f^W}"', from=2-1, to=2-2]
        \arrow["{f^V}", from=1-1, to=1-2]
        \arrow["{(f^*\phi)_X}", from=1-2, to=2-2]
      \end{tikzcd}\]
  
    \item[(iii)]\label{reduced} (Reducedness)\\
      Let $X$ be a compact anima and 
      let $\emptyset$ be the empty bundle over $X$. Then 
      $E^\emptyset(X) \cong 0$.
    \item[(iv)]\label{mayervietoris} (Mayer-Vietoris)\\
    For each pushout square of anima
    \[\begin{tikzcd}
      {X_1} & {X_2} \\
      {X_3} & {X_4}
      \arrow["{f_1}", from=1-1, to=1-2]
      \arrow["{f_3}", from=1-2, to=2-2]
      \arrow["{f_2}"', from=1-1, to=2-1]
      \arrow["{f_4}"', from=2-1, to=2-2]
      \arrow["\lrcorner"{anchor=center, pos=0.125, rotate=180}, draw=none, from=2-2, to=1-1]
      \arrow["H"{description}, Rightarrow, from=2-1, to=1-2]
    \end{tikzcd}\]
    and vector bundle $V$ over $X_4$ the induced sequence
    \[
      E^{V}(X_4) \xto{(f_3^V,f_4^V)} E^{f_3^*V}(X_2)\oplus E^{f_4^*V}(X_3)
      \xto{f_1^{f_3^*V} - ({\phi_H})_X \circ f_2^{f_4^*V}} E^{f_1^*f_3^*V}(X_1)
      \]
    is exact in the middle.
    \item[(v)]\label{thomiso}(The Thom Isomorphism)\\
    Let $V,W$ be two vector bundles over $X$. Let
    \[E^{V\oplus W}(S_X^V,X) =
    E^{p^*(V\oplus W)}(S_X^V) \ominus E^{V\oplus W}(X)
    \] be the complement of
    the direct summand inside $E^{p^*(V\oplus W)}(S_X^{V})$ of $E^{V\oplus W}(X)$
    due to the retraction $\sigma^{p^*(V\oplus W)} \circ p^{V\oplus W} = \id$.
    Then the map
    \[
    E^W(X) \xto{p^W} E^{p^*W}(S_X^V) \xto{(\theta_{(X)}^{(V,W)})_{S_X^V}} E^{p^*(V \oplus W)}(S_X^V) \to E^{V\oplus W}(S_X^V,X)
    \]
    is an isomorphism.
  \end{enumerate}
  
  \end{definition}
  
  We want to make the above definition rigorous, by observing that some of the statet 
  axioms of a genuine cohomology theory are implemented by a functor 
  out of $h\AVB^\op$. In general we can describe the homotopy category of the 
  total category of a Cartesian fibration as follows:
  
  \begin{remark}
  Let $p \colon \cC \to \mathcal I$ be a Cartesian fibration. We want to give a generators and 
  relation discription 
  of the homotopy category of $\cC$ in terms of the functor ${\mathrm{d}} p \colon \mathcal I^\op \to \catinfty$ that 
  classifies $p$. 
  First, consider the functor $h {\mathrm{d}} p \colon \mathcal I^\op \to \catinfty$, that 
  we obtain by postcomposing ${\mathrm{d}} p$ with the functor $h \colon \catinfty \to \catinfty$
  that sends an $\infty$-category $\cD$ to its homotopy category $h\cD$.
  The natural transformation $\id_{\catinfty} \to h$ induces a functor 
  $h\cC \to h\int_{\mathcal I^\op} h {\mathrm{d}} p$.
  We claim that this functor induces an equivalence between ordinary categories.
  TODO 
  Since $h {\mathrm{d}} p$ maps into the full subcategory $\mathcal{C}\mathrm{at}_1$ of 
  $\catinfty$ of ordinary categories, which is a $(2,1)-category$, the functor $h {\mathrm{d}} p$
  factorizes over the homotopy-$2$ category $\tau_{\leq 2} \mathcal I$. 
  In fact, the induced functor $\int_{\mathcal I^\op} h {\mathrm{d}} p 
  \to \int_{\mathcal \tau_{\leq 2}I^\op} h {\mathrm{d}} p $ is an 
  equivalence of categories.
  We conclude, that we can compute the homotopy category $h\cC$ by computing 
  $h \int_{\tau_{\leq 2}} h {\mathrm{d}} p$, i.e. the homotopy category 
  of the Grothendieck construction of the $2$-functor from the homotopy-$2$ category of 
  $\mathcal I^\op$ to the $(2,1)$-category of $1$-categories, given by the 
  fiber wise homotopy category of the straightening of $p$. 
  Luckily, there is a formula for that construction:
  
  An object of $h\cC$ is a pair $(i, [c])$ consisting of an object $i \in \mathcal I$
  and an equivalence class of objects $[c] \in h {\mathrm{d}} p(i)$.
  
  A morphism $(i, [c]) \to (j, [d])$ is an equivalcence class represented by a pair $(f, [\phi])$ consisting 
  of a morphism 
  $f \colon j \to i$ in $\mathcal I$ and homotopy class of morphisms $[\phi \colon c \to f^*d]$. 
  Two of such pairs $(f, [\phi])$ and $(g, [\psi])$ get identified if there exists 
  a homotopy $H \colon f \to g$ in $\mathcal I$ such that with the induced natural 
  isomorphism $H^* \colon f^* \simeq g^*$ between functors 
  $h {\mathrm{d}} p(j) \to h {\mathrm{d}} p(i)$
  the post composition $c \xto{[\phi]} f^*d \xto{H^*} g^*d$ is equal to the class 
  $[\psi]$.
  We can even further simplify the the discription of $h\cC$ by abusing the fact 
  that we can factorize every morphism in $\cC$ in a fiber wise morphism followed by a Cartesian 
  morphism. Let $(f,[\phi]) \colon (i,c) \to (j, d)$ represent a morphism in $h \cC$.
  Then we can write it as the composite $(f,[\phi]) = (f, [\id]) \circ (\id, [\phi])$
  \[\begin{tikzcd}
      {(i,c)} && {(j,d)} \\
      & {(i,f^*d)}
      \arrow["{(f,[\phi])}", from=1-1, to=1-3]
      \arrow["{(\id,[\phi])}"', from=1-1, to=2-2]
      \arrow["{(f,\id)}"', from=2-2, to=1-3].
  \end{tikzcd}\]
  The composition of two such decompositions $(f,[\phi]) = (f, [\id]) \circ (\id, [\phi])$ and 
  $(g,[\psi]) = (g, [\id]) \circ (\id, [\psi])$ is given by the decomposition $(\id, [\psi \phi])\circ (\id, gf)$
  \[\begin{tikzcd}
      {(i,c)} \\
      {(i,f^*d)} & {(j,d)} \\
      {(i,f^*g^*e)} & {(j, g^*e)} & {(k,b)}
      \arrow["{(\id,[\phi])}"', from=1-1, to=2-1]
      \arrow["{(f,\id)}"', from=2-1, to=2-2]
      \arrow["{(\id,[\psi])}", from=2-2, to=3-2]
      \arrow["{(g,\id)}"', from=3-2, to=3-3]
      \arrow["{(\id,[\psi])}"', dashed, from=2-1, to=3-1]
      \arrow["{(f,\id)}"', dashed, from=3-1, to=3-2]
  \end{tikzcd}\].
  Thus a functor $F \colon h\cC \to \cD$ into an ordinary category consists of the data 
  \begin{itemize}
    \item A specified object $F(i,c) \in \cD$ for each pair $i \in \mathcal I$ and $c \in  h {\mathrm{d}} p(i)$.
    \item A morphism $F(f) \colon F(i,f^*d) \to F(j,d)$ for every morphism $f \colon i \to j$ in $\cC$ and fixed object $d \in h {\mathrm{d}} p(j)$.
    \item A morphism $F([\phi]) \colon F(i, c) \to F(i, d)$ for every homotopy class of morphisms $[\phi \colon c \to d]$ in $ h {\mathrm{d}} p(i)$.
    \item A specified natural isomorphism $\epsilon_{H,d} \colon F(i,f^*d) \xto{\cong} F(i, g^*d)$ for every homotopy $H \colon f \simeq g$ between morphisms 
    $ i \to j $ in $\cC$ and object $d \in h {\mathrm{d}} p(j)$
  \end{itemize}
  such that 
  \begin{itemize}
    \item (Beck-Chavalley) Both composites agree $F(\phi) \circ F(f) = F(f) \circ F(f^*\phi)$
    \[\begin{tikzcd}
      {F(i,f^*c)} & {F(j, c)} \\
      {F(i,f^*d)} & {F(j,d)}
      \arrow["{F([f^*\phi])}"', from=1-1, to=2-1]
      \arrow["{F(f)}"', from=2-1, to=2-2]
      \arrow["{F(f)}", from=1-1, to=1-2]
      \arrow["{F([\phi])}", from=1-2, to=2-2]
    \end{tikzcd}\]
    \item $F(f) = F(g) \circ \epsilon_{H,d}$
    \[\begin{tikzcd}
      {F(i,f^*d)} \\
      && {F(j,d)} \\
      {F(i,g^*d)}
      \arrow["{\epsilon_{H,d}}"', from=1-1, to=3-1]
      \arrow["{F(f)}", from=1-1, to=2-3]
      \arrow["{F(g)}"', from=3-1, to=2-3]
    \end{tikzcd}\]
  \end{itemize}
  \end{remark}
  With the last remark we can now give an equivalent definition of a 
  genuine cohomology theory. 
  
  \begin{definition}[Genuine Cohomology Theories II]\label{truedef}  
    A \emph{genuine cohomology theory} $E$ is a functor $E \colon h (\AVB^\op) \to \mathrm{Ab} \colon (X,V) \mapsto E^V(X)$
    that satisfies the following axioms 
    \begin{enumerate}
      \item[(iii)](Reducedness)\\
      Let $X$ be a compact anima and 
      let $\emptyset$ be the empty bundle over $X$. Then 
      $E^\emptyset(X) \cong 0$.
    \item[(iv)](Mayer-Vietoris)\\
    For each pushout square of anima
    \[\begin{tikzcd}
      {X_1} & {X_2} \\
      {X_3} & {X_4}
      \arrow["{f_1}", from=1-1, to=1-2]
      \arrow["{f_3}", from=1-2, to=2-2]
      \arrow["{f_2}"', from=1-1, to=2-1]
      \arrow["{f_4}"', from=2-1, to=2-2]
      \arrow["\lrcorner"{anchor=center, pos=0.125, rotate=180}, draw=none, from=2-2, to=1-1]
      \arrow["H"{description}, Rightarrow, from=2-1, to=1-2]
    \end{tikzcd}\]
    and vector bundle $V$ over $X_4$ the induced sequence
    \[
      E^{V}(X_4) \xto{(f_3^V,f_4^V)} E^{f_3^*V}(X_2)\oplus E^{f_4^*V}(X_3)
      \xto{f_1^{f_3^*V} - ({\phi_H})_X \circ f_2^{f_4^*V}} E^{f_1^*f_3^*V}(X_1)
      \]
    is exact in the middle.
    \item[(v)](Thom Isomorphism)\\
    Let $V,W$ be two vector bundles over $X$.
    \iffalse 
     Let
    \[E^{V\oplus W}(S_X^V,X) =
    E^{p^*(V\oplus W)}(S_X^V) \ominus E^{V\oplus W}(X)
    \] be the complement of
    the direct summand inside $E^{p^*(V\oplus W)}(S_X^{V})$ of $E^{V\oplus W}(X)$
    due to the retraction $\sigma^{p^*(V\oplus W)} \circ p^{V\oplus W} = \id$.
    \fi 
    Then the map
    \[
    E^W(X) \xto{p^W} E^{p^*W}(S_X^V) \xto{(\theta_{(X)}^{(V,W)})_{S_X^V}} E^{p^*(V \oplus W)}(S_X^V) \to E^{V\oplus W}(S_X^V,X)
    \]
    is an isomorphism.
    \end{enumerate}
  \end{definition}  
  
  \begin{remark}
    Our motivation for this specific equivalent formutlation of the Eilenberg-Steenrod axioms for cohomology theories is, that it is generic enough, to 
    translate it to different contexts, e.g. equivariant homotopy theory, but it is specific enough to model the `correct' notion of cohomology theories
    instead of naive ones. We devoted the last chapter to propose a definition for equivariant genuine cohomology theories.
  \end{remark}
  
  \begin{definition}
    A \emph{cohomology theory of finite spectra} $E$ is a functor 
    $ (h \Sp^\omega)^\op \to \mathrm{Ab}$ that satisfies the following axioms:
    \begin{enumerate}
      \item[(i)](Reducedness) $E(0) \simeq 0$
      \item[(ii)](Exactness in the middle) Let $X \xto{f} Y \xto{g} Z$ be 
      a fiber sequence of finite spectra and let $a \in \ker E(f)$, 
      then there exists a $b \in E(Z)$ such that $E(g)(b) = a$.
    \end{enumerate}
  \end{definition}
  
  \begin{proposition}[Adams version of Brown Representability]
    Let $E$ be a cohomology theory of finite spectra. Then there exists a spectrum 
    $\mathcal E$ and a natural isomorphism $[-, \mathcal E] \xto{\simeq} E$ of functors 
    $(h \Sp^\omega)^\op \to \mathrm{Ab}$.
    It follows that the spectrum $\mathcal E$ is necesseraly unique.
  \end{proposition}

  \begin{proposition}
    The functor $E \mapsto E \circ \Th^-$ is an equivalence between the category
    \[\{\mathrm{Cohomology\ Theories\ on\ }\Sp^\omega\}\] 
    of cohomology theories on compact spectra and 
    the category \[
        \{\mathrm{Genuine\ Cohomology\ Theories}\}
       \] of genuine cohomology theories.
    In particular exists vor every genuine cohomology theory $E$ a unique spectrum $\mathcal E$ and a natural 
    equivalence 
    \[
      E^V(X) \simeq [X^{-V}, \mathcal E]  .
    \]
  \end{proposition}

\begin{proof}
    The category of genuine cohomology theories is modeled 
    by the full subcategory 
    \[
      \Fun^{\mathrm{M.V., d.s., red.}}(\AVB^\vop|_{\An^\omega}, \mathrm{Ab}^\op)^\op  
    \]
    of the functor category $\Fun(\AVB^\vop|_{\An^\omega}, \mathrm{Ab}^\op)^\op$ on those functors 
    $E \colon \AVB^\vop|_{\An^\omega} \to \mathrm{Ab}^\op$ that satisfies the axioms of Definition~\ref{truedef}.
    By Theorem~\ref{thrm:dream} the functor $E \mapsto E \circ \Th^-$ identifies the category of genuine cohomology 
    theories with a full subcategory of 
    \[
        \Fun^{\mathrm{add.}}(\Sp^\omega, \mathrm{Ab}^\op)^\op.
    \]
    We claim that this subcategory is precisely the one which is spanned by the cohomology theories.
    To show this we have to prove two things 
    \begin{itemize}
        \item Let $\mathcal E$ be a spectrum, then the functor 
            \[
                [\Th^-(-), \mathcal E] \colon \AVB^\vop|_{\An^\omega} \to \mathrm{Ab}^\op
            \]
            satisfies the Mayer-Vietoris axiom of Definition~\ref{truedef}.
        \item Let $E$ be a genuine cohomology and let $\tilde E \colon \Sp^\omega \to \Ab^\op$
              be the unique functor such that $E \simeq \tilde E \circ \Th^-$ holds, according to Theorem~\ref{thrm:dream}.
              Then $\tilde E$ is a cohomology theory.

    \end{itemize}
    The first claim follows from the fact that the functor $\Th^-$ sends pushout squares with constant coefficients 
    to pushout squares in spectra. 
    % \[\begin{tikzcd}
    %     {(A,V)} & {(B,V)} \\
    %     {(C,V)} & {(D,V)}
    %     \arrow[from=1-1, to=1-2]
    %     \arrow[from=1-2, to=2-2]
    %     \arrow[from=1-1, to=2-1]
    %     \arrow[from=2-1, to=2-2]
    % \end{tikzcd}\]
    The second claim follows from the fact that any pushout square
    \[\begin{tikzcd}
        {\mathcal A} & {\mathcal B} \\
        {\mathcal C} & {\mathcal D}
        \arrow[from=1-1, to=1-2]
        \arrow[from=1-2, to=2-2]
        \arrow[from=1-1, to=2-1]
        \arrow[from=2-1, to=2-2]
        \arrow["\lrcorner"{anchor=center, pos=0.125, rotate=180}, draw=none, from=2-2, to=1-1]
    \end{tikzcd}\]
     is of the form 
    \[\begin{tikzcd}
        {\Sigma^{\infty - n}A} & {\Sigma^{\infty - n}B} \\
        {\Sigma^{\infty - n}C} & {\Sigma^{\infty - n}D}
        \arrow[from=1-1, to=1-2]
        \arrow[from=1-2, to=2-2]
        \arrow[from=1-1, to=2-1]
        \arrow[from=2-1, to=2-2]
        \arrow["\lrcorner"{anchor=center, pos=0.125, rotate=180}, draw=none, from=2-2, to=1-1],
    \end{tikzcd}\]
    where 
    \[\begin{tikzcd}
        A & B \\
        C & D
        \arrow[from=1-1, to=1-2]
        \arrow[from=1-2, to=2-2]
        \arrow[from=1-1, to=2-1]
        \arrow[from=2-1, to=2-2]
        \arrow["\lrcorner"{anchor=center, pos=0.125, rotate=180}, draw=none, from=2-2, to=1-1]
    \end{tikzcd}\]
    is a pushout in $\An_*^\omega$.
    Consider the following pushout in $\AVB^\vop|_{\An^\omega}$:
    \[\begin{tikzcd}
        {(A,\R^n)} & {(B,\R^n)} \\
        {(C,\R^n)} & {(D,\R^n)}
        \arrow[from=1-1, to=1-2]
        \arrow[from=1-2, to=2-2]
        \arrow[from=1-1, to=2-1]
        \arrow[from=2-1, to=2-2]
        \arrow["\lrcorner"{anchor=center, pos=0.125, rotate=180}, draw=none, from=2-2, to=1-1]
    \end{tikzcd}.\]
    By the (Mayer-Vietoris) axiom it gets send to a sequence 
    \[
    E^n(D) \to E^n(B) \oplus E^n(C) \to E^n(A)    
    \]
    which is exact in the middle. Therefore also the sequence 
    \[
    \tilde E(\Sigma^{\infty - n} D) \to \tilde E(\Sigma^{\infty - n} B) \oplus \tilde E(\Sigma^{\infty - n} C)  \to \tilde E(\Sigma^{\infty - n} A)     
    \]
    is exact in the middle since we have 
    \[
    \tilde E(\Sigma^{\infty -n} X) \simeq \fib( E^n X \to E^n \pt)
    \]
    for any pointed finite space $X$.
\end{proof}
