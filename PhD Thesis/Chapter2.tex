\section{The Category of Spectra generated by Animae and Vector Bundles}\label{section:spectraGenerated}


\begin{definition}
    % Here definition of pushouts with constant coeffficients

\end{definition}

\begin{lemma}
    The functor $\Th \colon \AVB \to \An_*$ sends split pushouts with constant coefficients to pushouts.    
\end{lemma}

\begin{proposition}
    The functor $\mathrm{Th}^- \colon \AVB^\vop \to \Sp$ that sends a pair $(X,V)$ to the 
    Thom spectrum $X^{-V}$ of the negative bundle of $V$ is a functor that sends distinguished squares to pushouts.
\end{proposition}

\begin{proof}
    We have to show that 
    \[\begin{tikzcd}
        0 & {X^{-V}} \\
        {X^{-V\oplus W}} & {(S^{W})^{-V\oplus W}}
        \arrow[to=1-2, from=1-1]
        \arrow[to=1-1, from=2-1]
        \arrow[to=1-2, from=2-2]
        \arrow[to=2-2, from=2-1]
    \end{tikzcd}\]
    is a pushout.
    This square is the image of the square in $T\An$:
    \[\begin{tikzcd}
        {0_X} & {\mathbb S^{-V}_X} \\
        {\mathbb S_X^{-V\oplus W}} & {\pi_! \pi^* \mathbb S_X^{-V\oplus W}}
        \arrow[to=1-2, from=1-1]
        \arrow[to=1-1, from=2-1]
        \arrow[to=1-2, from=2-2]
        \arrow[to=2-2, from=2-1]
    \end{tikzcd}\]
    under the colimit preserving functor $T\An \to \Sp$ that `carries out the colimit'.
    By the projection formula this square is equivalent to the square 
    \[\begin{tikzcd}
        {0_X} & {\mathbb S^{-V}_X} \\
        {\mathbb S_X^{-V} \otimes \mathbb S_X^{-W}} & {\mathbb S_X^{-V} \otimes \mathbb S_X^{-W}\otimes\pi_! \pi^* \mathbb S_X}
        \arrow[to=1-2, from=1-1]
        \arrow[to=1-1, from=2-1]
        \arrow[to=1-2, from=2-2]
        \arrow[to=2-2, from=2-1]
    \end{tikzcd}.\]
    Since the functor $\AVB^\vop \to T\An$ is symmetric monoidal we see that the above square is the tensor product of the image of 
    the simple distinguished square with $\mathbb S_X^{-V}$. So we can assume $V = 0$. So we want to prove that 
\[\begin{tikzcd}
	{0_X} & {\mathbb S_X} \\
	{\mathbb S_X^{-W}} & {\mathbb S_X^{-W}\otimes\pi_! \pi^* \mathbb S_X}
	\arrow[to=1-2, from=1-1]
	\arrow[to=1-1, from=2-1]
	\arrow[to=1-2, from=2-2]
	\arrow[to=2-2, from=2-1]
\end{tikzcd}\]
is a pushout square in $T\An$. This is a pushout if and only if for each point inclusion $x \colon \pt \to X$ the pullback 
\[\begin{tikzcd}
	{x^*0_X} & {x^*\mathbb S_X} \\
	{x^*\mathbb S_X^{-W}} & {x^*(\mathbb S_X^{-W}\otimes\pi_! \pi^* \mathbb S_X)}
	\arrow[to=1-2, from=1-1]
	\arrow[to=1-1, from=2-1]
	\arrow[to=1-2, from=2-2]
	\arrow[to=2-2, from=2-1]
\end{tikzcd}\]
is a pushout. 
So we can assume that $X = \pt$. That is we have to show that 
\[\begin{tikzcd}
	0 & {\mathbb S} \\
	{\mathbb S^{-W}} & {\mathbb S^{-W}\otimes \colim_{S^W} \mathbb S}
	\arrow[to=1-2, from=1-1]
	\arrow[to=1-1, from=2-1]
	\arrow[to=1-2, from=2-2]
	\arrow[to=2-2, from=2-1]
\end{tikzcd}\]
is a pushout in spectra for each sphere $S^W$.
Let us now try to understand the non-trivial maps in that diagram.
The bottom map comes from the point inclusion at $\infty \in S^W$, namely its the map 
\[
\mathbb S^{-W} \otimes \colim_{\pt} \mathbb S \xto{\mathbb S^{-W}\otimes \colim_{\infty} \mathbb S} \mathbb S^{-W} \otimes \colim_{S^W} \mathbb S.
\]
and the right map composes as 
\[
\mathbb S^{-W} \otimes \colim_{S^W} \mathbb S \xto{\mathbb S^{-W}\otimes \colim_{S^W} \Sigma^\infty \theta} \mathbb S^{-W} \otimes \colim_{S^W} \mathbb S^W \simeq \colim_{S^W}\mathbb S \xto{\colim_\pi \mathbb S} \colim_\pt \mathbb S    
\]
By tensoring the above square with the invertible object $\mathbb S^W$ we can instead show that
\[\begin{tikzcd}
	0 & {\mathbb S^W} \\
	{\colim_\pt \mathbb S} & {\colim_{S^W} \mathbb S}
	\arrow[from=1-1, to=1-2]
	\arrow[from=2-1, to=1-1]
	\arrow["{\colim_\infty \mathbb S}"', from=2-1, to=2-2]
	\arrow["{\colim_{\pi}\Sigma^\infty\theta}"', from=2-2, to=1-2]
\end{tikzcd}\]
is a pushout square.
This square is the image under the functor $\Sigma^\infty$ of the square 
\[\begin{tikzcd}
	\pt & {S^W} \\
	{\pt_+} & {S^W_+}
	\arrow["\infty", from=1-1, to=1-2]
	\arrow[from=2-2, to=1-2]
	\arrow[from=2-1, to=1-1]
	\arrow["{\infty_+}"', from=2-1, to=2-2]
\end{tikzcd}\]
where the right map is the composite 
\[
S^W_+ \xto{\theta} S^W_+ \wedge S^W \xto{\pi \wedge S^W} \pt_+ \wedge S^W.    
\]
We claim that $\theta$ is given by the diagonal 
\[
S^W_+ \xto{\Delta} S^W_+ \times S^W_+ \xto{\mathrm{can}} S^W_+ \wedge S^W.
\]
This clearly shows that the right map of the square above is given by the identity on the $S^W$ component, which 
makes the square a pushout.
By tracing back the definition we find that $\theta$ is the canonical map 
\[
  \colim^{\in \An_*}_{\Map((\pt,0),(\pt,W))}  \Map(((\pt,0),(\pt,0))) \xto{\colim_f (g \mapsto f \circ g)} \colim^{\in \An_*}_{\Map((\pt,0),(\pt,W))} \Map(((\pt,0),(\pt,W)))
\]
which is equivalent to the map above.
\end{proof}

\begin{remark} 
    The last proposition shows that we have natural cofiber sequences 
    \[
      X^{-V \oplus W} \xto{ \sigma_\infty^{- V \oplus W}} (S^W)^{-V \oplus W} \xto{\pi^{-\theta}} X^{-V}.
    \]
    These split natural as the left map admits a left inverse via 
    \[
    \pi^{-V\oplus W} \colon (S^W)^{-V \oplus W} \to X^{-V \oplus W}.
    \]
    So we have a natural decomposition 
    \[
        (S^W)^{-V \oplus W} \simeq  X^{-V \oplus W} \oplus X^{-V}.
    \]
    Suppose $V \oplus W \simeq \R^n_X$ is a trivial bundle, than we have a square in $\AVB^\vop$
    \[\begin{tikzcd}
        {(X,\R^n_X)} & {(S^W,\R^n_{S^W})} \\
        {(\pt, \R^n)} & {(\mathrm{Th}(W),\R^n_{\mathrm{Th}(W)})}
        \arrow["{(r,\id)}"', from=1-1, to=2-1]
        \arrow["{(\infty,\id)}"', from=2-1, to=2-2]
        \arrow["{(\sigma_\infty, \id)}", from=1-1, to=1-2]
        \arrow["{(\mathrm{can}, \id)}", from=1-2, to=2-2]
    \end{tikzcd}\]
    which get send to a pushout via $\Th^-$ since it is a pushout in the anima coordinate and constant in the vector bundle coordinate.
    Thus combining with the previous result we have a natural equivalence 
    \[
      X^{-V} \to \Sigma^{\infty - n}\Th(W)  
    \]
    once we fix a choice of a complement bundle $W$ of $V$. 
\end{remark}

We have seen that $\Th^-$ respects pushout squares in the anima coordinate and sends distinguished squares to 
pushouts. The next theorem will show that it is the universal functor that does this at least over compact anima. 

\begin{definition}
    Let $\cD$ be a category. We call a pushout square
    \[\begin{tikzcd}
        A & B \\
        C & D
        \arrow["f", from=1-1, to=1-2]
        \arrow["g"', from=1-1, to=2-1]
        \arrow[from=2-1, to=2-2]
        \arrow[from=1-2, to=2-2]
        \arrow["\lrcorner"{anchor=center, pos=0.125, rotate=180}, draw=none, from=2-2, to=1-1]
    \end{tikzcd}\]
    in $\cD$ \emph{split} if at least one of the maps $f$ or $g$ admits a retraction.
    We say that $\cD$ \emph{has all split pushouts} if for every span $B \xot{f} A \xto{g} C$, where either 
    $f$ or $g$ is a retraction, the pushout $B \coprod_A C$ exists.
\end{definition}

\begin{theorem}\label{thrm:dream}
    
    For any pointed $\infty$-cagegory $\cD$
    with split pushouts the restriction along the functor
    $\Th^- \colon \cX|_{\An^\omega} \to \Sp^\omega, (X,V) \mapsto X^{-V}$
    induces a functor
    $ (\Th^-)^* \colon \Fun(\Sp^\omega, \cD) \to \Fun(\cX|_{\An^\omega} , \cD)$
    which is fully faithful when restricted to the subcategory consisting of those
    functors $\Sp^\omega \to \cD$ which preserve split pushouts and are reduced.
    The essential image of that subcategory consists precisely of those functors
    $\Phi \colon \cX|_{\An^\omega}  \to \cD$ which
    \begin{enumerate}
      \item send distinguished squares to pushouts,
      \item send $(X,\emptyset)$ for all compact anima $X$ to the zero object,
      \item send squares which are constant in the bundle coordinate and split pushouts in the anima variable to pushouts. \footnote{By this we strictly mean a square that is a Cartesian lift.}
    \end{enumerate}
\end{theorem}

\begin{lemma}\label{lemma:reduction}
    The following diagram commutes 
    \[\begin{tikzcd}
        {\AVB^\vop(n)|_{\An^\omega}} && {} && {\An_*^\omega} \\
        & {\AVB^\vop|_{\An^\omega}} & {\Sp^\omega} \\
        {\AVB^\vop(n+1)|_{\An^\omega}} &&&& {\An_*^\omega}
        \arrow["{\Th^\perp}", from=1-1, to=1-5]
        \arrow["{\Th^\perp}"', from=3-1, to=3-5]
        \arrow["{\oplus \overrightarrow{\R}}"', from=1-1, to=3-1]
        \arrow["{\Th^-}", from=2-2, to=2-3]
        \arrow["\Sigma", from=1-5, to=3-5]
        \arrow["{\Sigma^{\infty-n}}", from=1-5, to=2-3]
        \arrow["{\Sigma^{\infty-(n+1)}}", from=3-5, to=2-3]
        \arrow["{\overleftarrow{\pr}}"', from=1-1, to=2-2]
        \arrow["{\overleftarrow{\pr}}", from=3-1, to=2-2]
    \end{tikzcd}\]
\end{lemma}

\begin{proof}
    We show that the outer and the upper square commute. The rest is clear.
    \begin{itemize} 
        \item 
        We start by showing the commutativity of the outer square.
        We observe that we can factor the vertical functors as follows 
        \[
            \AVB^\vop(n)|_{\An^\omega} \xto{\overrightarrow{\pr}} \AVB|_{\An^\omega} \to \int_{X \in \An^\omega} \Fun(X,\An_*^\omega) \xto{\colim} \An_*^\omega 
        \]
    \end{itemize} 
    Where the middle functor is induced from 
    \[
        S = \Map((\pt,0), -) \colon \AffLin \to \An_*^\omega.  
    \]  
    Since $\overrightarrow \pr \circ (\oplus \overrightarrow{\R}) \simeq  (\oplus \R )\circ \overrightarrow{\pr}$ 
    and $\oplus \R$ is also induced from its restriction to $\AffLin$ we only need to show that 
    \[\begin{tikzcd}
        \AffLin & {\An^\omega_*} \\
        \AffLin & {\An^\omega_*}
        \arrow["{\oplus \R}"', from=1-1, to=2-1]
        \arrow["\Sigma", from=1-2, to=2-2]
        \arrow["S", from=1-1, to=1-2]
        \arrow["S"', from=2-1, to=2-2]
    \end{tikzcd}\]
    commutes, which it oviously does.
    \item Now we show that the upper square commutes. 
    A reduction argument like before shows that we only need to show that the following diagram commutes
    \[\begin{tikzcd}
        && {\AffLin^\op(n)} \\
        {\AffLin^\op} &&&& \AffLin \\
        & {{\Sp^\omega}{^\op}} && {\Sp^\omega}
        \arrow["{\overleftarrow{\pr}}"', from=1-3, to=2-1]
        \arrow["{\overrightarrow{\pr}}", from=1-3, to=2-5]
        \arrow["{\Sigma^{\infty - n} S}", from=2-5, to=3-4]
        \arrow["{\mathbb D}"', from=3-2, to=3-4]
        \arrow["{\Sigma^\infty S}"', from=2-1, to=3-2]
    \end{tikzcd}\]
    that is we have to provide an equivalence between the functors 
    \[
    (\overleftarrow V, \overrightarrow{W}, \alpha \colon V \oplus W \to \R^n) \mapsto \mathbb D \Sigma^\infty S^V    
    \]
    and 
    \[
    (\overleftarrow V, \overrightarrow{W}, \alpha \colon V \oplus W \to \R^n) \mapsto \Sigma^{\infty - n} S^W.
    \]
    The morphism $\alpha$ induces a natural map 
    \[
    S^W \to \Map(S^V, S^n)    
    \]
    which induces a natural equivalence 
    \[
    \Sigma^\infty S^W \to \mathrm{map}(\Sigma^\infty S^V, \Sigma^\infty S^n)    
    \]
    which after delooping yields the desired equivalence.
\end{proof}

\begin{proposition}\label{prop:dream2}
    For any pointed $\infty$-cagegory $\cD$
    with split pushouts the restriction along the functor
    $\Th^\perp \colon \cX(n)|_{\An^\omega} \to \An_*^\omega, (X,\overleftarrow{V},\overrightarrow{W}, \alpha) \mapsto \Th_X(W)$
    induces a functor
    $ (\Th^\perp)^* \colon \Fun(\An_*^\omega, \cD) \to \Fun(\cX(n)|_{\An^\omega} , \cD)$
    which is fully faithful when restricted to the subcategory consisting of those
    functors $\An_*^\omega \to \cD$ which preserve split pushouts and are reduced.
    The essential image of that subcategory consists precisely of those functors
    $\Phi \colon \cX(n)|_{\An^\omega}  \to \cD$ which
    \begin{enumerate}
      \item send distinguished squares to pushouts,
      \item send $(X,\emptyset,\emptyset, \infty)$ for all compact anima $X$ to the zero object,
      \item send squares which are constant in the bundle coordinate and split pushouts in the anima variable to pushouts.
    \end{enumerate}
\end{proposition}

\begin{proof}[Proof of Theorem~\ref{thrm:dream}]
    \begin{eqnarray*}   
        \Fun^{\mathrm{d.s,red.,s.p.}}(\AVB^\vop, \mathcal D) & \stackrel{\mathrm{Lemma}~\ref{lemma:reduction}}{\simeq} & \lim_n 
        \Fun^{\mathrm{d.s,red.,s.p.}}(\AVB^\vop(n), \mathcal D) \\
        & \stackrel{\mathrm{Prop.}~\ref{prop:dream2}}{\simeq} & \lim_n \Fun^{\mathrm{s.p.}}(\An_*^\omega, \mathcal D) \\
        & \simeq & \Fun^{\mathrm{s.p.}}(\Sp^\omega, \mathcal D)
    \end{eqnarray*} 
\end{proof}
\begin{proof}[Proof of Proposition~\ref{prop:dream2}]
    Given a functor $\psi \colon \AVB^\vop(n) \to \mathcal D$ we define 
    the functor 
    \[
    \psi|_{\An_*^\omega} \colon \An_*^\omega \to \mathcal D    
    \]
    by letting it send a pointed anima $\pt \to X$ to the value 
    \[
    \psi|_{\An_*^\omega}(X) := \mathrm{cof}\left( \psi(\iotarnzero(\pt))\to \psi(\iotarnzero(X))\right).  
    \]
    We claim that the functor $\res_{\An_*^\omega} \colon \psi \mapsto \psi|_{\An_*^\omega}$ is inverse to the functor $(\Th^\perp)^*$ when restricted to 
    the full subcategories 
    $\Fun^{\mathrm{d.s,red.,s.p.}}(\AVB^\vop(n), \mathcal D)$ and 
    $\Fun^{\mathrm{red.,s.p.}}(\An_*^\omega, \cD)$ of the respective functor categories.

    We start with a reality check by showing the inclusions 
    
        \[ 
            \res_{\An_*^\omega}(\Fun^{\mathrm{red.,s.p.}}(\AVB^\vop(n), \mathcal D)) \subset \Fun^{\mathrm{red.,s.p.}}(\An_*^\omega, \cD),
        \]
         \[
            (\Th^\perp)^*(\Fun^{\mathrm{red.,s.p.}}(\An_*^\omega, \cD)) \subset \Fun^{\mathrm{d.s,red.,s.p.}}(\AVB^\vop(n), \mathcal D):
        \]


    \begin{itemize}
        \item The inclusion 
        \[
            \res|_{\An_*^\omega}(\Fun) \subset \Fun^\mathrm{red.}
        \]
        follows immediately from the definitions.
        \item The inclusion 
        \[
            \res|_{\An_*^\omega}(\Fun^\mathrm{s.p.}) \subset \Fun^{\mathrm{s.p.}}
        \]
        folows from the fact that the functor $\iotarnzero$ sends split pushouts to split pushouts that are constant in the vector bundle 
        coordinates.
        \item The inclusion 
        \[
            (\Th^\perp)^*(\Fun^\mathrm{red.}) \subset \Fun^\mathrm{red.}
        \]
        follows from the fact that $\Th_X(\emptyset) \simeq \pt$ for every anima $X$.
        \item The inclusion 
        \[
            (\Th^\perp)^*(\Fun^\mathrm{s.p.}) \subset \Fun^\mathrm{s.p., d.s.}
        \]
        follows immediately when we have shown that the functor $\Th \colon \AVB \to \An_*$ sends distinguished squares and squares that are split pushouts 
        with constant coefficients to pushouts. 
        Let 
        \[\begin{tikzcd}
            {(X,\emptyset)} & {(X,W)} \\
            {(X,0)} & {(S^W_X,0)}
            \arrow["{(\id,\infty)}", from=2-1, to=1-1]
            \arrow["{(\id,\infty)}", from=1-1, to=1-2]
            \arrow["{(\sigma_\infty,\id)}"', from=2-1, to=2-2]
            \arrow["{(\pi,\theta)}"', from=2-2, to=1-2]
        \end{tikzcd}\]
        be a simple distinguished square in $\AVB$.
        The image of this square under $S$ is given by the square 
        \[\begin{tikzcd}
            X & {S^W_X} \\
            {X \coprod X} & {S^W_X \coprod S^X_X}
            \arrow["{\mathrm{fold}}", from=2-1, to=1-1]
            \arrow["{\sigma_\infty}", from=1-1, to=1-2]
            \arrow["{\sigma_\infty \coprod \sigma_\infty}"', from=2-1, to=2-2]
            \arrow["{(\id, \sigma_\infty \pi)}"', from=2-2, to=1-2]
        \end{tikzcd}\]\footnote{The zigzag identity of the adjunction $\iota_0 \dashv S$ shows that the first coordinate of the right arrow is the identity.}
        After quotienting out the added points at $\infty$ we are left with the square 
        \[\begin{tikzcd}
            \pt & {\Th_X(W)} \\
            {X \coprod \pt} & {S^W_X \coprod \pt}
            \arrow[from=2-1, to=1-1]
            \arrow["\infty", from=1-1, to=1-2]
            \arrow["{\sigma_\infty \coprod \id}"', from=2-1, to=2-2]
            \arrow["{(\mathrm{can},\infty)}"', from=2-2, to=1-2]
        \end{tikzcd}\]
        which is obviously a pushout.

        Now let 
        \[\begin{tikzcd}
            {(A,W)} & {(B,W)} \\
            {(C,W)} & {(D,W)}
            \arrow[from=1-1, to=1-2]
            \arrow[from=2-1, to=2-2]
            \arrow[from=1-2, to=2-2]
            \arrow[from=1-1, to=2-1]
        \end{tikzcd}\]
        be a split pushout with constant coefficients. We want to show that 
        \[\begin{tikzcd}
            {\Th_A(W)} & {\Th_B(W)} \\
            {\Th_C(W)} & {\Th_D(W)}
            \arrow[from=1-1, to=1-2]
            \arrow[from=2-1, to=2-2]
            \arrow[from=1-2, to=2-2]
            \arrow[from=1-1, to=2-1]
        \end{tikzcd}\]
        is a pushout. By pulling back along points of $\cD$ we can assume $D \simeq \pt$. Then $W \simeq \R^n$ for some $n$.
        That is we look at the square 
        \[\begin{tikzcd}
            {(A,\R^n)} & {(B,\R^n)} \\
            {(C,\R^n)} & {(\pt,\R^n)}
            \arrow[from=1-1, to=1-2]
            \arrow[from=2-1, to=2-2]
            \arrow[from=1-2, to=2-2]
            \arrow[from=1-1, to=2-1].
        \end{tikzcd}\]
        The image under $\Th$ is 
        \[\begin{tikzcd}
            {\Sigma^n A_+} & {\Sigma^n B_+} \\
            {\Sigma^nC_+} & {\Sigma^nD_+}
            \arrow[from=1-1, to=1-2]
            \arrow[from=2-1, to=2-2]
            \arrow[from=1-2, to=2-2]
            \arrow[from=1-1, to=2-1]
        \end{tikzcd}\]
        which is clearly a pushout. 
        % Todo: Prove that in a lemma
    \end{itemize}

    To show that $\res|_{\An_*^\omega}$ is inverse to $(\Th^\perp)^*$ we construct functors $\mathcal F \colon \Fun(\An_*^\omega, \mathcal D) \to \Fun(\An_*^\omega, \mathcal D)$
    and $\mathcal G, \mathcal H \colon \Fun(\AVB^\vop(n), \mathcal D) \to \Fun(\AVB^\vop(n),\mathcal D)$ and natural transformations 
    \[
    \res|_{\An_*^\omega} (\Th^\perp)^* \xrightarrow{\rho_1} \mathcal F \xleftarrow{\rho_2}   \id_{\Fun(\An_*^\omega, \mathcal D)} 
    \]
    \[
        (\Th^\perp)^* \res|_{\An_*^\omega} \xleftarrow{\eta_1} \mathcal G \xto{\eta_2} \mathcal H \xot{\eta_3} \id_{\Fun(\AVB^\vop(n), \mathcal D)}   
    \]
    so that the natural transformations $\rho_i$ evaluete to equivalences on the full subcategory 
    $\Fun^{\mathrm{red.,s.p.}}(\An_*^\omega, \cD)$ and 
    likewise $\eta_i$ evaluate to equivalences on the subcategory 
    $\Fun^{\mathrm{d.s,red.,s.p.}}(\AVB^\vop(n), \mathcal D)$.

    Let $\mathcal F$ be the localization functor corresponding to the full subcategory of reduced functors. That is 
    \[
      \mathcal F (\phi) := \phi^{\mathrm{red}}\colon X \mapsto \mathrm{cof}\left(\phi(\pt) \to \phi(X)\right).
    \]
    It comes with a natural transformation $\rho_2 \colon \id \to \mathcal F$, that evaluates as the canonical map 
    $\phi(X) \to \phi^\mathrm{red}(X)$ into the cofiber.
    A simple calculation shows that 
    \[
    (\phi \circ \Th^\perp)|_{\An_*^\omega} (X) \simeq \mathrm{cof}(\phi(\pt_+) \to \phi(X_+)).
    \]
    The functorial split pushout 
    \[\begin{tikzcd}
        {\pt_+} & {X_+} \\
        \pt & X
        \arrow[from=1-1, to=1-2]
        \arrow[from=1-1, to=2-1]
        \arrow[from=2-1, to=2-2]
        \arrow[from=1-2, to=2-2]
        \arrow[bend right=30, dotted, from=1-2, to=1-1]
        \arrow[bend left=30, dotted, from=2-2, to=2-1]
    \end{tikzcd}\]
    induces a natural transformation 
    $ \rho_1 \colon \res|_{\An_*^\omega} (\Th^\perp)^* \to \mathcal F$.

    Let $\mathcal G$ be the functor that sends a functor $\psi \colon \AVB^\vop(n) \to \mathcal D$ to the functor 
    \[
    \mathrm{cof}\left(\psi\iotarnzero\pr \xto{\psi\iotarnzero\sigma_\infty}\psi\iotarnzero S^\perp\right)    
    \]
    and let $\mathcal H$ send $\psi$ to 
    \[
    \psi^\mathrm{red} := \mathrm{cof}(\psi(\infty) \colon \psi \iotaemptyempty \pr \to \psi).
    \]
    The evident map $\psi \to \psi^\mathrm{red}$ refines to a natural transformation 
    $\eta_3 \colon \id \to \mathcal H$.

    The square~\ref{fig:endosquareavb(n)} of endofucntors in Proposition~\ref{prop: AVB(n)} yields a functorial square 
    \[\begin{tikzcd}
        {\psi \iotaemptyempty} & \psi \\
        {\psi \iotarnzero \pr} & {\psi \iotarnzero S^\perp}
        \arrow[from=1-1, to=1-2]
        \arrow[from=2-2, to=1-2]
        \arrow[from=2-1, to=1-1]
        \arrow[from=2-1, to=2-2]
    \end{tikzcd}\]
    that induces a natural map $\mathcal G(\psi) \to \psi^\mathrm{red}$ that is a transformation $\eta_2 \colon \mathcal G \to \mathcal H$.
    Finally the split pushout square of functors 
    \[\begin{tikzcd}
        \pr & {S^\perp} \\
        {\mathrm{const}_{\pt}} & {\Th^\perp}
        \arrow["\sigma_\infty"', from=1-1, to=1-2]
        \arrow[from=1-2, to=2-2]
        \arrow[from=1-1, to=2-1]
        \arrow[from=2-1, to=2-2]
        \arrow[bend right=30, dotted, from=1-2, to=1-1]
        \arrow[bend left=30, dotted, from=2-2, to=2-1]
    \end{tikzcd}\]
    induces a square 
    \[\begin{tikzcd}
        {\psi \iotarnzero \pr} & {\psi \iotarnzero S^\perp} \\
        {\psi \iotarnzero \mathrm{const}_{\pt}} & {\psi \iotarnzero \Th^\perp}
        \arrow[from=1-1, to=1-2]
        \arrow[from=1-2, to=2-2]
        \arrow[from=1-1, to=2-1]
        \arrow[from=2-1, to=2-2]
    \end{tikzcd}\]
    that ultimately leads to a natural map $\mathcal G(\psi) \to \psi|_{\An_*^\omega} \circ \Th^\perp$ that refines 
    to a natural transformation 
    $ \eta_1 \colon \mathcal G \to (\Th^\perp)^* \res|_{\An_*^\omega}$.

    The subcategories of the functor categories in question are chosen precisely such that all of the transformations 
    $\eta_i, \rho_j$ are equivalences when restricted to them, that is:
    \begin{itemize}
        \item $\rho_1 \colon (\phi \circ \Th^\perp)|_{\An_*^\omega} \to \phi^\mathrm{red}$ is an equivalence if $\phi$ preserves split pushouts,
        \item $\rho_2 \colon \phi \to \phi^\mathrm{red}$ is an equivalence if $\phi$ is reduced,
        \item $\eta_1 \colon \mathcal G(\psi) \to \psi_{\An_*^\omega} \circ \Th^\perp$ is an equivalence if $\psi$ preserves split pushouts in the anima variable that are 
        constant on the vector bundle coordinates, 
        \item $\eta_2 \colon \mathcal G(\psi) \to \psi^\mathrm{red}$ is an equivalence if $\psi$ sends distinguished squares to pushouts,
        \item $\eta_3 \colon \psi \to \psi^\mathrm{red}$ is an equivalence if $\psi$ sends objects of the form $\iotaemptyempty(X)$ to the zero object.
    \end{itemize}

\end{proof}

\begin{remark}
    This proof also shows that 
    \[\begin{tikzcd}
        {\Fun^{\mathrm{red.,s.p.}}(\An_*^\omega,\mathcal D)} & {\Fun^{\mathrm{red.,s.p.}}(\AVB^\vop(n)|_{\An^\omega},\mathcal D)}
        \arrow[""{name=0, anchor=center, inner sep=0}, "{(\Th^\perp)^*}"', shift right=2, from=1-1, to=1-2]
        \arrow[""{name=1, anchor=center, inner sep=0}, "{\res|_{\An_*^\omega}}"', shift right=2, from=1-2, to=1-1]
        \arrow["\dashv"{anchor=center, rotate=90}, draw=none, from=0, to=1]
    \end{tikzcd}\]    
is a Bousfield co-localization. The category of colocal objects is the full subcategory 
\[
\Fun^{\mathrm{d.s,red.,s.p.}}(\AVB^\vop(n), \mathcal D) \subset \Fun^{\mathrm{red.,s.p.}}(\AVB^\vop(n), \mathcal D) 
\]
of functors that additionally send distinguished squares to pushouts.
\end{remark}
