\section{The Category of Anima and Affine Vector Bundles and the Thom Diagonal}\label{sec:AVBs}

\subsection{Affine Linear Homomorphisms}\label{sec:AffLin}

TODO Definition Category $\AffLin$ 


TODO Universal Property of $\AffLin$

\subsection{The Thom Diagonal in the Category $\AVB$}\label{section:ThomDiagonal}
\begin{construction}
    We consider the category 
    \[ 
        \AVB := \int_{X \in \An^\op} \Fun(X, \AffLin);
    \] 
    the unstraightening of the functor that associates to 
    each anima the category of 
    affine linear vector bundles over it. 
    We call the category $\AVB$ \emph{the category of affine linear 
    vector bundles}.
    Its objects consist of data $(X,V)$ where 
    \begin{itemize}
        \item $X$ is an anima 
        \item $V$ is a functor $X \to \AffLin$, which we call \emph{vector bundle over $X$}
    \end{itemize}


\end{construction}

\begin{remark}
    Any functor $X \to \AffLin$ lands in the core of $\AffLin$, which is 
    equivalent to the anima
    \[ 
        \left( \coprod_{n \geq 0} BO(n) \right)_+,
    \] 
    thus the terminology vector bundle is justified as long as we agree
    that the distinguished functor 
    \[
       \emptyset \colon  X \to \pt = (\emptyset)_+ \to \left(\coprod_{n \geq 0} BO(n)\right)_+,  
    \]
    which we view as the \emph{empty bundle} on $X$, is considered to be a vector bundle.
\end{remark}

\begin{construction}\label{construction: mainfunctors}
    By construction $\AVB$ carries a Cartesian fibration
     \[
        \pr \colon \AVB \to \An
     \]
    which forgets the data of the equipped vector bundle. That is 
    $\pr(X,V) = X$ extracts the underlying anima $X$ out of the data $(X,V) \in \AVB$.
    Since $\pr$ is a Cartesian fibration we obtain restriction/pullback functors 
    \[
     \res_X^-(-) \colon \An/_X \times \Fun(X,\AffLin) \to \AVB,   
    \]  
    That sends a pair of a morphism $f \colon Y \to X$ and a vector bundle $V$ on $X$ 
    to the pair $\res^Y_X(V) := (Y, f^*V)$ where $f^*V$ is the vector bundle on $Y$ that 
    arises as the composite 
    \[ 
        f^*Y \colon Y \xrightarrow{f} X \xrightarrow{V} \AffLin.
    \]
    In particular, if we fix $V \in \AffLin$ we have a functor
    \[ 
        \iota_V := \res_\pt^-(V) \colon \An \to \AVB.     
    \]
    If we fix an anima $X$ and call the unique map from $X$ to the point $r$, then 
    $ \iota_V(X) = (X,r^*V)$ and $r^*V$ is the trivial 
    rank $\mathrm{dim}(V)$ bundle on $X$.

    We define the functor $S \colon \AVB \to \An$ to be the corepresented functor 
    \[ 
    S(X,V) := S^V_X := S^V := \Map_{\AVB}(\iota_0(\pt),(X,V)) = \Map((\pt, 0),(X,V)).    
    \]
    We call $S^V_X$ the \emph{associated (unstable) spherical fibration of $V$ over $X$}.

\end{construction}

\begin{lemma}The following statements are true
    \begin{enumerate}
        \item The functor $\pr$ is corepresented by $(\pt, \emptyset)$.
        \item The functor $\iota_\emptyset$ is left and right adjoint to $\pr$.
        \item The functor $\iota_0$ is left adjoint to $S$.
    \end{enumerate}
    \label{lemma:insight}
\end{lemma}

\begin{proof}
    By construction of the category $\AVB$, we can identify the fiber of 
    \[
      \pr_{(X,V),(Y,W)} \colon \Map((X,V), (Y,W)) \to \Map(X,Y)
    \]
    at $f \colon X \to Y$ to be 
    \[
      \Map_{\Fun(X,\AffLin)}(V, f^* W).  
    \]
    If $W$ is the constant $\emptyset$ bundle then so is $f^* W$.
    The $\emptyset$ bundle is the zero object of $\Fun(X,\AffLin)$, hence if either 
    $V$ or $W$ is the $\emptyset$ bundle then $\pr_{(X,V),(Y,W)}$ has contractable fibers, i.e. is an equivalence.
    This proves 1 and 2. 
    The previous calculation also specializes to the equivalence 
    \[
    \Map_{\AVB}(\iota_0 X, (Y,W)) \simeq \colim_{f \colon X \to Y} \Map_{\Fun(X,\AffLin)}(0,f^* W).    
    \]
    The anima $\Map_{\Fun(X,\AffLin)}(0,f^* W)$ is easily seen to be equivalent to the anima of sections of 
    the associated spherical fibration of $W$ over $Y$ pulled back along $f$, i.e. 
    \[
    \Map_{\Fun(X,\AffLin)}(0,f^* W) \simeq \Gamma_X( S^W \times_{Y, f} X).
    \]
    In general, if one has a map $Z \to Y$ of anima on can compute 
    $\Map(X,Z)$ via the colimit
    \[
      \colim_{f \colon X \to Y} \Gamma_X( Z \times_{Y, f} X)  
    \]
    as seen by the following pullback squares 
    \[\begin{tikzcd}
        {\Gamma_X(Z \times_{Y,f}X)} & {\Map(X,Z\times_{Y,f}X)} & {\Map(X,Z)} \\
        {\{\id\}} & {\Map(X,X)} & {\Map(X,Y)}
        \arrow[from=1-1, to=2-1]
        \arrow[from=1-1, to=1-2]
        \arrow[from=1-2, to=2-2]
        \arrow[from=2-1, to=2-2]
        \arrow[from=1-2, to=1-3]
        \arrow[from=1-3, to=2-3]
        \arrow["{f_*}", from=2-2, to=2-3]
        \arrow["\lrcorner"{anchor=center, pos=0.125}, draw=none, from=1-2, to=2-3]
        \arrow["\lrcorner"{anchor=center, pos=0.125}, draw=none, from=1-1, to=2-2]
    \end{tikzcd}\]
    Putting these observations together gives us the equivalence 
    \[
        \Map_{\AVB}(\iota_0 X, (Y,W)) \simeq  \colim_{f \colon X \to Y} \Gamma_X( S^W \times_{Y, f} X)  \simeq \Map(X,S^W_Y).
    \]
    This proves 3.
\end{proof}

\begin{construction}
    Let 
    \[
      \theta \colon \iota_0 S \to \id_{\AVB}  
    \]
    be the counit of the adjuntion $(\iota_0 \dashv S)$.
    We call $\theta_{(X,V)}$ the \emph{Thom diagonal of $V$ on $X$}.
    Let 
    \[ 
        \sigma_\infty \colon \pr \to S    
    \]
    be the unique natural transformation, which is by Yoneda induced from the unique map $\iota_\emptyset(\pt) \to \iota_0(\pt)$, 
    which comes from the unique map $\emptyset \to 0$ in $\AffLin$.
    We call $\sigma_\infty \colon X \to S^V_X$ \emph{the section at $\infty$}.
    The composite of the unit $\id_{\AVB} \to \iota_\emptyset \pr$ and the counit $\iota_\emptyset \pr \to \id_{\AVB}$
    of the adjunctions $\iota_\emptyset \dashv \pr \dashv \iota_\emptyset$ defines an endomorphism
    of the identity on $\AVB$ which we will denote with $\infty$
    and call it the \emph{translation to $\infty$}.\footnote{This endomorphism of the identity is an artefact of the fact that $\AVB$ arises as a fibered category whose fibers are pointed categories}
    This endomorphism induces an action of the category $B(\F_2,\cdot)$ on $\AVB$, that on mapping anima 
    $\F_2 \times \Map((X,V),(Y,W)) \to \Map((X,V),(Y,W))$ sends $(0, \phi)$ to $\infty \circ \phi$.
    The map $\infty \circ \phi$ only depends on the value $\pr(\phi) =: f$, so we define 
    $\infty_{f} := \infty \circ \phi$ and drop $f$ from the notation if $f = \id$. $\infty_f$ is a functorial lift of $f$ along $\pr$ and defines a section 
    of the projection $\Map((X,V),(Y,W)) \to \Map(X,Y)$.
    \label{constr:thomdiag1}
\end{construction}

    The next lemma shows that over the section at infinity the Thom diagonal acts as the translation to infinity.
\begin{proposition}
    We have a unique homotopy
    \[
      \theta \circ \iota_0(\sigma_\infty) \simeq \infty , 
    \]
    or equivalently the following square has a unique filler 
    \[\begin{tikzcd}
        {\iota_\infty \pr} & {\id_{\AVB}} \\
        {\iota_0 \pr} & {\iota_0S}
        \arrow["{\iota_0(\sigma_\infty)}"', from=2-1, to=2-2]
        \arrow["\theta"', from=2-2, to=1-2]
        \arrow["\infty", from=2-1, to=1-1]
        \arrow["\infty", from=1-1, to=1-2]
    \end{tikzcd}\label{diag:AVBsquare}\]\label{prop: thomatinfty}
\end{proposition}

\begin{proof}
    % By the formula for the left adjoint of a morphism into a right adjoint functor, we immediately see that 
    % $\theta \circ \iota_0(\sigma_\infty)$ is the left adjoint morphism to $\sigma_\infty$. In that way both 
    % morphisms determine each other. Therefore we only need to verify that $\sigma_\infty$ is right adjoint to the 
    % morphism $\infty \colon \iota_0 \pr \to \id_{\AVB}$.
    % Let us denote the unit $\id_{\An} \to S \circ \iota_0$ with $j$. We have to compute $S(\infty) \circ j_{\pr} \colon \pr \to S$.
    % Since both functors are corepresented we only have to compute the value of $S(\infty) \circ j_{\pr}$ at the object $(\pt, \emptyset)$.
    By adjunction a natural transformation $\iota_0 \pr \to \id_{\AVB}$ corresponds one to one to a natural transformation between 
    $\pr \to S$. But because $\pr$ is corepresented by $(\pt, \emptyset)$ and $S(\pt, \emptyset) \simeq \Map((\pt, 0), (\pt, \emptyset))\simeq \pt$
    there exists essentially only one natural transformation. 
\end{proof}

\begin{remark}
Evaluated on an object $(X,V) \in \AVB$ the square~\ref{diag:AVBsquare} is of the form 
\[\begin{tikzcd}
	{(X,\emptyset)} & {(X,V)} \\
	{(X,0)} & {(S^V,0)}
	\arrow[from=2-1, to=1-1]
	\arrow[from=1-1, to=1-2]
	\arrow[from=2-1, to=2-2]
	\arrow[from=2-2, to=1-2]
\end{tikzcd}\label{diag:AVBsquareeval}\]
\end{remark}

% \section{Einleitung}
% % Text...

% \section{Grundlagen}
% % Text...

% \section{Hauptergebnisse}
% % Text...

% \section{Beweise}
% % Text...

% \section{Beispiel}
% % Text...

% \section{Schlussfolgerung}
% % Text...

\subsection{Thom Diagonal in the Category $\AVB^\vop$}

\begin{definition}
    The category $\AVB^\vop$ is the vertical dual of $\AVB$, that is 
    \[
      \AVB^\vop := \int_{X \in \An^\op} \Fun(X, \AffLin^\op)  
    \]
\end{definition}

\begin{remark}
    One might expect that a similar square to the square~\ref{diag:AVBsquare} exists 
    for the category $\AVB^\vop$ but the functoriality breaks at the functor $S$.
    Nonetheless there still exists squares of the form 
    \[\begin{tikzcd}
        {(X,\emptyset)} & {(X,0)} \\
        {(X,V)} & {(S^V,V)}
        \arrow[from=2-1, to=1-1]
        \arrow[from=1-1, to=1-2]
        \arrow[from=2-1, to=2-2]
        \arrow[from=2-2, to=1-2]
    \end{tikzcd}\]
    analoguous to the squares~\ref{diag:AVBsquareeval}. The minor difference is that these 
    squares are not functorial in $(X,V) \in \AVB^\vop$.
    To fix this lack of functoriality of the square-\ref{diag:AVBsquare} we will 
    introduce in the next section categories $\AVB(n)$ for each natural number $n$, 
    which filter the category $\AVB^\vop$.
    Over these categories $\AVB(n)$ we can prove a version of Proposition-\ref{prop: thomatinfty}.
\end{remark}

\begin{definition} 
    We call a square $\square \to \AVB^\vop$ \emph{simple distinguished} if it is equivalent to one of the 
    form 
    \[\begin{tikzcd}
        {(X,\emptyset)} & {(X,0)} \\
        {(X,V)} & {(S^V,V)}
        \arrow[from=2-1, to=1-1]
        \arrow[from=1-1, to=1-2]
        \arrow[from=2-1, to=2-2]
        \arrow[from=2-2, to=1-2]
    \end{tikzcd}.\]
    Let $W$ be a vector bundle over $X$. We can add $W$ to the square above and obtain a square 
    \[\begin{tikzcd}
        {(X,\emptyset)} & {(X,W)} \\
        {(X,V \oplus W)} & {(S^V,V \oplus W)}
        \arrow[from=2-1, to=1-1]
        \arrow[from=1-1, to=1-2]
        \arrow[from=2-1, to=2-2]
        \arrow[from=2-2, to=1-2]
    \end{tikzcd}.\]
    We call squares equivalent to one of that form \emph{distinguished}.     
\end{definition}
\begin{remark}
    Let $E \colon \left(\AVB^\vop\right)^\op \to \cC$ be a contravariant functor into a pointed category 
    with fibers. Suppose $E$ sends objects of the form $(X,\emptyset)$ to $0$. 
    Then for each collection of data $(X,V),(X,W) \in \AVB^\vop$ we define the object
    \[
        E^{W - V}(X) := \fib( E^{W}(S^V) \to E^{W}(X)).
    \]
    A distinguished square gives rise to a morphism 
    \[
        E^{W}(X) \to E^{V \oplus W - V}(X)  
    \]
    which is an equivalence if $E$ sends distinguished 
    squares to pullbacks.
    If additionally $E$ suffices an excision / Mayer-Vietoris property in the 
    $X$ variable, then one can further compute 
    \[
    E^{n - V}(X) \simeq \tilde{E}^n(\Th(V))    
    \]   
\end{remark}

\subsection{Thom Diagonal in the Category $\AVB^\vop(n)$}\label{sec:AVB(n)}

To address the issue that we are not able to establish Proposition~\ref{prop: thomatinfty}
in the context of the category $\AVB^\vop$ we will introduce a filtration of categories 
\[
  \AVB^\vop(n) \to \AVB^\vop(n+1) \to \dots \to \AVB^\vop.
\]
We will be able to prove a version of Proposition~\ref{prop: thomatinfty} for the filtration pieces $\AVB^\vop(n)$.
In other words, our goal for this section is to prove the following proposition.

\begin{proposition}\label{prop: AVB(n)}
 There exist Cartesian fibrations $\AVB^\vop(n) \xto{\pr} \An$ together with functors of 
 Cartesian fibrations 
 \[
   \AVB^\vop \xleftarrow{\overleftarrow{\pr}} \AVB^\vop(n) \xrightarrow{\overrightarrow{\pr}} \AVB 
 \]
 such that
 \begin{enumerate}
    \item the endofunctor $- \oplus \R \colon (X,V) \mapsto (X, V \oplus \R_X) \colon \AVB \to \AVB$ extends to a
    functor $- \oplus \overrightarrow{\R} \colon \AVB^\vop(n) \to \AVB^\vop(n + 1)$ so that the following diagram is 
    rendered commutative 
    \[\begin{tikzcd}
        {\AVB^\vop} & {\AVB^\vop(n)} & \AVB \\
        {\AVB^\vop} & {\AVB^\vop(n + 1)} & \AVB
        \arrow["{ -\oplus\overrightarrow{\R}}"', dashed, from=1-2, to=2-2]
        \arrow["{\overrightarrow{\pr}}", from=1-2, to=1-3]
        \arrow["{\overrightarrow{\pr}}"', from=2-2, to=2-3]
        \arrow["{ - \oplus \R}", from=1-3, to=2-3]
        \arrow["{\overleftarrow{\pr}}"', from=1-2, to=1-1]
        \arrow["{\overleftarrow{\pr}}", from=2-2, to=2-1]
        \arrow[Rightarrow, no head, from=1-1, to=2-1]
        \label{}
    \end{tikzcd}\]
    and the induced functor 
    \[
      \colim_{n \to \infty} \AVB^\vop(n) \xto{\overleftarrow{\pr}} \AVB^\vop   
    \]
    is an equivalence over the category of compact anima, that is 
    \[
        \colim_{n \to \infty} \AVB^\vop(n) \times_{\An} \An^\omega \xto{\overleftarrow{\pr}} \AVB^\vop \times_{\An} \An^\omega
    \]
    is an equivalence, 
    \item there exists a diagram $\square \to \mathrm{End}(\AVB^\vop(n))$ of endofunctors  of the form 
    \begin{equation}
    % \begin{figure}
        % \centering
        \begin{tikzcd}
        {\iota_{\overleftarrow{\emptyset},\overrightarrow{\emptyset}}\pr} & \id \\
        {\iota_{\overleftarrow{\R^n},\overrightarrow{0}}\pr} & {\iota_{\overleftarrow{\R^n},\overrightarrow{0}}S^\perp}
        \arrow["\infty", Rightarrow, from=2-1, to=1-1]
        \arrow["\infty", Rightarrow, from=1-1, to=1-2]
        \arrow["{\iota_{\overleftarrow{\R^n},\overrightarrow{0}}\sigma_\infty}"', Rightarrow, from=2-1, to=2-2]
        \arrow["\theta"', Rightarrow, from=2-2, to=1-2]
    \end{tikzcd}
    % \caption{Functorial resolution}
    \label{fig:endosquareavb(n)}
    % \end{figure}
\end{equation}
    that gets sends to distinguished squares under $\overleftarrow{\pr}$.
 \end{enumerate}
\end{proposition}

We begin by constructing the category $\AVB^\vop(n)$.

\subsection{Construction of $\AVB^\vop(n)$}

\begin{construction}
    First we will introduce the intermediate category $\AWB(n)$.
    Let $\AWB(n)$ be the pullback of categories 
    \[\begin{tikzcd}
        {\AWB(n)} & {\AVB/(\pt,\R^n)} \\
        {\AVB \times_{\An} \AVB} & \AVB
        \arrow["\oplus"', from=2-1, to=2-2]
        \arrow["{\mathrm{source}}", from=1-2, to=2-2]
        \arrow["q"', from=1-1, to=2-1]
        \arrow[from=1-1, to=1-2]
        \arrow["\lrcorner"{anchor=center, pos=0.125}, draw=none, from=1-1, to=2-2]
    \end{tikzcd}.\]
    One can think of an object of $\AWB(n)$ to consists of the data 
    \begin{itemize}
        \item an anima $X$,
        \item two vector bundles $V,W$ over $X$,
        \item an affine linear morphism $\alpha \colon V\oplus W \to \R^n_X$.
    \end{itemize}
\end{construction}
The morhisms in $\AWB(n)$ consist of covariant morphisms in both vector bundle
coordinates. This is one of the major differences to the category $\AVB^\vop(n)$.
There we want only one vector bundle coordinate to be covariant, while the other coordinate 
has contravariant morphisms. To remidy this, we will model $\AVB^\vop(n)$ as a subcategory of 
a span category. The following lemma will be useful. 
\begin{lemma}
Let $q \colon \cC \to \cD$ be a pullback preserving functor and let 
$(\cD, b.w., f.w.)$ endow $\cD$ with the structure of an adequate triple. 
Set $q^*(b.w.) := \{ f \colon \Delta^1 \to \cC | q(f) \in b.w.\}$
and $q^*(f.w.)$ likewise.
Then $(\cC, q^*(b.w.), q^*(f.w.))$ is an adequate triple and $q$ refines 
to a functor of adequate triples. 
\end{lemma}

\begin{construction}
    We first endow the category $\AVB \times_{\An} \AVB$ with the structure of an 
    adequate triple via 
    \small
    \[
      (\AVB^{\times_{\An} 2}, b.w., f.w.) := (
        \AVB^{\times_{\An} 2}, (\AVB \times_{\An} \An^\simeq ) \times_{\An} \AVB^\simeq, 
        \mathrm{Cart}(\AVB) \times_{\An} \AVB) . 
    \]
    \normalsize
    We can think of the category $\mathrm{Span}(\AVB^{\times_{\An} 2}, b.w., f.w.)$
    as follows 
    \begin{itemize}
        \item An object consists of the data of an anima $X$ and two vector bundles $V,W$ 
        over $X$, 
        \item A morphism $(X,V,W) \to (Y,V',W')$ consists of the data
        \begin{itemize}
            \item A morphism $f \colon X \to Y$, 
            \item A morphism $\phi \colon f^*V' \to V$
            \item A morhism $\psi \colon W \to f^*W'$.
        \end{itemize}
    \end{itemize}
    In fact the category $\mathrm{Span}(\AVB^{\times_{\An} 2}, b.w., f.w.)$ is equivalent to 
    the category $\AVB^{\vop}\times_{\An}\AVB$. 
    We are more interested in the category 
    \[ 
        \cD(n) := \mathrm{Span}(\AWB(n), q^*b.w., q^*f.w.).
    \]
    The objects $(X,V, W, \alpha \colon (X, V \oplus W) \to (\pt, \R^n))$ of $\cD(n)$ have an 
    additional datum of a map from the sum of the vector bundles to the object $(\pt, \R^n)$.
    As mentioned earlier this is equivalent to the datum of a map 
    \[
    V \oplus W \to \R^n_X .  
    \]
    The morphisms $(X, V, W, \alpha) \to (Y, V', W', \beta)$ of $\cD(n)$ contain an additional 
    compatibility datum 
    \[\begin{tikzcd}
        & {f^*V' \oplus W} \\
        {V\oplus W} && {f^*V' \oplus f^*W'} \\
        & {\R^n_X}
        \arrow["{\phi \oplus W}"', from=1-2, to=2-1]
        \arrow["{f^*V' \oplus\psi}", from=1-2, to=2-3]
        \arrow["\alpha"', from=2-1, to=3-2]
        \arrow["{f^*\beta}", from=2-3, to=3-2]
    \end{tikzcd}\]
    Finally let $\AVB^\vop(n) \subset \cD(n)$ be the full subcategory on those objects 
    \[
        (X,V,W,\alpha \colon (X, V \oplus W) \to (\pt, \R^n))  
    \] where either 
    \begin{itemize}
        \item $\alpha$ is a Cartesian morphism or 
        \item $V = W = \emptyset_X$.
    \end{itemize}
    In the first case, $\alpha$ identifies $W$ as an orthogonal complement bundle of $V$ inside of 
    $\R^n_X$.
    By construcion we have a functor 
    \[
        \AVB^\vop(n) \subset \cD(n) \to \AVB^\vop \times_{\An} \AVB  \to \AVB^\vop \times \AVB
    \]
    Let its projections be called 
    \begin{eqnarray*}
        \overleftarrow{\pr} & \colon & \AVB^\vop(n)  \to \AVB^\vop \\
        \overrightarrow{\pr} & \colon & \AVB^\vop(n)  \to  \AVB
    \end{eqnarray*}

    Moreover the functor 
    \[
        \id_{\AVB} \times_{\id_{\An}} (- \oplus \R) \colon \AVB \times_{\An} \AVB \to \AVB \times_{\An} \AVB        
    \]
    lifts to a functor $- \oplus \overrightarrow{\R} \colon \AWB(n) \to \AWB(n + 1)$ under $q$, 
    such that $- \oplus \overrightarrow{\R}$ is compatible with the adequate triple structure. 
    This induces a functor 
    $- \oplus \overrightarrow{\R} \colon \cD(n) \to \cD(n + 1)$ which restricts to a functor 
    \[
        - \oplus \overrightarrow{\R} \colon \AVB^\vop(n) \to \AVB^\vop(n + 1).
    \] 

\end{construction}

    It is evident by construction that the functor $-\oplus \overrightarrow{\R}$
    renders the following diagram commutative 
    \[\begin{tikzcd}
        {\AVB^\vop} & {\AVB^\vop(n)} & \AVB \\
        {\AVB^\vop} & {\AVB^\vop(n + 1)} & \AVB
        \arrow["{ -\oplus\overrightarrow{\R}}"', dashed, from=1-2, to=2-2]
        \arrow["{\overrightarrow{\pr}}", from=1-2, to=1-3]
        \arrow["{\overrightarrow{\pr}}"', from=2-2, to=2-3]
        \arrow["{ - \oplus \R}", from=1-3, to=2-3]
        \arrow["{\overleftarrow{\pr}}"', from=1-2, to=1-1]
        \arrow["{\overleftarrow{\pr}}", from=2-2, to=2-1]
        \arrow[Rightarrow, no head, from=1-1, to=2-1].
    \end{tikzcd}\]

\subsubsection{Proof of Proposition~\ref{prop: AVB(n)} Part 1}

    We now want to show that the induced functor 
    \[
    \overleftarrow{\pr} \colon \AVB^\vop(\infty) := \colim_n \AVB^\vop(n) \to \AVB^\vop   
    \]
    induces an equivalence 
    \[
    \overleftarrow{\pr}_{\An^\omega} \colon \AVB^\vop(\infty) \times_{\An} \An^\omega \to \AVB^\vop \times_{\An} \An^\omega.
    \]
\begin{remark}
    It is not true that $\overleftarrow{\pr}$ is an equivalence. First our argument 
    wont offer more than an equivalences over finite anima as it uses that 
    filtered colimits in $\catinfty$ commute with finite limits and not all limits. Second it is 
    not even true that $\overleftarrow{\pr}$ is essentially surjective as this would imply that 
    all vector bundles admit a finite dimensional orthogonal complement, which is absurd when one considers 
    the universal rank one bundle $V_{\mathrm{uni}} \to BO(1)$.
    One can verify by a calculation in characteristic classes that a potential complement of 
    $V_\mathrm{uni}$ cannot have a bounded rank.
\end{remark}

    We observe that our construction of $\AVB^\vop(n)$ is compatible with its parametrization 
    over $\An$, that means that we can construct $\AVB^\vop(n)$ as the unstraightening of a functor 
    that sends $X$ to the functor category $\Fun(X, \AffLin^\op(n))$. Where $\AffLin^\op(n)$ is some 
    category constructed out of $\AffLin$ in a similar fashion to how we have constructed $\AVB^\vop(n)$
    out of $\AVB$. 

\begin{construction}
    Let $\mathcal W(n)$ be the pullback of categories 
    \[\begin{tikzcd}
        {\mathcal W(n)} & {\AffLin/\R^n} \\
        {\AffLin^{\times 2}} & \AffLin
        \arrow["{\mathrm{source}}", from=1-2, to=2-2]
        \arrow["\oplus"', from=2-1, to=2-2]
        \arrow[from=1-1, to=2-1]
        \arrow[from=1-1, to=1-2]
        \arrow["\lrcorner"{anchor=center, pos=0.125}, draw=none, from=1-1, to=2-2]
    \end{tikzcd}\]
    Let $\tilde{r.e.}, \tilde{l.e.}$ be defined as 
    \begin{eqnarray*} 
        \tilde{r.e.} & := & \left(\AffLin \times \AffLin^\simeq \right) \times_{\AffLin} \AffLin/\R^n \\
        \tilde{l.e.} & := & \left(\AffLin^\simeq \times \AffLin \right) \times_{\AffLin} \AffLin/\R^n.
    \end{eqnarray*}
    The categories of morphisms $\tilde{r.e.}, \tilde{l.e.}$ endow $\mathcal W(n)$ with the structure of 
    an adequate triple. 
    Let $\AffLin^\op(n)$ be the full subcategory of
    \[
        \Span(\mathcal W(n), \tilde{r.e.}, \tilde{l.e.}) 
    \]
    on those objects $(V,W, \alpha \colon  V \oplus W \to \R^n)$ where either 
    \begin{itemize}
        \item $V = W = \emptyset$ or 
        \item $\alpha$ is an equivalence.
    \end{itemize}

    A typical morphism of $\AffLin^\op(n)$ is of the form 
          \[
            (V,W,\alpha\colon  V\oplus W \xto{\simeq} \R^n) \xot{f}
            (S,U,\beta \colon S \oplus U \to \R^n) \xto{g}
            (V',W'\alpha' \colon V'\oplus W' \xto{\simeq} \R^n)  
          \]
          In particular it gives us a commutative diagram 
          \[\begin{tikzcd}
              && {S \oplus U} \\
              {V\oplus W} && {\R^n} && {V'\oplus W'}
              \arrow["{(\overleftarrow{\pr}g,\overrightarrow{\pr}g)}", from=1-3, to=2-5]
              \arrow["{(\overleftarrow{\pr}f\overrightarrow{\pr}f)}"', from=1-3, to=2-1]
              \arrow["{\alpha }"', from=2-1, to=2-3]
              \arrow[from=1-3, to=2-3]
              \arrow["{\alpha'}", from=2-5, to=2-3]
          \end{tikzcd}\]
          where $\overrightarrow{\pr}f$ and $\overleftarrow{\pr}g$ are equivalences or $V=W=\emptyset$.
          Hence, we can think of the data of a morphism in $\AffLin^\op(n)$ to 
          consist of  
          \begin{itemize}
              \item A map $\phi \colon V' \to V$,
              \item A map $\psi \colon W \to W'$,
              \item A commutative square 
              \[\begin{tikzcd}
                  {V'\oplus W} & {V'\oplus W'} \\
                  {V\oplus W} & {\R^n}
                  \arrow["\alpha"', from=2-1, to=2-2]
                  \arrow["{\alpha'}", from=1-2, to=2-2]
                  \arrow["{\phi\oplus W}"', from=1-1, to=2-1]
                  \arrow["{V'\oplus \psi}", from=1-1, to=1-2]
              \end{tikzcd}\]
          \end{itemize} 
\end{construction}

\begin{lemma}
    Let $X \in \An$ and let $(\cC, b.w._{\cC}, f.w._{\cC})$ be an adequate triple. 
    The functor category $\cC^X$ is an adequate triple by setting
    \[
    b.w._{\cC^X} := (b.w._{\cC})^X    
    \]    
    and likewise with $f.w._{\cC^X}$.
    Then the natural map into the limit 
    \[
        \Span(\cC^X, b.w._{\cC^X}, f.w._{\cC^X}) \to \Span(\cC,b.w._{\cC}, f.w._{\cC})^X
    \]
    is an equivalence.
\end{lemma}

\begin{proof}
    To show that this functor is an equivalence we show that the corresponding map 
    \[
        \left( \Span(\cC^X, b.w._{\cC^X}, f.w._{\cC^X})^{\Delta^\bullet}\right)^\simeq \to \left(\Span(\cC,b.w._{\cC}, f.w._{\cC})^{X \times \Delta^\bullet}\right)^\simeq
    \]
    of Segal anima is an equivalence.
    For $\bullet = 0$ this follows from the fact that the core of a category $\cD$ is equivalent to the core of its span category 
    \[
        \cD^\simeq \subset \left(\Span(\cD, b.w._{\cD}, f.w._{\cD})\right)^\simeq \subset \left(\Span(\cD)\right)^\simeq \simeq \cD^\simeq 
    \]
    and the fact that the core is right adjoint to the inclusion of anima into categories
    \[
      \left(\cD^X\right)^\simeq \simeq (\cD^\simeq)^X.
    \]
    For $\bullet \geq 1$ we can reduce to $\bullet = 1$ by the Segal property of the span category construction as a complete Segal anima.
    In that case we find
    \begin{eqnarray*}
        \left(\Span(\cC^X, b.w._{\cC^X}, f.w._{\cC^X} )^{\Delta^1}\right)^\simeq & \simeq & \left(b.w._{\cC^X} \times_{\cC^X}f.w._{\cC^X}\right)^\simeq \\
        % & \simeq & (\left(b.w. \times_{\cC} f.w. \right)^X)^\simeq \\
        & \simeq & \left((b.w. \times_{\cC} f.w.)^X\right)^\simeq \\
        & \simeq & \left(\Span(\cC,f.w._{\cC}, b.w._{\cC})^{X \times \Delta^1}\right)^\simeq
    \end{eqnarray*}

\end{proof}

\begin{proposition}\label{prop: asunst}
    % The exists a diagram where the vertical arrows are equivalences 
    % \tiny
    % \[\begin{tikzcd}[column sep=tiny]
    %     {\AVB^\vop(n)} & {\Span(\left( \int_{X \in \An^\op} \AffLin^X\right)^{\times 2}\times_{\left( \int_{X \in \An^\op} \AffLin^X\right)}\left( \int_{X \in \An^\op} \AffLin^X/\mathrm{const}_{\R^n}\right), r.e., l.e.)} \\
    %     {\int_{X \in \An^\op} \AffLin^\op(n)^X} & {\int_{X \in \An^\op} \Span(\mathcal W(n), \tilde{r.e.}, \tilde{l.e.})^X}
    %     \arrow[hook, from=1-1, to=1-2]
    %     \arrow[Rightarrow, no head, from=1-2, to=2-2]
    %     \arrow[Rightarrow, no head, from=1-1, to=2-1]
    %     \arrow[hook, from=2-1, to=2-2]
    % \end{tikzcd}\]
    % \normalsize

There exists an equivalence of categories 
\[
    \cD(n) = \Span(\AWB(n), r.e., l.e.) \simeq \int_{X \in \An^\op}\Span(\mathcal W(n), \tilde{r.e.}, \tilde{l.e.})^X.
\]
Furthermore over the category of compact anima
the full subcategories 
\[
    \AVB^\vop(n) \times_{\An}\An^\omega \subset \cD(n) \times_{\An}\An^\omega
\]
and 
\[
    \int_{X \in (\An^\omega)^\op} \AffLin^\op(n)^X \subset \int_{X \in (\An^\omega)^\op}\Span(\mathcal W(n), \tilde{r.e.}, \tilde{l.e.})^X
\]
coincide under this identification.


\end{proposition}
\begin{proof}
    By using the equivalence 
    \[
    \AVB/(\pt, \R^n) \simeq \int_{X \in \An^\op} (\AffLin/\R^n)^X   
    \]
    we find that $\AWB(n)$ is a pullback of Cartesian fibrations over $\An$ along functors that 
    are functors of Cartesian fibrations.
    As such it is a Cartesian fibration over $\An$ too, namely
    \[
    \AWB(n) \simeq \int_{X \in \An^\op} (\AffLin^X)^{\times 2} \times_{\AffLin^X} \times (\AffLin/\R^n)^X \simeq \int_{X \in \An^\op} \mathcal W(n)^X
    \]
    By CITE FABIAN!!!, we have 
    \[
    \cD(n) = \Span(\AWB(n), r.e., l.e.) \simeq \int_{X \in \An^\op} \Span(\mathcal W(n)^X, \tilde{r.e.}^X, \tilde{l.e.}^X)    
    \]
    The last lemma shows 
    \[
        \int_{X \in \An^\op} \Span(\mathcal W(n)^X, \tilde{r.e.}^X, \tilde{l.e.}^X) \simeq \int_{X \in \An^\op} \Span(\mathcal W(n), \tilde{r.e.}, \tilde{l.e.})^X  
    \]
    \dots


\end{proof}    
    

    We conclude that in particular $\overleftarrow{\pr}_{\An^\omega}$ is a functor of Cartesian fibrations over $\An^\omega$
    and the straightend functors $(\An^\omega)^\op \to \catinfty$ of its source and target 
    are both finite limit preserving 
    and thus it is enough to show that $\overleftarrow{\pr}$ is an equivalence over the point, i.e.
    \[ 
        \AffLin^\op(\infty) := \colim_n \AffLin^\op(n) \to \AffLin^\op
    \]
    is an equivalence. 

  \begin{lemma}\label{colimitlemma}
    The functor $-\oplus \overrightarrow{\R} \colon {\cO}^{\times 2} \to {\cO}^{\times 2}
      \colon (V,W) \mapsto (V, W\oplus \R)$ induces a functor
    $-\oplus \overrightarrow{\R} \colon \cO^\op(n)\to \cO^\op(n+1)$,
    compatible with the functors
    $\cO^\op(n) \to \Span({\mathcal W(n)}, r.e., l.e. ) \xto{(pr_1)_*} \Span(\cO,
      {\cO}, {\cO}^\simeq) \simeq ({\cO})^\op$
    so that we have an equivalence
    \[
      \AffLin^\op(\infty) := \colim (\cO^\op(0) \xto{-\oplus \overrightarrow{\R}} \cO^\op(1)
      \xto{-\oplus \overrightarrow{\R}} \dots) \xrightarrow{\simeq} {\cO}^\op 
      .\]
  \end{lemma}
  \begin{proof}
    The existence of the induced functor
    $\cO(n)^\op \to \cO(n+1)^\op$ and the compatibility is clear.
    We show that the functor
    \[
     \AffLin^\op(\infty) \to ({\cO})^\op
    \]
    is fully faithful and essentially surjective.
    The essential surjectivity is clear as every object of $\AffLin$ is 
    equivalent to $\R^k$ or $\emptyset$ for some $k$.
    And these are evidently in the image. 
    We now show that the functor is fully faithful.
    Fix objects $(V,W,\alpha \colon V \oplus W \xto{\simeq} \R^n),
      (V',W',\alpha' \colon V' \oplus W' \xto{\simeq} \R^n) \in \cO^\op(n)$ and let
    \[
      M := \Map((V,W,\alpha \colon V \oplus W \xto{\simeq} \R^{n}),
      (V',W',\alpha' \colon V \oplus W' \xto{\simeq} \R^{n}))
      .\]
    In the case $\emptyset \in \{V,W,V',W'\}$ we easily compute
    $ M \simeq \pt \simeq \Map_{(\cO)^\op}(V',\emptyset) \simeq
      \Map_{(\cO)^\op}(\emptyset, V)$ and the claim follows from the fact
    $(\emptyset,\emptyset) \oplus \overrightarrow{\R} \simeq
      (\emptyset,\emptyset) \in \cO(n+1)$ as $\emptyset \in \cO$
    is an absorbing element with respect to $\oplus$.
  
    So suppose $V,W,V',W' \neq \emptyset$. By unraveling the construction of
    $\cO^\op(n)$ we have that $M$ is equivalent to the core of the
    full subcategory of
    \[
      \Fun(\Lambda^2_0, {\mathcal W(n)}) \times_{(\ev_1,\ev_2),
        {\mathcal W(n)}\times {\mathcal W(n)}} \{(V,W,\alpha),(V',W',\alpha')\}
    \]
    on those wedges
    \[
      (V,W,\alpha) \xleftarrow{f} (A,B,\beta) \xto{g} (V',W',\beta)
    \]
    for which $f$ is a right equivalence and $g$ is a left equivalence.
    By writing $ \Fun(\Lambda^2_0, {\mathcal W(n)}) $ as the
    pullback ${\mathcal W(n)}^{\Delta^1} \times_{s,{\mathcal W(n)},s} {\mathcal W(n)}^{\Delta^1}$
    and commuting limits with limits we find that $M$ is equivalent to the core of
    \[
      r.e./(V,W,\alpha) \times_{{\mathcal W(n)}} l.e./(V',W',\alpha').
    \]
    The category of right equivalences, for example,
    is itself a pullback of categories.
    % , see \eqref{eq:defnre}.
    Therefore, we can compute $r.e./(V,W,\alpha)$, after
    commuting limits again, as the pullback of
    \[
      \cO/V \times (\cO)^\simeq/W \xto{(\oplus, \R^n)} \cO/V\oplus W \times
      \cO/\R^n \xot{(s,t)} (\cO)^{\Delta^1}/\alpha
    \]
    The category $(\cO)^\simeq/W$ is contractible, so we can omit it in the above wedge.
    Moreover, the above wedge is of the form
    $A \times_{\pt} \pt \to B \times_{\pt} C \leftarrow D \times_D D$,
    so we can compute its pullback $P$ as the
    pullback of  $A \times_B D \to D \leftarrow \pt \times_C D$.
    Let us first compute
    \begin{eqnarray*}
      \pt \times_C D &=& \{\id_{\R^n}\} \times_{\cO/\R^n,t} (\cO)^{\Delta^1}/\alpha \\
      & \simeq & \{\id_{\R^n}\} \times_{(\cO)^{\Delta^1}, \ev_{\{1\} \subset \{1,2\} }}
      \Fun(\Pow(\langle 2 \rangle), \cO)
      \times_{(\cO)^{\Delta^1}, \ev_{\{2\} \subset \{1,2\} }} \{\alpha \} \\
      & \simeq & \cO / V \oplus W  = B .
    \end{eqnarray*}
    One can check that the composition $B = \pt \times_C D \to D \to B$ is the identity,
    and that makes the map from $P$ to $A = \cO/V$ an equivalence.
    Similarly one can compute that $l.e./(V',W',\alpha')$ is equivalent to
    $\cO/W'$.
    If we put these results together we can identify $M$ with the core of the pullback
    \[
      \cO/V \to {\mathcal W(n)} \leftarrow \cO/W'
    \]
    where the left map sends a map $f\colon U \to V$ to the triple
    $(U,W, U\oplus W \xto{f \oplus W} V \oplus W \xto{\alpha} \R^n)$ and
    the right map sends $g \colon U \to W'$ to the triple
    $(V',U,V' \oplus U \xto{V' \oplus g} V' \oplus W' \xto{\alpha'} \R^n)$.
  
    There is an evident map from
    $\Map(V',V) \times_{\Map(V'\oplus W,\R^n)} \Map(W,W')$
    into $\cO/V \times_{{\mathcal W(n)}} \cO/W'$. In fact it is obtained
    as the pullback:
    \[\begin{tikzcd}[column sep=tiny]
        {\Map(V',V)\times_{\Map(V'\oplus W,\R^n)}\Map(W,W')} && {\cO/V \times_{{\mathcal W(n)}}\cO/W} \\
        \\
        \\
        {\pt \times_{\pt} \pt} && {\cO \times_{\id\times\{W\},\cO \times \cO,\{V'\}\times\id}\cO}
        \arrow[from=1-1, to=4-1]
        \arrow["{(V',W)}", from=4-1, to=4-3]
        \arrow["{s\times_ss}", from=1-3, to=4-3]
        \arrow[from=1-1, to=1-3]
        \arrow["\lrcorner"{anchor=center, pos=0.125}, draw=none, from=1-1, to=4-3]
      \end{tikzcd}\]
    After taking cores the lower map becomes an equivalence, which shows that
    $M$ is equivalent to $\Map(V',V) \times_{\Map(V'\oplus W,\R^n)} \Map(W,W')$.
    Under this identification the forgetful functor $\cO(n) \to (\cO)^\op$
    induces the projection $\Map(V',V) \times_{\Map(V'\oplus W,\R^n)} \Map(W,W') \to \Map(V',V)$
    and the functor $-\oplus \overrightarrow{\R} \colon \cO(n) \to \cO(n+1)$
    can be identified with the map
    \[
      \Map(V',V) \times_{\Map(V'\oplus W,\R^n)} \Map(W,W') \to
      \Map(V',V) \times_{\Map(V'\oplus W \oplus \R,\R^{n+1})} \Map(W\oplus \R,W'\oplus \R)
    \]
    Now, it becomes clear, that the induced functor
    $\colim \cO(n) \to (\cO)^\op$ is fully faithful,
    whence we have proven that the map
    \begin{eqnarray*}
      \Map(W\oplus \R^k, W' \oplus \R^k) &\to& \Map(V' \oplus W \oplus \R^k, \R^{n+k}) \\
      \phi &\mapsto & \alpha' \circ (\id_{V'} \oplus \phi)
    \end{eqnarray*}
    becomes an equivalence in the colimit for $k \to \infty$.
    Since $\alpha'$ is an equivalence, we may as well show the claim for the system
    \begin{eqnarray*}
      \Map(W\oplus \R^k, W' \oplus \R^k) &\to&
      \Map(V' \oplus W \oplus \R^k, V' \oplus W' \oplus \R^k) \\
      \phi &\mapsto & \id_{V'} \oplus \phi
    \end{eqnarray*}
  
    Both anima of affine linear embeddings are Thom anima
    $M_i$ for vector bundles $\xi_i \to A_i$,  $i = 1,2$, 
    \begin{itemize}
      \item $A_1 = \mathrm{IsoEmb}({W \oplus \R^k},{W' \oplus \R^k})$
      \item $A_2 = \mathrm{IsoEmb}({V' \oplus W \oplus \R^k},{V' \oplus W' \oplus \R^k})$
    \end{itemize}
    of rank
    $\dim W' - \dim W$.
    Moreover the map between $M_1 \to M_2$ comes from a map of vector bundles
    $\xi_1 \to \xi_2$, which induces isomorphisms on fibers.
    Hence we can reduce to check that $A_1 \to A_2$ becomes an equivalence for 
    $k \to \infty$.
    After picking bases $W \cong \R^l, W' \cong \R^{l'}, V' \cong \R^{m'}$,
    we obtain equivalences $A_1 \simeq O(l' + k)/O(l' - l)$ and
    $A_2 \simeq O(m' + l' + k)/O(l' - l)$.
    Under these choices, 
    we can identify the map $A_1 \to A_2$ with the standard inclusion
    $O(l' + k)/O(l' - l) \subset O(m' + l' + k)/O(l' - l)$,
    which yields the desired equivalence on colimits for $k \to \infty$.
  \end{proof}

\subsubsection{Proof of Proposition~\ref{prop: AVB(n)} Part 2}

Similar to Construction~\ref{construction: mainfunctors} we will define analogues 
of the functors $\iota, S$ in the context of $\AVB^\vop(n)$.

\begin{construction}
    By Proposition~\ref{prop: asunst} $\AVB^\vop(n)$ is Cartesian over $\An$ 
    with fiber over the point given by $\AffLin^\op(n)$, so we have a pullback/restriction functor
    \[
        \res \colon \An \times \AffLin^\op(n) \to \AVB^\vop(n) \colon (X, Z) \mapsto \res^X_\pt(Z) = (X, r^*Z)
    \]
    where $r \colon X \to \pt$ is the unique map.
    For any $Z \in \AffLin^\op(n)$ let 
    \[
        \iota_Z(-) = \res^{-}_{\pt}(Z) \colon \An \to \AVB^\vop(n)    
    \]
\end{construction}

Consider the following objects of $\AffLin^\op(n)$.
\begin{itemize}
    \item $Z_1 = (\emptyset, \emptyset, \emptyset \oplus \emptyset = \emptyset \xto{\infty} \R^n)$
    \item $Z_2 = (\R^n, 0, \R^n \oplus 0 = \R^n)$.
\end{itemize}
One can check that $Z_1$ is the zero object of $\AffLin^\op(n)$ and that $Z_2$ corepresents the functor 
that sends $(V,W,\alpha)$ to $S^W$.
We set 
\begin{itemize}
    \item $\iota_{\overleftarrow{\emptyset},\overrightarrow{\emptyset}} := \iota_{Z_1} \colon \An \to \AVB^\vop(n)$
    \item $\iota_{\overleftarrow{\R^n},\overrightarrow{0}} := \iota_{Z_2} \colon \An \to \AVB^\vop(n)$.
\end{itemize}
The unique morphism $Z_2 \to Z_1$ induces a natural transformation \[
    \infty \colon \iota_{\overleftarrow{\R^n},\overrightarrow{0}} \Rightarrow 
\iota_{\overleftarrow{\emptyset},\overrightarrow{\emptyset}}.
\]
Now consider the projection 
\[\begin{tikzcd}
	{\AVB^\vop(n)} && \AVB \\
	& \An
	\arrow["{\overrightarrow{\pr}}", from=1-1, to=1-3]
	\arrow["\pr"', from=1-1, to=2-2]
	\arrow["\pr", from=1-3, to=2-2]
\end{tikzcd}\]
and set 
\[
S^\perp = S \circ \overrightarrow{\pr} \colon \AVB^\vop(n) \to \AVB \to \An.  
\]
The natural transformation $\sigma_\infty \colon \pr \Rightarrow S$ induces a 
natural transformation $\pr \Rightarrow S^\perp$ which we also call $\sigma_\infty$.

\begin{lemma}The following statements are true
    \begin{enumerate}
        \item The functor $\pr \colon \AVB^\vop(n) \to \An$ is corepresented by $\iotaemptyempty(\pt)$.
        \item The functor $\iotaemptyempty$ is left and right adjoint to $\pr$.
        \item The functor $S^\perp$ is corepresented by $\iotarnzero(\pt)$.
        \item The functor $\iotarnzero$ is left adjoint to $S^\perp$.
    \end{enumerate}\label{lemma:favouritelemma}
\end{lemma}

\begin{proof}
    We strictly follow the proof of Lemma~\ref{lemma:insight}.
    The fiber of the map 
    \[
    \Map_{\AVB^\vop(n)}((X,\overleftarrow{V},\overrightarrow{W}, \alpha), (Y,\overleftarrow{V'},\overrightarrow{W'},\beta)) \to \Map(X,Y)   
    \]
    at $f \colon X \to Y$ can be identified with 
    \[
    M := \Map_{\AffLin^\op(n)^X}((\overleftarrow{V},\overrightarrow{W}, \alpha), (\overleftarrow{f^*V'},\overrightarrow{f^*W'},f^*\beta)).
    \]
    The proof of Lemma~\ref{colimitlemma} shows that $M$ can be computed as the pullback 
    \[
      \Map_{\AffLin^X}(f^*V', V) \times_{\Map_{\AffLin^X}(f^*V' \oplus W, \R^n_X)} \Map_{\AffLin^X}(W,f^*W').
    \] 
    For $V = W = \emptyset$ this is contractible, which shows 1. and 2.
    For $V = \R^n_X, W = 0, \alpha = \mathrm{triv}$, we immediately see that the pullback is equivalent to 
    \[
      \Map_{\AffLin^X}(0,f^*W').  
    \]
    From here we can follow the rest of the proof of Lemma~\ref{lemma:insight}.
\end{proof}

\begin{construction}
    Let
    \[
        \theta \colon \iotarnzero S^\perp \Rightarrow \id_{\AVB^\vop(n)}    
    \]
    be the counit of the adjunction 
    $\iotarnzero \dashv S^\perp$. 
    Let 
    \[
      \infty \colon \iotaemptyempty \pr \Rightarrow \id_{\AVB^\vop(n)}  
    \]
    be the counit of the adjunction 
    $\iotaemptyempty \dashv \pr$.
\end{construction}

\begin{proposition}
    We have a unique (up to an anima of contractable choices) homotopy
    \[
      \theta \circ \iotarnzero(\sigma_\infty) \simeq \infty , 
    \]
    or equivalently the following square has a unique (up to an anima of contractable choices) filler 
    \[\begin{tikzcd}
        {\iotaemptyempty \pr} & {\id_{\AVB^\vop(n)}} \\
        {\iotarnzero \pr} & {\iotarnzero S^\perp}
        \arrow["{\iotarnzero(\sigma_\infty)}"', Rightarrow, from=2-1, to=2-2]
        \arrow["\theta"', Rightarrow, from=2-2, to=1-2]
        \arrow["\infty", Rightarrow, from=2-1, to=1-1]
        \arrow["\infty", Rightarrow, from=1-1, to=1-2]
    \end{tikzcd}\label{diag:AVB(n)square}\]\label{prop: thomatinftyinavb(n)}
\end{proposition}
\begin{proof}
    With the help of Lemma~\ref{lemma:favouritelemma} we compute 
    \begin{eqnarray*}
        & & \Nat_{\Fun(\AVB^\vop(n), \AVB^\vop(n))}(\iotarnzero \pr, \id_{\AVB^\vop(n)}) \\ 
        & \simeq & \Nat_{\Fun(\AVB^\vop(n), \An)}(\pr, S^\perp)\\
        & \simeq & \Nat_{\Fun(\AVB^\vop(n), \An)}(\Map(\iotaemptyempty(\pt), -), S^\perp) \\
        & \simeq & S^\perp( \iotaemptyempty(\pt)) = \pt 
    \end{eqnarray*}
\end{proof}

\subsubsection{Proof of Proposition~\ref{prop: AVB(n)} Part 3}
We end this section by showing that the functor $\overleftarrow{\pr}$ maps the square~\ref{fig:endosquareavb(n)}
to a diagram of distinguised squares.

\begin{lemma}
    Let $(\overleftarrow{V},\overrightarrow{W},\alpha) \in \AffLin^\op(n)$. Then we have equivalences of functors 
    \begin{enumerate}
        \item \[
            \overleftarrow{\pr} \circ \iota_{(\overleftarrow{V},\overrightarrow{W},\alpha)} \simeq \iota_V \colon \An \to \AVB^\vop
        \]
        \item \[
            \overrightarrow{\pr} \circ \iota_{(\overleftarrow{V},\overrightarrow{W},\alpha)} \simeq \iota_W \colon \An \to \AVB
        \]
        
    \end{enumerate}
\label{lemma:iotasagree}
\end{lemma}
\begin{proof}
    Since $\overleftarrow{\pr},\overrightarrow{\pr}$ are functors of Cartesian fibrations, we only have to provide equivalences 
    of the values of the point.
    But by construction the left hand and right hand sides agree on the point. 
\end{proof}

By the last lemma we compute that the postcomposition of square~\ref{fig:endosquareavb(n)} with $\overleftarrow{\pr}$ is equivalent 
to the square 
\[\begin{tikzcd}
	{\iota_{\emptyset}\pr} & {\overleftarrow{\pr}} \\
	{\iota_{\R^n}\pr} & {\iota_{\R^n}S^\perp}
	\arrow["\infty", Rightarrow, from=2-1, to=1-1]
	\arrow["\infty", Rightarrow, from=1-1, to=1-2]
	\arrow["{\iota_{\R^n}\sigma^\infty}"', Rightarrow, from=2-1, to=2-2]
	\arrow["{\overleftarrow{\pr}(\theta)}"', Rightarrow, from=2-2, to=1-2]
\end{tikzcd}\]

Consider the counit $\epsilon$ of the adjunction $\iota_0 S \dashv \id_{\AVB}$ of Construction~\ref{constr:thomdiag1}.
On an object $(X,W)$ the morphism $\epsilon_{(X,W)}$ consists of the data of a projection of anima 
\[
\pi \colon S^W \to X    
\]
and an affine linear morphism of vector bundles over $S^W$:
\[
\vartheta \colon 0 \to \pi^*W.    
\]

We want to show that the value of the above square on an object $(X,V,W,\alpha \colon V \oplus W \to \R^n_X)$ is distinguished.
The missing piece is to identify the value of $\overleftarrow{\pr}(\theta)$.
It consists of the information of a morphism of anima 
\[
\pi \colon S^W \to X    
\]
and an affine linear morphism of vector bundles over $S^W$:
\[
\phi \colon \pi^*V \to \R^n_{S^W}.    
\]
In fact $\theta$ itself consists of an additional data of 
an affine linear morphism of vector bundles over $S^W$:
\[
\psi \colon 0 \to \pi^*W
\]
and a compatibility datum 
\[\begin{tikzcd}
	{\pi^*V \oplus 0} & {\pi^*V \oplus \pi^*W} \\
	{\R^n_{S^W} \oplus 0} & {\R^n_{S^W}}
	\arrow["{\pi^*V \oplus \psi}", from=1-1, to=1-2]
	\arrow["{\pi^*\alpha}", from=1-2, to=2-2]
	\arrow["{\mathrm{triv}}"', from=2-1, to=2-2]
	\arrow["{\phi\oplus 0}"', from=1-1, to=2-1]
\end{tikzcd}.\]
To show that the square in question is distinguished we
can assume that $\phi$ is given by $\pi^*V \oplus \psi$.
By \`subtracting' $V$ we only need to show that the squares 
\[\begin{tikzcd}
	{(X,\emptyset)} & {(X,0)} \\
	{(X,W)} & {(S^W,W)}
	\arrow["\infty", from=1-1, to=1-2]
	\arrow["{(\sigma_\infty,\id)}"', from=2-1, to=2-2]
	\arrow["\infty", from=2-1, to=1-1]
	\arrow["{(\pi,\psi)}"', from=2-2, to=1-2]
\end{tikzcd}\]
and 
\[\begin{tikzcd}
	{(X,\emptyset)} & {(X,0)} \\
	{(X,W)} & {(S^W,W)}
	\arrow["\infty", from=1-1, to=1-2]
	\arrow["{(\sigma_\infty,\id)}"', from=2-1, to=2-2]
	\arrow["\infty", from=2-1, to=1-1]
	\arrow["{(\pi,\vartheta)}"', from=2-2, to=1-2]
\end{tikzcd}\]
are equivalent. In other words we need to show
the above square is simple distinguished. 
The following lemma answers this affirmatively.
With Lemma~\ref{lemma:iotasagree} and the definition $S^\perp = S\circ \overrightarrow{\pr}$ in mind we have 
\begin{lemma}
    The natural transformations 
    \[
    \overrightarrow{\pr}(\theta)   \colon \iota_n S^\perp \Rightarrow \overrightarrow{\pr} 
    \]
    and 
    \[
    \epsilon_{\overrightarrow{\pr}} \colon \iota_n S^\perp \Rightarrow \overrightarrow{\pr} 
    \]
    are equivalent.
\end{lemma}

\begin{proof}
    By Lemma~\ref{lemma:iotasagree} we see that $\overrightarrow{\pr}(\theta)$ is a natural transformation 
    \[
      \overrightarrow{\pr} \circ \iotarnzero \circ S^\perp \Rightarrow \overrightarrow{\pr}.  
    \]
    Under the adjunction $\iota_n = \overrightarrow{\pr} \circ \iotarnzero \dashv S$ its adjoint transformation 
    \[
    (\overrightarrow{\pr}(\theta))^\flat \colon S^\perp \Rightarrow S \circ \overrightarrow{\pr} = S^\perp   
    \]
    is computed as the composite
    \[\begin{tikzcd}
        {S^\perp} & {S\circ\overrightarrow{\pr} \circ \iotarnzero \circ S^\perp} & {S \circ \overrightarrow{\pr}} & {S^\perp}
        \arrow["{\eta_{S^\perp}}", Rightarrow, from=1-1, to=1-2]
        \arrow["{S(\overrightarrow{\pr}(\theta))}", Rightarrow, from=1-2, to=1-3]
        \arrow[Rightarrow, no head, from=1-3, to=1-4],
    \end{tikzcd}\]
    where 
    $ \eta_{S^\perp}$ is the value of the unit    
    \[
        \eta \colon \id_{\An} \Rightarrow S\iota_0
    \]
    of the adjunction $\iota_n = \overrightarrow{\pr} \circ \iotarnzero \dashv S$
    on the values of the functor $S^\perp$.
    We first compute $\eta$.
    We claim that $\eta_X$ is given by the $0$-section 
    \[
    X \to S^0_X = S^0 \times X = \{0, \infty\} \times X. 
    \]
    By Yoneda it is enough to check this for $X = \pt$.
    Since $S$ is corepresented by $\iota_0(\pt)$ the value $\eta_\pt$ is determined by the functor $\iota_0$:
    \[\begin{tikzcd}
        \pt & {\Map(\pt,\pt)} \\
        {S(\iota_0(\pt))} & {\Map((\pt,0),(\pt,0))}
        \arrow["{\iota_0}", from=1-2, to=2-2]
        \arrow["{\eta_\pt}", from=1-1, to=2-1]
        \arrow["\simeq", from=1-1, to=1-2]
        \arrow["\simeq"', from=2-1, to=2-2]
    \end{tikzcd}\]
    Under the lower horizontal identification the morphism $\id_{(\pt,0)}$ corresponds to $0 \in S^0_\pt \simeq \{0,\infty\}$.

    We claim that $(\overrightarrow{\pr}(\theta))^\flat$ is the identity transformation $S^\perp \Rightarrow S^\perp$.
    Since $S^\perp$ is corepresented by $\iotarnzero(\pt)$ we only have to verify this on $\iotarnzero(\pt)$, that is 
    we want to understand the map
    \[
    S(S^0, 0) \xrightarrow{S(\overrightarrow{\pr}(\theta_{\iotarnzero(\pt)}))} S(\pt, 0)
    \]
    over the zero section 
    \[
    S^0 \to S(S^0, 0).
    \]
    To pick a point $a$ in the zero section of $S(S^0, 0)$ we need to pick a point $a \in S^0$
    and consider the induced morphism 
    \[
        (\pt,0) \xto{\iota_0(a)} (S^0,0)
    \]
    under $\iota_0$.
    To understand the composite 
    \[
        (\pt,0) \xto{\iota_0(a)} (S^0,0) \xto{\overrightarrow{\pr}(\theta)} (\pt, 0)
    \]
    we observe that we can lift $\iota_0(a)$ along $\overrightarrow{\pr}$, namely by 
    \[
      \iotarnzero(a) \colon \iotarnzero(\pt) \to \iotarnzero(S^0).  
    \]
    The composite $\overrightarrow{\pr}(\theta) \circ \iota_0(a)$ is then equivalent to the image 
    of $\theta \circ \iotarnzero(a)$ under $\overrightarrow{\pr}$.
    But 
    \[
        \theta \circ \iotarnzero(a) \colon \iotarnzero(\pt) \to \iotarnzero(\pt) 
    \]
    is the left adjoint morphism $a^\sharp$ of 
    \[
      a \colon \pt \to S^\perp(\iotarnzero(\pt)) = S^0.  
    \]
    Since $\overrightarrow{\pr}$ is an equivalence between the mapping anima $\Map(\iotarnzero(\pt)), \iotarnzero(\pt)$ 
    and $\Map(\iota_0(\pt), \iota_0(\pt))$, we can recover $a$ out of $\overrightarrow{\pr}(a^\sharp)$.
    That is we have shown that the natural transformation $(\overrightarrow{\pr}(\theta))^\flat$ is invertible. 
    But there is only one invertible natural transformation $S^\perp \Rightarrow S^\perp$ namely the identity.
    The left adjoint natural transformation $\phi^\sharp \colon \iota_n S^\perp \Rightarrow \overrightarrow{\pr}$ of
    a transformation $\phi \colon S^\perp \Rightarrow S \circ \overrightarrow{\pr}$ is computed by the composite 
    \[\begin{tikzcd}
        {\iota_n\circ S^\perp} & {\iota_n \circ S \circ \overrightarrow{\pr}} & {\overrightarrow{\pr}}
        \arrow["{\iota_n \phi}", Rightarrow, from=1-1, to=1-2]
        \arrow["{\epsilon_{\overrightarrow{\pr}}}", Rightarrow, from=1-2, to=1-3]
    \end{tikzcd}\]
    Thus 
    \begin{eqnarray*}
        \overrightarrow{\pr}(\theta) & \simeq & ((\overrightarrow{\pr}(\theta))^\flat)^\sharp \\
        & \simeq & \id^\sharp \\
        & \simeq & \epsilon_{\overrightarrow{\pr}} \circ \iota_n(\id) \\
        & \simeq & \epsilon_{\overrightarrow{\pr}} \circ \id \\ 
        & \simeq & \epsilon_{\overrightarrow{\pr}}
    \end{eqnarray*}  
\end{proof}
