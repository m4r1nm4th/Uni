\section{Introduction}

Cohomology theories arise in various contexts in mathematics and are a powerful invariant of the objects that one wants to study.
Most cohomology theories are homotopy theoretic in nature and the first mathematicians that axiomatized cohomology theories in the most
basic setting of Homotopy Theory, namely in the category of homotopy types or anima, were Eilenberg and Steenrod.
In their formulation of the axioms they used a grading of cohomology theoires by the integers.
But in many other contexts of mathematics and even within Homotopy Theory it is common to use a different grading.
To name a few, Motivic Homotopy Theory, Hodge Theory and Equivariant Homotopy Theory. 
For example in Equivariant Homotopy Theory it is common to 
grade cohomology theories over the representation of a fixed group $G$.
These can be thought of as (equivariant) vector bundles over the 
terminal $G$-anima $G/G$.
This is analogues to to how a natural number can be thought of as the dimension of a vector bundle over the point in anima.
Inspired by this analogy we introduce the notion of genuine cohomology theories (GCTs).
We now give a quick description of what a genuine cohomology theory essentially is.

A \emph{genuine cohomology theory (in the category of anima with coefficients in abelian groups)} 
consists of a collection $E^V(X)$ of abelian groups for each compact anima $X$ and vector bundle $V$ over $X$
\footnote{A vector bundle $V$ over the anima $X$ is a map $V \colon X \to \coprod_n BO(n)$.},
morphisms 
\[f^* := E^W(f) \colon E^W(Y) \to E^{f^*W}(X)\] for each map $f \colon X \to Y$ of anima and vector bundle $W$ over 
$Y$,
and 
\[\alpha_* := E^\alpha(X) \colon E^V(X) \to E^W(X)\] for each map $\alpha \colon V \to W$ between vector bundles $V,W$ over $X$, 
that satisfy various compatibility requirements, most importantly 
a certain map 
\[
E^V(X) \to E^{V \oplus W}(S^W,X)    
\]
is an equivalence, which we will call the \emph{Thom isomorphism}.
We will give a more detailed description of GCTs in Section~\ref{section:Genuinecohomologytheories}.

Essential to GCTs is the grading over vector bundles, but if we restrict a GCT $E$ only to trivial bundles 
we obtain a (classical) cohomology theory
\[
  E^n(X) := E^{\R^n}(X).  
\]

Our main result states that every (classical) cohomology theory arises this way.
That is, it extends to a GCT and furthermore this extension to a GCT is unique.
In other words:

\begin{theorem}\label{thm:introGCTs}Restriction of a genuine cohomology theory to a (classical) cohomology theory is 
    an equivalence of categories 
\[
\{\mathrm{GCTs\ on\ \An^\omega\ with\ coefficients\ in\ Ab}\} \simeq 
\{\mathrm{Cohomology\ Theories\ on\ \An^\omega}\}.\]
\end{theorem}

A key insight in this paper is that we generalize GCTs 
so that we allow them to take values in any arbitrary pointed $\infty$-category
instead of abelian groups.
To keep track of the additional coherences that appear when working with $\infty$-categories, we model 
GCTs as certain functors out of a category $\AVB^\vop|_{\An^\omega}$ that keeps track of the 
combinatorics of anima and vector bundles. 
In this greater generality of working with arbitrary pointed $\infty$-category as the target of GCTs, we are 
able to prove a much stronger result than Theorem~\ref{thm:introGCTs}.

\begin{theorem}\label{thm:introUniversalGCT} There exists a universal GCT
    \[\Th^- \colon \AVB^\vop|_{\An^\omega} \to \Sp^\omega. \]
That is, for every GCT $E \colon \AVB^\vop|_{\An^\omega} \to \cD^\op$ there exists 
a unique functor $\mathcal E \colon \Sp^\omega \to \cD^\op$ such that we have a factorization
\[\begin{tikzcd}
	{\AVB^\vop|_{\An^\omega}} && {\Sp^\omega} \\
	& {\mathcal D^\op}
	\arrow["{\Th^-}", from=1-1, to=1-3]
	\arrow["{\mathcal E}", dashed, from=1-3, to=2-2]
	\arrow["E"', from=1-1, to=2-2].
\end{tikzcd}.\]
\end{theorem}
A more detailed version of Theorem~\ref{thm:introUniversalGCT} is given by 
Theorem~\ref{thrm:dream} in Section~\ref{section:spectraGenerated}. There we give a 
precise analysis of which functors $\mathcal E$ arise through Theorem~\ref{thm:introUniversalGCT}.
For example if we let $\cD = \Ab$, then GCTs $E$ with values in $\Ab$ induce 
cohomology theories $\mathcal E \colon \Sp^\omega \to \Ab^\op$ and that is how 
Theorem~\ref{thm:introUniversalGCT} proves Theorem~\ref{thm:introGCTs}.
Another important case is $\cD = \An_*$, then the functors $\mathcal E \colon \Sp^\omega \to \An_*^\op$
that arise are excisive functors, and therefore $GCTs$ with values in $\An_*$ are spectra.

\medskip

We now give some background on some objects and ideas that appear in the formulation and 
proof of Theorem~\ref{thm:introUniversalGCT}.

First of all the category $\AVB^\vop|_{\An^\omega}$ on which we model GCTs can be thought of as having objects given by pairs $(X,-V)$ where $X$ is a compact anima and $V$ is a vector bundle over $X$.
% In Section~\ref{sec:AVBs} we construct different kinds of categories of anima and vector bundles. 
% The most important categories are called $\AVB$ and $\AVB^\vop$. 
% Both can be seen as categories of pairs of anima and vector bundles over them, but the main difference is 
% the functoriality in the vector bundle.
% While a morphism in $\AVB$ consists of the data of a morphism $\alpha \colon V \to f^*W$ into 
% the pulled back vector bundle $f^*W$, 
A morphism $(X,-V) \to (Y,-W)$ in $\AVB^\vop$ contains the data of a morhpism $f \colon X \to Y$ of anima and a morphism 
$\alpha \colon f^*W \to V$ out of the pulled back bundle.
Therefore we should think of morphism in $\AVB^\vop$ as morphisms between the `duals' of the vector bundles $V$, $f^*W$.
This is the reason why we use the notation $(X,-V)$ for objects of $\AVB^\vop$ and think of $-V$ as a
variable that keeps track of how to (twisted) desuspend $X$.

Another nodal point about the category $\AVB^\vop|_{\An^\omega}$ is that the morphisms between 
vector bundles are affine linear. This is important since the aforementioned Thom isomorphism is 
induced by the Thom diagonal, which is a strict affine linear map between vector bundles.
We introduce affine linear maps in depth in Section~\ref{sec:AffLin}, but one can think of 
an affine linear morphism of vector spaces $\phi \colon V \to W$ as the composition 
of an isometric linear embedding $f \colon V \to W$ followed by a translation $\tau \colon W \to W$ along a vector 
$w$ out of the orthogonal complement $f(V)^\perp \cup \{\infty\}$ of the image of $f$.
Since we also want to allow $w = \infty$, we cannot model affine linear morphisms as set theoretic maps between vector spaces. 
They are best understood as a subspace of the space of pointed continuous functions between the one point compactifications $S^V,S^W$.

Let $V$ be a vector bundle over $X$. We have an associated spherical fibration $\pi \colon S^V \to X$
which one can think of as the fiberwise one point compactification of $V$. 
Over $S^V$ there is a specific affine linear map $\theta \colon 0 \to \pi^*V$ from the zero bundle into the 
pulled back bundle of $V$, that is called the \emph{Thom diagonal} that we want to describe now. 
An affine linear morphism from the $0$ vector space into another vector space $W$ is fully 
determined by its value of the $0$ vector, which can be any element of $S^W$.
Therefore a morphism $0 \to \pi^*V$ is given by a section of $S^{\pi^*V} \to S^V$.
But $S^{\pi^*V} \simeq S^V \times S^V$ and the Thom diagonal corresponds under this identification 
unsurprisingly with the diagonal section.
The map $(S^V, -V) \xto{\theta} (S^V, -0) \xto{\pi} (X, -0)$ which we sometimes also call the Thom diagonal induces on associated 
Thom spectra of the negative bundle the projection to the right summand of the splitting 
\[
  (S^V)^{-V} \simeq X^{-V} \oplus X^{-0}.
\]
It turns out that the functor $(X,-V) \mapsto X^{-V}$ is the universal GCT and therefore the universal functor 
that among others splits the Thom diagonal.

\medskip

BIS HIER HAB ICH DEN TEXT VERAENDERT

The key idea of the proof of Theorem~\ref{thrm:dream} is to construct a functorial Thom diagonal.
This construction is strictly not possible for the category $\AVB^\vop$, but one can construct it
functorially for the category $\AVB$. We explain this in Section~\ref{section:ThomDiagonal}.
In Section~\ref{sec:AVB(n)} we introduce categories $\AVB^\vop(n)$.
In this paper they play the technical role, that they filter the category $\AVB^\vop$ over the category $\An^\omega$, that is 
\[
\colim_n \AVB^\vop(n)|_{\An^\omega} \simeq \AVB^\vop|_{\An^\omega},    
\]
but the categories $\AVB^\vop(n)$ also admit a functiorial construction of the Thom diagonal similar to the category $\AVB$.

In Section~\ref{section:spectraGenerated} we study the Thom spectrum $X^{-V}$ of the negative vector bundle $(X,-V)$, which 
is a functor $\AVB^\vop \to \Sp$. Phrased differently, Thereom~\ref{thrm:dream} shows that the 
negative Thom spectrum functor satisfies a universal property. 
Simply put, its the universal functor that preserves certain colimits and splits the 
Thom diagonal.

In Section~\ref{section:Genuinecohomologytheories}




