%%%%%%%%%%%% ALLGEMEINER STUFF %%%%%%%%%%%%%%%%%%%%%%%%%%%
%\documentclass[a4paper,12pt, pdftex]{article}
\documentclass[a4paper, 12pt]{amsart}

%\usepackage{lmodern} %schönere Buchstaben
%\usepackage{ngerman} %Deutsch
%\usepackage{amsmath} %Mathe
%\usepackage{amssymb} %Symbole
%\usepackage{amsthm} %Theoreme
\usepackage{latexsym, amscd, amsfonts, eucal, mathrsfs, amsmath, amssymb, amsthm, xr, stmaryrd, tikz-cd, bbold}
%\usetikzlibrary{shapes.geometric}
%\usetikzlibrary{arrows.meta}
%\tikzset{commutative diagrams/arrow style=math font}

\usepackage[utf8]{inputenc} %Umlaute im Quelltext
\usepackage[T1]{fontenc} %Umlaute in der Pdf Suche
%\usepackage[pdftex, allcolors=black]{hyperref} %Clickable Links

%\usepackage{faktor} %Quotienten algebraischer Objekte
\usepackage{graphicx} %Grafiken
\usepackage{setspace} %Zeilenabstände
%\usepackage{tikz} %Zeichnungen
\usepackage{enumerate} %Aufzaehlungen
\usepackage{url} %webpages in references
%\usepackage{mathrsfs}
%\usetikzlibrary{arrows}


%Die Aufzählungen schöner numerieren:
%\renewcommand\theenumi{\alph{enumi})}
%\renewcommand\labelenumi{\theenumi}
%\setlength{\parindent}{0em} %Keine Einrückungen bei neuen Absätzen
\usepackage{parskip} %Macht keine Einrückungen bei neuen Absätzen und die Absätze sehen scheisse aus
%\usepackage[arrow,curve]{xy}
%\usepackage{diagrams} %Kommutative Diagramme


\usepackage{geometry}

 \geometry{
 a4paper,
 total={180mm,207mm}, %default = {120mm, 207mm}
 left=15mm, %default = 35mm
 top=35mm,
 headsep=12mm,
 footskip = 15mm,
 }

 \iffalse
 \geometry{
 a4paper,
 total={120mm,207mm}, %default = {120mm, 207mm}
 left=35mm, %default = 35mm
 top=35mm,
 headsep=12mm,
 footskip = 15mm,
 }
 \fi
% \usepackage{fancyhdr}
% \fancyhead{}
% \fancyhead[LE]{$p$-Adic Stable Homotopy Theory and Anderson Duality}
% \fancyhead[CO]{Felix Janssen}
% \fancyhead[R]{Chapter \thesection}
% \fancyfoot{}
% \fancyfoot[C]{\thepage}



\usepackage{pdfpages}
\usepackage{hyperref}
%\iffalse
\hypersetup{
    unicode=false,          % non-Latin characters in Acrobat’s bookmarks
    pdftoolbar=true,        % show Acrobat’s toolbar?
    pdfmenubar=true,        % show Acrobat’s menu?
    pdffitwindow=false,     % window fit to page when opened
    pdfstartview={FitH},    % fits the width of the page to the window
    pdftitle={My title},    % title
    pdfauthor={Author},     % author
    pdfsubject={Subject},   % subject of the document
    pdfcreator={Creator},   % creator of the document
    pdfproducer={Producer}, % producer of the document
    pdfkeywords={keyword1, key2, key3}, % list of keywords
    pdfnewwindow=true,      % links in new PDF window
    colorlinks=true,       % false: boxed links; true: colored links
    linkcolor=red,          % color of internal links (change box color with linkbordercolor)
    linkbordercolor = red,
    citecolor=green,        % color of links to bibliography
    filecolor=magenta,      % color of file links
    urlcolor=cyan,           % color of external links
    urlbordercolor={1 1 1}  % color of border around links
}
%\fi

%\oddsidemargin=0.5in \evensidemargin=0.5in \textwidth=6in
%\textheight=8.5in

%%%%%%%font designs%%%%%%%%%%just untab a group%%%%%%%%%%

%\usepackage[cmintegrals,cmbraces]{newtxmath}
%\usepackage{ebgaramond-maths}
%\usepackage[T1]{fontenc}

%\usepackage{libertine}
%\usepackage{libertinust1math}
%\usepackage[T1]{fontenc}

%\usepackage[T1]{fontenc}
%\usepackage{newpxtext,eulerpx}

%\usepackage[default,regular,black]{sourceserifpro}
%\usepackage[T1]{fontenc}

%\usepackage[XCharter]{newtxmath}
\usepackage[T1]{fontenc}
\usepackage{textcomp}
%\usepackage[utf8]{inputenc}

%%%%%%%%%%%%%%%%%%%%%% THEOREME %%%%%%%%%%%%%%%%%%%%%%%%%%%%
\renewcommand{\proofname}{Proof}

\theoremstyle{plain}
\newtheorem{thm}{Theorem}
\newtheorem{lem}[thm]{Lemma}
\newtheorem{cor}[thm]{Corollary}
\newtheorem{prop}[thm]{Proposition}

\theoremstyle{remark}
\newtheorem{rem}{Remark}

\theoremstyle{definition}
\newtheorem{defn}[thm]{Definition}
\newtheorem{exam}[thm]{Example}
\newtheorem{exer}[thm]{Exercise}
%%%%%%%%%%%%%%% abbreviations %%%%%%%%%%%%%%%%%%%
\newcommand{\cal}{\mathcal}
\newcommand{\cA}{\mathcal A}
\newcommand{\cB}{\mathcal B}
\newcommand{\cC}{\mathcal C}
\newcommand{\cD}{\mathcal D}
\newcommand{\cE}{\mathcal E}
\newcommand{\cF}{\mathcal F}
\newcommand{\cG}{\mathcal G}
\newcommand{\cH}{\mathcal H}
\newcommand{\spaces}{\mathcal S}
\DeclareMathOperator{\globalspaces}{\spaces_{glob}}
%\newcommand{\cO}{\mathcal O}
\DeclareMathOperator{\sSet}{sSet}
\DeclareMathOperator{\Orb}{Orb}
\newcommand{\onto}{\twoheadrightarrow}
\newcommand{\into}{\hookrightarrow}
%\DeclareMathOperator{\Psh}{Psh}
\newcommand{\h}{\mathrm{h}}
\DeclareMathOperator{\Th}{Th}
\DeclareMathOperator{\CAlg}{CAlg}
\newcommand{\twiggle}{\rightsquigarrow}

%%%%%%%%%%%%%%% MATHEMATISCHE MAKROS %%%%%%%%%%%%%%%%
\newcommand{\p}{\mathbb P}
\newcommand{\R}{\mathbb R}
\newcommand{\N}{\mathbb N}
\newcommand{\C}{\mathbb C}
\newcommand{\Z}{\mathbb Z}
\newcommand{\Q}{\mathbb Q}
\newcommand{\F}{\mathbb F}
\newcommand{\D}{\mathbb D}
\newcommand{\E}{\mathbb E}
\renewcommand{\S}{\mathbb S}
\newcommand{\aff}{\mathbb A}
\newcommand{\Sch}{\mathrm{Sch}}
\newcommand{\bB}{\mathbb B}
\DeclareMathOperator{\AffLin}{AffLin}
%%% \newcommand{\cX}{\mathcal{X}}
\newcommand{\catinfty}{\mathcal C\mathrm{at}_{\infty}}
\newcommand\f[3]{#1 \colon #2 \rightarrow #3}
% \newcommand\xto[1]{\xrightarrow{#1}}
\newcommand\xot[1]{\xleftarrow{#1}}
\newcommand{\op}{\mathrm{op}}
\DeclareMathOperator{\Ch}{Ch}
\newcommand{\acts}{\circlearrowright}
\newcommand{\stca}{\circlearrowleft}
\newcommand{\normal}{\vartriangleleft}
\newcommand{\tate}[1]{^{t#1}}
\renewcommand{\L}{\mathbb{L}}
\newcommand{\trunc}[1]{\tau_{\leq #1}}
\newcommand{\Pre}{\mathrm{Pre}}
\DeclareMathOperator{\thh}{THH}
\DeclareMathOperator{\tc}{TC}
\DeclareMathOperator{\hh}{HH}
\DeclareMathOperator{\Lan}{Lan}
\DeclareMathOperator{\Ran}{Ran}
\newcommand{\cotangent}[2]{\mathrm{L}_{#1/#2}}
% \newcommand{\omega}{\mathrm{fin}}
\DeclareMathOperator{\Rex}{Rex}
\DeclareMathOperator{\Lex}{Lex}


%%%%%%%%%%%%%%%%Tikzcd makros%%%%%%%%%%%%%%%%%%%%%%%%%
\newcommand{\adj}[4]{
  \begin{tikzcd}[ampersand replacement=\&]
  #1\arrow[r, shift left=1ex, "#3"{name=G}] \& #2 \arrow[l, shift left=.5ex, "#4"{name=F}]
            \arrow[phantom, from=F, to=G, , "\scriptscriptstyle\boldsymbol{\top}"]
  \end{tikzcd}%
}

\newcommand{\sq}[8]{
\begin{tikzcd}[ampersand replacement=\&]
  #1 \arrow[r, "#5"] \arrow[d, "#6"'] \& #2 \arrow[d, "#7"]                                   \\
#3 \arrow[r, "#8"']                 \& #4
  \end{tikzcd}%
}

\newcommand{\po}[8]{
\begin{tikzcd}[ampersand replacement=\&]
  #1 \arrow[r, "#5"] \arrow[d, "#6"'] \& #2 \arrow[d, "#7"]                                   \\
#3 \arrow[r, "#8"']                 \& #4 \arrow[lu, "\ulcorner", phantom, very near start]
  \end{tikzcd}%
}

\newcommand{\pb}[8]{
\begin{tikzcd}[ampersand replacement=\&]
#4 \arrow[r, "#8"] \arrow[d, "#7"'] \arrow[rd, "\lrcorner", phantom, very near start] \& #2 \arrow[d, "#5"]     \\
#3 \arrow[r, "#6"']                                                                   \& #1
\end{tikzcd}%
}

\newcommand{\zseq}[5]{
\begin{tikzcd}[ampersand replacement=\&]
\dots \arrow[r, "#3"] \& #1 \arrow[r, "#4"] \& #2 \arrow[r, "#5"] \& \dots
\end{tikzcd}
}

\newcommand{\nseq}[4]{
\begin{tikzcd}[ampersand replacement=\&]
#1 \arrow[r, "#3"] \& #2 \arrow[r, "#4"] \& \dots
\end{tikzcd}
}

\newcommand{\tow}[4]{
\begin{tikzcd}[ampersand replacement=\&]
\vdots \arrow[d, "#4"] \\
#2 \arrow[d, "#3"]     \\
#1
\end{tikzcd}
}



%%%%%%%%%%%%%%%%%%%%%%%%%%%%%%%%%%%%%%%%%%%%%%%%%%%%%%%%%%%



\newcommand{\substeq}{\operatorname{\subseteq}}
\newcommand{\s}{\mathfrak{S}}
\newcommand{\bs}{\backslash}
\newcommand{\tensor}{\otimes}
\newcommand{\del}{\partial}

\DeclareMathOperator{\rk}{rk}
\newcommand{\fib}{\operatorname{fib}}
\newcommand{\cof}{\operatorname{cof}}
\newcommand{\GL}{\operatorname{GL}}
\newcommand{\Ext}{\operatorname{Ext}}
\newcommand{\Tor}{\operatorname{Tor}}
\newcommand{\supp}{\operatorname{supp}}
\newcommand{\Mat}{{\operatorname{Mat}}}
\newcommand{\tr}{{\operatorname{tr}}}
\newcommand{\colim}{\operatorname{colim}}
%\newcommand{\sym}{\operatorname{Sym}}
\newcommand{\id}{\operatorname{id}}
\newcommand{\im}{\operatorname{im}}
%\newcommand{\lh}{\operatorname{LH}}
%\newcommand{\mathspan}{\operatorname{span}}
\newcommand{\Spec}{\operatorname{Spec}}
%\newcommand{\diag}{\operatorname{diag}}
%\newcommand{\dirlim}{\underrightarrow\lim}
\newcommand{\coker}{\operatorname{coker}}
\newcommand{\Hom}{\operatorname{Hom}}
\newcommand{\End}{\operatorname{End}}
\DeclareMathOperator{\map}{map}
\DeclareMathOperator{\Map}{Map}
\DeclareMathOperator{\Set}{Set}
\DeclareMathOperator{\Fun}{Fun}
\DeclareMathOperator{\Sp}{Sp}
\DeclareMathOperator{\Nat}{Nat}
\DeclareMathOperator{\Ab}{Ab}
\DeclareMathOperator{\Top}{Top}
\DeclareMathOperator{\TopGrpd}{TopGrpd}
\DeclareMathOperator{\pt}{pt}
\DeclareMathOperator{\Span}{Span}

\let\oldphi\phi
\let\phi\varphi
\let\varphi\oldphi
\renewcommand{\epsilon}{\varepsilon}
\renewcommand{\emptyset}{\varnothing}
\DeclareMathOperator{\Ind}{Ind}
\newcommand{\Psh}{\mathcal P}
