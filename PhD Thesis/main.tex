\documentclass{article}

% Pakete
\usepackage{amsmath} % Mathematische Symbole und Umgebungen
\usepackage{amssymb} % Zusätzliche mathematische Symbole
\usepackage{amsthm} % Theorem-Umgebung
\usepackage{tikz-cd} % Kommutative Diagramme
\usepackage{enumitem} % Anpassbare Aufzählungen
\usepackage{hyperref} % Hyperlinks
\usepackage{showkeys}
\usetikzlibrary{decorations.pathmorphing}

% Abkürzungen

\newcommand{\xto}{\xrightarrow}
\newcommand{\xot}{\xleftarrow}


\newcommand{\R}{\mathbb{R}} % Reelle Zahlen
\newcommand{\N}{\mathbb{N}} % Natürliche Zahlen
\newcommand{\Z}{\mathbb{Z}} % Ganze Zahlen
\newcommand{\Q}{\mathbb{Q}} % Rationale Zahlen
\newcommand{\C}{\mathbb{C}} % Komplexe Zahlen
\newcommand{\F}{\mathbb{F}}

\newcommand{\cC}{\mathcal{C}}
\newcommand{\cD}{\mathcal{D}}
\newcommand{\cE}{\mathcal{E}}


\newcommand{\cO}{\AffLin}
\newcommand{\cX}{\AVB^{\vop}}

\newcommand{\Ab}{\mathrm{Ab}}
\newcommand{\Psh}{\mathcal{P}}



\newcommand{\pt}{\mathrm{pt}}
\newcommand{\op}{\mathrm{op}}
\newcommand{\vop}{\mathrm{vop}}
\newcommand{\pr}{\mathrm{pr}}
\newcommand{\id}{\mathrm{id}}
\newcommand{\ev}{\mathrm{ev}}
\newcommand{\Pow}{\mathrm{P}}


\newcommand{\iotaemptyempty}{\iota_{\overleftarrow{\emptyset},\overrightarrow{\emptyset}}}
\newcommand{\iotarnzero}{\iota_{\overleftarrow{\R^n},\overrightarrow{0}}}


\DeclareMathOperator{\AVB}{AVB}
\DeclareMathOperator{\AWB}{AWB}
\DeclareMathOperator{\An}{An}
\DeclareMathOperator{\Sp}{Sp}
\newcommand{\catinfty}{\mathcal{C}\mathrm{at}_{\infty}}
\DeclareMathOperator{\Fun}{Fun}
\DeclareMathOperator{\AffLin}{AffLin}
\DeclareMathOperator{\res}{res}
\DeclareMathOperator{\Map}{Map}
\DeclareMathOperator*{\colim}{colim}
\DeclareMathOperator*{\fib}{fib}
\DeclareMathOperator{\Th}{Th}
\DeclareMathOperator{\Span}{Span}
\DeclareMathOperator{\Nat}{Nat}



% Theorems
\newtheorem{theorem}{Theorem}
\newtheorem{proposition}{Proposition}
\newtheorem{lemma}{Lemma}
\newtheorem{corollary}{Corollary}
\newtheorem{definition}{Definition}
\newtheorem{example}{Example}
\newtheorem{construction}{Construction}
\newtheorem{remark}{Remark}

\begin{document}

\title{Genuine Cohomology Theories and the Thom Diagonal}
\author{Marin Janssen}
\date{\today}

\maketitle

\begin{abstract}
    We will present a set of axioms that describe cohomology theories but with an extended grading over vector bundles instead of integers.
    To show that these axioms are equivalent to the Eilenberg-Steenrod axioms we will give a generators-relation description of the category of finite spectra.
    The generators can be thought of a pair consisting of an anima and a vector bundle over it.
    In order to prove this, we make the Thom anima construction an endofunctor of our category, which can be thought of as a twisted suspension.
    It is connected to the identity via the Thom diagonal. The relation that one has to quotient by to obtain finite spectra is morally giving 
    by inverting the Thom diagonal.
\end{abstract}

\tableofcontents

% \section{Introduction}

Cohomology theories arise in various contexts in mathematics and are a powerful invariant of the objects that one wants to study.
Most cohomology theories are homotopy theoretic in nature and the first mathematicians that axiomatized cohomology theories in the most
basic setting of Homotopy Theory, namely in the category of homotopy types or anima, were Eilenberg and Steenrod.
In their formulation of the axioms they used a grading of cohomology theoires by the integers.
But in many other contexts of mathematics and even within Homotopy Theory it is common to use a different grading.
To name a few, Motivic Homotopy Theory, Hodge Theory and Equivariant Homotopy Theory. 
For example in Equivariant Homotopy Theory it is common to 
grade cohomology theories over the representation of a fixed group $G$.
These can be thought of as (equivariant) vector bundles over the 
terminal $G$-anima $G/G$.
This is analogues to to how a natural number can be thought of as the dimension of a vector bundle over the point in anima.
Inspired by this analogy we introduce the notion of genuine cohomology theories (GCTs).
We now give a quick description of what a genuine cohomology theory essentially is.

A \emph{genuine cohomology theory (in the category of anima with coefficients in abelian groups)} 
consists of a collection $E^V(X)$ of abelian groups for each compact anima $X$ and vector bundle $V$ over $X$
\footnote{A vector bundle $V$ over the anima $X$ is a map $V \colon X \to \coprod_n BO(n)$.},
morphisms 
\[f^* := E^W(f) \colon E^W(Y) \to E^{f^*W}(X)\] for each map $f \colon X \to Y$ of anima and vector bundle $W$ over 
$Y$,
and 
\[\alpha_* := E^\alpha(X) \colon E^V(X) \to E^W(X)\] for each map $\alpha \colon V \to W$ between vector bundles $V,W$ over $X$, 
that satisfy various compatibility requirements, most importantly 
a certain map 
\[
E^V(X) \to E^{V \oplus W}(S^W,X)    
\]
is an equivalence, which we will call the \emph{Thom isomorphism}.
We will give a more detailed description of GCTs in Section~\ref{section:Genuinecohomologytheories}.

Essential to GCTs is the grading over vector bundles, but if we restrict a GCT $E$ only to trivial bundles 
we obtain a (classical) cohomology theory
\[
  E^n(X) := E^{\R^n}(X).  
\]

Our main result states that every (classical) cohomology theory arises this way.
That is, it extends to a GCT and furthermore this extension to a GCT is unique.
In other words:

\begin{theorem}\label{thm:introGCTs}Restriction of a genuine cohomology theory to a (classical) cohomology theory is 
    an equivalence of categories 
\[
\{\mathrm{GCTs\ on\ \An^\omega\ with\ coefficients\ in\ Ab}\} \simeq 
\{\mathrm{Cohomology\ Theories\ on\ \An^\omega}\}.\]
\end{theorem}

A key insight in this paper is that we generalize GCTs 
so that we allow them to take values in any arbitrary pointed $\infty$-category
instead of abelian groups.
To keep track of the additional coherences that appear when working with $\infty$-categories, we model 
GCTs as certain functors out of a category $\AVB^\vop|_{\An^\omega}$ that keeps track of the 
combinatorics of anima and vector bundles. 
In this greater generality of working with arbitrary pointed $\infty$-category as the target of GCTs, we are 
able to prove a much stronger result than Theorem~\ref{thm:introGCTs}.

\begin{theorem}\label{thm:introUniversalGCT} There exists a universal GCT
    \[\Th^- \colon \AVB^\vop|_{\An^\omega} \to \Sp^\omega. \]
That is, for every GCT $E \colon \AVB^\vop|_{\An^\omega} \to \cD^\op$ there exists 
a unique functor $\mathcal E \colon \Sp^\omega \to \cD^\op$ such that we have a factorization
\[\begin{tikzcd}
	{\AVB^\vop|_{\An^\omega}} && {\Sp^\omega} \\
	& {\mathcal D^\op}
	\arrow["{\Th^-}", from=1-1, to=1-3]
	\arrow["{\mathcal E}", dashed, from=1-3, to=2-2]
	\arrow["E"', from=1-1, to=2-2].
\end{tikzcd}.\]
\end{theorem}
A more detailed version of Theorem~\ref{thm:introUniversalGCT} is given by 
Theorem~\ref{thrm:dream} in Section~\ref{section:spectraGenerated}. There we give a 
precise analysis of which functors $\mathcal E$ arise through Theorem~\ref{thm:introUniversalGCT}.
For example if we let $\cD = \Ab$, then GCTs $E$ with values in $\Ab$ induce 
cohomology theories $\mathcal E \colon \Sp^\omega \to \Ab^\op$ and that is how 
Theorem~\ref{thm:introUniversalGCT} proves Theorem~\ref{thm:introGCTs}.
Another important case is $\cD = \An_*$, then the functors $\mathcal E \colon \Sp^\omega \to \An_*^\op$
that arise are excisive functors, and therefore $GCTs$ with values in $\An_*$ are spectra.

\medskip

We now give some background on some objects and ideas that appear in the formulation and 
proof of Theorem~\ref{thm:introUniversalGCT}.

First of all the category $\AVB^\vop|_{\An^\omega}$ on which we model GCTs can be thought of as having objects given by pairs $(X,-V)$ where $X$ is a compact anima and $V$ is a vector bundle over $X$.
% In Section~\ref{sec:AVBs} we construct different kinds of categories of anima and vector bundles. 
% The most important categories are called $\AVB$ and $\AVB^\vop$. 
% Both can be seen as categories of pairs of anima and vector bundles over them, but the main difference is 
% the functoriality in the vector bundle.
% While a morphism in $\AVB$ consists of the data of a morphism $\alpha \colon V \to f^*W$ into 
% the pulled back vector bundle $f^*W$, 
A morphism $(X,-V) \to (Y,-W)$ in $\AVB^\vop$ contains the data of a morhpism $f \colon X \to Y$ of anima and a morphism 
$\alpha \colon f^*W \to V$ out of the pulled back bundle.
Therefore we should think of morphism in $\AVB^\vop$ as morphisms between the `duals' of the vector bundles $V$, $f^*W$.
This is the reason why we use the notation $(X,-V)$ for objects of $\AVB^\vop$ and think of $-V$ as a
variable that keeps track of how to (twisted) desuspend $X$.

Another nodal point about the category $\AVB^\vop|_{\An^\omega}$ is that the morphisms between 
vector bundles are affine linear. This is important since the aforementioned Thom isomorphism is 
induced by the Thom diagonal, which is a strict affine linear map between vector bundles.
We introduce affine linear maps in depth in Section~\ref{sec:AffLin}, but one can think of 
an affine linear morphism of vector spaces $\phi \colon V \to W$ as the composition 
of an isometric linear embedding $f \colon V \to W$ followed by a translation $\tau \colon W \to W$ along a vector 
$w$ out of the orthogonal complement $f(V)^\perp \cup \{\infty\}$ of the image of $f$.
Since we also want to allow $w = \infty$, we cannot model affine linear morphisms as set theoretic maps between vector spaces. 
They are best understood as a subspace of the space of pointed continuous functions between the one point compactifications $S^V,S^W$.

Let $V$ be a vector bundle over $X$. We have an associated spherical fibration $\pi \colon S^V \to X$
which one can think of as the fiberwise one point compactification of $V$. 
Over $S^V$ there is a specific affine linear map $\theta \colon 0 \to \pi^*V$ from the zero bundle into the 
pulled back bundle of $V$, that is called the \emph{Thom diagonal} that we want to describe now. 
An affine linear morphism from the $0$ vector space into another vector space $W$ is fully 
determined by its value of the $0$ vector, which can be any element of $S^W$.
Therefore a morphism $0 \to \pi^*V$ is given by a section of $S^{\pi^*V} \to S^V$.
But $S^{\pi^*V} \simeq S^V \times S^V$ and the Thom diagonal corresponds under this identification 
unsurprisingly with the diagonal section.
The map $(S^V, -V) \xto{\theta} (S^V, -0) \xto{\pi} (X, -0)$ which we sometimes also call the Thom diagonal induces on associated 
Thom spectra of the negative bundle the projection to the right summand of the splitting 
\[
  (S^V)^{-V} \simeq X^{-V} \oplus X^{-0}.
\]
It turns out that the functor $(X,-V) \mapsto X^{-V}$ is the universal GCT and therefore the universal functor 
that among others splits the Thom diagonal.

\medskip

BIS HIER HAB ICH DEN TEXT VERAENDERT

The key idea of the proof of Theorem~\ref{thrm:dream} is to construct a functorial Thom diagonal.
This construction is strictly not possible for the category $\AVB^\vop$, but one can construct it
functorially for the category $\AVB$. We explain this in Section~\ref{section:ThomDiagonal}.
In Section~\ref{sec:AVB(n)} we introduce categories $\AVB^\vop(n)$.
In this paper they play the technical role, that they filter the category $\AVB^\vop$ over the category $\An^\omega$, that is 
\[
\colim_n \AVB^\vop(n)|_{\An^\omega} \simeq \AVB^\vop|_{\An^\omega},    
\]
but the categories $\AVB^\vop(n)$ also admit a functiorial construction of the Thom diagonal similar to the category $\AVB$.

In Section~\ref{section:spectraGenerated} we study the Thom spectrum $X^{-V}$ of the negative vector bundle $(X,-V)$, which 
is a functor $\AVB^\vop \to \Sp$. Phrased differently, Thereom~\ref{thrm:dream} shows that the 
negative Thom spectrum functor satisfies a universal property. 
Simply put, its the universal functor that preserves certain colimits and splits the 
Thom diagonal.

In Section~\ref{section:Genuinecohomologytheories}






\section{The Category of Anima and Affine Vector Bundles and the Thom Diagonal}\label{sec:AVBs}

\subsection{Affine Linear Homomorphisms}\label{sec:AffLin}

TODO Definition Category $\AffLin$ 


TODO Universal Property of $\AffLin$

\subsection{The Thom Diagonal in the Category $\AVB$}\label{section:ThomDiagonal}
\begin{construction}
    We consider the category 
    \[ 
        \AVB := \int_{X \in \An^\op} \Fun(X, \AffLin);
    \] 
    the unstraightening of the functor that associates to 
    each anima the category of 
    affine linear vector bundles over it. 
    We call the category $\AVB$ \emph{the category of affine linear 
    vector bundles}.
    Its objects consist of data $(X,V)$ where 
    \begin{itemize}
        \item $X$ is an anima 
        \item $V$ is a functor $X \to \AffLin$, which we call \emph{vector bundle over $X$}
    \end{itemize}


\end{construction}

\begin{remark}
    Any functor $X \to \AffLin$ lands in the core of $\AffLin$, which is 
    equivalent to the anima
    \[ 
        \left( \coprod_{n \geq 0} BO(n) \right)_+,
    \] 
    thus the terminology vector bundle is justified as long as we agree
    that the distinguished functor 
    \[
       \emptyset \colon  X \to \pt = (\emptyset)_+ \to \left(\coprod_{n \geq 0} BO(n)\right)_+,  
    \]
    which we view as the \emph{empty bundle} on $X$, is considered to be a vector bundle.
\end{remark}

\begin{construction}\label{construction: mainfunctors}
    By construction $\AVB$ carries a Cartesian fibration
     \[
        \pr \colon \AVB \to \An
     \]
    which forgets the data of the equipped vector bundle. That is 
    $\pr(X,V) = X$ extracts the underlying anima $X$ out of the data $(X,V) \in \AVB$.
    Since $\pr$ is a Cartesian fibration we obtain restriction/pullback functors 
    \[
     \res_X^-(-) \colon \An/_X \times \Fun(X,\AffLin) \to \AVB,   
    \]  
    That sends a pair of a morphism $f \colon Y \to X$ and a vector bundle $V$ on $X$ 
    to the pair $\res^Y_X(V) := (Y, f^*V)$ where $f^*V$ is the vector bundle on $Y$ that 
    arises as the composite 
    \[ 
        f^*Y \colon Y \xrightarrow{f} X \xrightarrow{V} \AffLin.
    \]
    In particular, if we fix $V \in \AffLin$ we have a functor
    \[ 
        \iota_V := \res_\pt^-(V) \colon \An \to \AVB.     
    \]
    If we fix an anima $X$ and call the unique map from $X$ to the point $r$, then 
    $ \iota_V(X) = (X,r^*V)$ and $r^*V$ is the trivial 
    rank $\mathrm{dim}(V)$ bundle on $X$.

    We define the functor $S \colon \AVB \to \An$ to be the corepresented functor 
    \[ 
    S(X,V) := S^V_X := S^V := \Map_{\AVB}(\iota_0(\pt),(X,V)) = \Map((\pt, 0),(X,V)).    
    \]
    We call $S^V_X$ the \emph{associated (unstable) spherical fibration of $V$ over $X$}.

\end{construction}

\begin{lemma}The following statements are true
    \begin{enumerate}
        \item The functor $\pr$ is corepresented by $(\pt, \emptyset)$.
        \item The functor $\iota_\emptyset$ is left and right adjoint to $\pr$.
        \item The functor $\iota_0$ is left adjoint to $S$.
    \end{enumerate}
    \label{lemma:insight}
\end{lemma}

\begin{proof}
    By construction of the category $\AVB$, we can identify the fiber of 
    \[
      \pr_{(X,V),(Y,W)} \colon \Map((X,V), (Y,W)) \to \Map(X,Y)
    \]
    at $f \colon X \to Y$ to be 
    \[
      \Map_{\Fun(X,\AffLin)}(V, f^* W).  
    \]
    If $W$ is the constant $\emptyset$ bundle then so is $f^* W$.
    The $\emptyset$ bundle is the zero object of $\Fun(X,\AffLin)$, hence if either 
    $V$ or $W$ is the $\emptyset$ bundle then $\pr_{(X,V),(Y,W)}$ has contractable fibers, i.e. is an equivalence.
    This proves 1 and 2. 
    The previous calculation also specializes to the equivalence 
    \[
    \Map_{\AVB}(\iota_0 X, (Y,W)) \simeq \colim_{f \colon X \to Y} \Map_{\Fun(X,\AffLin)}(0,f^* W).    
    \]
    The anima $\Map_{\Fun(X,\AffLin)}(0,f^* W)$ is easily seen to be equivalent to the anima of sections of 
    the associated spherical fibration of $W$ over $Y$ pulled back along $f$, i.e. 
    \[
    \Map_{\Fun(X,\AffLin)}(0,f^* W) \simeq \Gamma_X( S^W \times_{Y, f} X).
    \]
    In general, if one has a map $Z \to Y$ of anima on can compute 
    $\Map(X,Z)$ via the colimit
    \[
      \colim_{f \colon X \to Y} \Gamma_X( Z \times_{Y, f} X)  
    \]
    as seen by the following pullback squares 
    \[\begin{tikzcd}
        {\Gamma_X(Z \times_{Y,f}X)} & {\Map(X,Z\times_{Y,f}X)} & {\Map(X,Z)} \\
        {\{\id\}} & {\Map(X,X)} & {\Map(X,Y)}
        \arrow[from=1-1, to=2-1]
        \arrow[from=1-1, to=1-2]
        \arrow[from=1-2, to=2-2]
        \arrow[from=2-1, to=2-2]
        \arrow[from=1-2, to=1-3]
        \arrow[from=1-3, to=2-3]
        \arrow["{f_*}", from=2-2, to=2-3]
        \arrow["\lrcorner"{anchor=center, pos=0.125}, draw=none, from=1-2, to=2-3]
        \arrow["\lrcorner"{anchor=center, pos=0.125}, draw=none, from=1-1, to=2-2]
    \end{tikzcd}\]
    Putting these observations together gives us the equivalence 
    \[
        \Map_{\AVB}(\iota_0 X, (Y,W)) \simeq  \colim_{f \colon X \to Y} \Gamma_X( S^W \times_{Y, f} X)  \simeq \Map(X,S^W_Y).
    \]
    This proves 3.
\end{proof}

\begin{construction}
    Let 
    \[
      \theta \colon \iota_0 S \to \id_{\AVB}  
    \]
    be the counit of the adjuntion $(\iota_0 \dashv S)$.
    We call $\theta_{(X,V)}$ the \emph{Thom diagonal of $V$ on $X$}.
    Let 
    \[ 
        \sigma_\infty \colon \pr \to S    
    \]
    be the unique natural transformation, which is by Yoneda induced from the unique map $\iota_\emptyset(\pt) \to \iota_0(\pt)$, 
    which comes from the unique map $\emptyset \to 0$ in $\AffLin$.
    We call $\sigma_\infty \colon X \to S^V_X$ \emph{the section at $\infty$}.
    The composite of the unit $\id_{\AVB} \to \iota_\emptyset \pr$ and the counit $\iota_\emptyset \pr \to \id_{\AVB}$
    of the adjunctions $\iota_\emptyset \dashv \pr \dashv \iota_\emptyset$ defines an endomorphism
    of the identity on $\AVB$ which we will denote with $\infty$
    and call it the \emph{translation to $\infty$}.\footnote{This endomorphism of the identity is an artefact of the fact that $\AVB$ arises as a fibered category whose fibers are pointed categories}
    This endomorphism induces an action of the category $B(\F_2,\cdot)$ on $\AVB$, that on mapping anima 
    $\F_2 \times \Map((X,V),(Y,W)) \to \Map((X,V),(Y,W))$ sends $(0, \phi)$ to $\infty \circ \phi$.
    The map $\infty \circ \phi$ only depends on the value $\pr(\phi) =: f$, so we define 
    $\infty_{f} := \infty \circ \phi$ and drop $f$ from the notation if $f = \id$. $\infty_f$ is a functorial lift of $f$ along $\pr$ and defines a section 
    of the projection $\Map((X,V),(Y,W)) \to \Map(X,Y)$.
    \label{constr:thomdiag1}
\end{construction}

    The next lemma shows that over the section at infinity the Thom diagonal acts as the translation to infinity.
\begin{proposition}
    We have a unique homotopy
    \[
      \theta \circ \iota_0(\sigma_\infty) \simeq \infty , 
    \]
    or equivalently the following square has a unique filler 
    \[\begin{tikzcd}
        {\iota_\infty \pr} & {\id_{\AVB}} \\
        {\iota_0 \pr} & {\iota_0S}
        \arrow["{\iota_0(\sigma_\infty)}"', from=2-1, to=2-2]
        \arrow["\theta"', from=2-2, to=1-2]
        \arrow["\infty", from=2-1, to=1-1]
        \arrow["\infty", from=1-1, to=1-2]
    \end{tikzcd}\label{diag:AVBsquare}\]\label{prop: thomatinfty}
\end{proposition}

\begin{proof}
    % By the formula for the left adjoint of a morphism into a right adjoint functor, we immediately see that 
    % $\theta \circ \iota_0(\sigma_\infty)$ is the left adjoint morphism to $\sigma_\infty$. In that way both 
    % morphisms determine each other. Therefore we only need to verify that $\sigma_\infty$ is right adjoint to the 
    % morphism $\infty \colon \iota_0 \pr \to \id_{\AVB}$.
    % Let us denote the unit $\id_{\An} \to S \circ \iota_0$ with $j$. We have to compute $S(\infty) \circ j_{\pr} \colon \pr \to S$.
    % Since both functors are corepresented we only have to compute the value of $S(\infty) \circ j_{\pr}$ at the object $(\pt, \emptyset)$.
    By adjunction a natural transformation $\iota_0 \pr \to \id_{\AVB}$ corresponds one to one to a natural transformation between 
    $\pr \to S$. But because $\pr$ is corepresented by $(\pt, \emptyset)$ and $S(\pt, \emptyset) \simeq \Map((\pt, 0), (\pt, \emptyset))\simeq \pt$
    there exists essentially only one natural transformation. 
\end{proof}

\begin{remark}
Evaluated on an object $(X,V) \in \AVB$ the square~\ref{diag:AVBsquare} is of the form 
\[\begin{tikzcd}
	{(X,\emptyset)} & {(X,V)} \\
	{(X,0)} & {(S^V,0)}
	\arrow[from=2-1, to=1-1]
	\arrow[from=1-1, to=1-2]
	\arrow[from=2-1, to=2-2]
	\arrow[from=2-2, to=1-2]
\end{tikzcd}\label{diag:AVBsquareeval}\]
\end{remark}

% \section{Einleitung}
% % Text...

% \section{Grundlagen}
% % Text...

% \section{Hauptergebnisse}
% % Text...

% \section{Beweise}
% % Text...

% \section{Beispiel}
% % Text...

% \section{Schlussfolgerung}
% % Text...

\subsection{Thom Diagonal in the Category $\AVB^\vop$}

\begin{definition}
    The category $\AVB^\vop$ is the vertical dual of $\AVB$, that is 
    \[
      \AVB^\vop := \int_{X \in \An^\op} \Fun(X, \AffLin^\op)  
    \]
\end{definition}

\begin{remark}
    One might expect that a similar square to the square~\ref{diag:AVBsquare} exists 
    for the category $\AVB^\vop$ but the functoriality breaks at the functor $S$.
    Nonetheless there still exists squares of the form 
    \[\begin{tikzcd}
        {(X,\emptyset)} & {(X,0)} \\
        {(X,V)} & {(S^V,V)}
        \arrow[from=2-1, to=1-1]
        \arrow[from=1-1, to=1-2]
        \arrow[from=2-1, to=2-2]
        \arrow[from=2-2, to=1-2]
    \end{tikzcd}\]
    analoguous to the squares~\ref{diag:AVBsquareeval}. The minor difference is that these 
    squares are not functorial in $(X,V) \in \AVB^\vop$.
    To fix this lack of functoriality of the square-\ref{diag:AVBsquare} we will 
    introduce in the next section categories $\AVB(n)$ for each natural number $n$, 
    which filter the category $\AVB^\vop$.
    Over these categories $\AVB(n)$ we can prove a version of Proposition-\ref{prop: thomatinfty}.
\end{remark}

\begin{definition} 
    We call a square $\square \to \AVB^\vop$ \emph{simple distinguished} if it is equivalent to one of the 
    form 
    \[\begin{tikzcd}
        {(X,\emptyset)} & {(X,0)} \\
        {(X,V)} & {(S^V,V)}
        \arrow[from=2-1, to=1-1]
        \arrow[from=1-1, to=1-2]
        \arrow[from=2-1, to=2-2]
        \arrow[from=2-2, to=1-2]
    \end{tikzcd}.\]
    Let $W$ be a vector bundle over $X$. We can add $W$ to the square above and obtain a square 
    \[\begin{tikzcd}
        {(X,\emptyset)} & {(X,W)} \\
        {(X,V \oplus W)} & {(S^V,V \oplus W)}
        \arrow[from=2-1, to=1-1]
        \arrow[from=1-1, to=1-2]
        \arrow[from=2-1, to=2-2]
        \arrow[from=2-2, to=1-2]
    \end{tikzcd}.\]
    We call squares equivalent to one of that form \emph{distinguished}.     
\end{definition}
\begin{remark}
    Let $E \colon \left(\AVB^\vop\right)^\op \to \cC$ be a contravariant functor into a pointed category 
    with fibers. Suppose $E$ sends objects of the form $(X,\emptyset)$ to $0$. 
    Then for each collection of data $(X,V),(X,W) \in \AVB^\vop$ we define the object
    \[
        E^{W - V}(X) := \fib( E^{W}(S^V) \to E^{W}(X)).
    \]
    A distinguished square gives rise to a morphism 
    \[
        E^{W}(X) \to E^{V \oplus W - V}(X)  
    \]
    which is an equivalence if $E$ sends distinguished 
    squares to pullbacks.
    If additionally $E$ suffices an excision / Mayer-Vietoris property in the 
    $X$ variable, then one can further compute 
    \[
    E^{n - V}(X) \simeq \tilde{E}^n(\Th(V))    
    \]   
\end{remark}

\subsection{Thom Diagonal in the Category $\AVB^\vop(n)$}\label{sec:AVB(n)}

To address the issue that we are not able to establish Proposition~\ref{prop: thomatinfty}
in the context of the category $\AVB^\vop$ we will introduce a filtration of categories 
\[
  \AVB^\vop(n) \to \AVB^\vop(n+1) \to \dots \to \AVB^\vop.
\]
We will be able to prove a version of Proposition~\ref{prop: thomatinfty} for the filtration pieces $\AVB^\vop(n)$.
In other words, our goal for this section is to prove the following proposition.

\begin{proposition}\label{prop: AVB(n)}
 There exist Cartesian fibrations $\AVB^\vop(n) \xto{\pr} \An$ together with functors of 
 Cartesian fibrations 
 \[
   \AVB^\vop \xleftarrow{\overleftarrow{\pr}} \AVB^\vop(n) \xrightarrow{\overrightarrow{\pr}} \AVB 
 \]
 such that
 \begin{enumerate}
    \item the endofunctor $- \oplus \R \colon (X,V) \mapsto (X, V \oplus \R_X) \colon \AVB \to \AVB$ extends to a
    functor $- \oplus \overrightarrow{\R} \colon \AVB^\vop(n) \to \AVB^\vop(n + 1)$ so that the following diagram is 
    rendered commutative 
    \[\begin{tikzcd}
        {\AVB^\vop} & {\AVB^\vop(n)} & \AVB \\
        {\AVB^\vop} & {\AVB^\vop(n + 1)} & \AVB
        \arrow["{ -\oplus\overrightarrow{\R}}"', dashed, from=1-2, to=2-2]
        \arrow["{\overrightarrow{\pr}}", from=1-2, to=1-3]
        \arrow["{\overrightarrow{\pr}}"', from=2-2, to=2-3]
        \arrow["{ - \oplus \R}", from=1-3, to=2-3]
        \arrow["{\overleftarrow{\pr}}"', from=1-2, to=1-1]
        \arrow["{\overleftarrow{\pr}}", from=2-2, to=2-1]
        \arrow[Rightarrow, no head, from=1-1, to=2-1]
        \label{}
    \end{tikzcd}\]
    and the induced functor 
    \[
      \colim_{n \to \infty} \AVB^\vop(n) \xto{\overleftarrow{\pr}} \AVB^\vop   
    \]
    is an equivalence over the category of compact anima, that is 
    \[
        \colim_{n \to \infty} \AVB^\vop(n) \times_{\An} \An^\omega \xto{\overleftarrow{\pr}} \AVB^\vop \times_{\An} \An^\omega
    \]
    is an equivalence, 
    \item there exists a diagram $\square \to \mathrm{End}(\AVB^\vop(n))$ of endofunctors  of the form 
    \begin{equation}
    % \begin{figure}
        % \centering
        \begin{tikzcd}
        {\iota_{\overleftarrow{\emptyset},\overrightarrow{\emptyset}}\pr} & \id \\
        {\iota_{\overleftarrow{\R^n},\overrightarrow{0}}\pr} & {\iota_{\overleftarrow{\R^n},\overrightarrow{0}}S^\perp}
        \arrow["\infty", Rightarrow, from=2-1, to=1-1]
        \arrow["\infty", Rightarrow, from=1-1, to=1-2]
        \arrow["{\iota_{\overleftarrow{\R^n},\overrightarrow{0}}\sigma_\infty}"', Rightarrow, from=2-1, to=2-2]
        \arrow["\theta"', Rightarrow, from=2-2, to=1-2]
    \end{tikzcd}
    % \caption{Functorial resolution}
    \label{fig:endosquareavb(n)}
    % \end{figure}
\end{equation}
    that gets sends to distinguished squares under $\overleftarrow{\pr}$.
 \end{enumerate}
\end{proposition}

We begin by constructing the category $\AVB^\vop(n)$.

\subsection{Construction of $\AVB^\vop(n)$}

\begin{construction}
    First we will introduce the intermediate category $\AWB(n)$.
    Let $\AWB(n)$ be the pullback of categories 
    \[\begin{tikzcd}
        {\AWB(n)} & {\AVB/(\pt,\R^n)} \\
        {\AVB \times_{\An} \AVB} & \AVB
        \arrow["\oplus"', from=2-1, to=2-2]
        \arrow["{\mathrm{source}}", from=1-2, to=2-2]
        \arrow["q"', from=1-1, to=2-1]
        \arrow[from=1-1, to=1-2]
        \arrow["\lrcorner"{anchor=center, pos=0.125}, draw=none, from=1-1, to=2-2]
    \end{tikzcd}.\]
    One can think of an object of $\AWB(n)$ to consists of the data 
    \begin{itemize}
        \item an anima $X$,
        \item two vector bundles $V,W$ over $X$,
        \item an affine linear morphism $\alpha \colon V\oplus W \to \R^n_X$.
    \end{itemize}
\end{construction}
The morhisms in $\AWB(n)$ consist of covariant morphisms in both vector bundle
coordinates. This is one of the major differences to the category $\AVB^\vop(n)$.
There we want only one vector bundle coordinate to be covariant, while the other coordinate 
has contravariant morphisms. To remidy this, we will model $\AVB^\vop(n)$ as a subcategory of 
a span category. The following lemma will be useful. 
\begin{lemma}
Let $q \colon \cC \to \cD$ be a pullback preserving functor and let 
$(\cD, b.w., f.w.)$ endow $\cD$ with the structure of an adequate triple. 
Set $q^*(b.w.) := \{ f \colon \Delta^1 \to \cC | q(f) \in b.w.\}$
and $q^*(f.w.)$ likewise.
Then $(\cC, q^*(b.w.), q^*(f.w.))$ is an adequate triple and $q$ refines 
to a functor of adequate triples. 
\end{lemma}

\begin{construction}
    We first endow the category $\AVB \times_{\An} \AVB$ with the structure of an 
    adequate triple via 
    \small
    \[
      (\AVB^{\times_{\An} 2}, b.w., f.w.) := (
        \AVB^{\times_{\An} 2}, (\AVB \times_{\An} \An^\simeq ) \times_{\An} \AVB^\simeq, 
        \mathrm{Cart}(\AVB) \times_{\An} \AVB) . 
    \]
    \normalsize
    We can think of the category $\mathrm{Span}(\AVB^{\times_{\An} 2}, b.w., f.w.)$
    as follows 
    \begin{itemize}
        \item An object consists of the data of an anima $X$ and two vector bundles $V,W$ 
        over $X$, 
        \item A morphism $(X,V,W) \to (Y,V',W')$ consists of the data
        \begin{itemize}
            \item A morphism $f \colon X \to Y$, 
            \item A morphism $\phi \colon f^*V' \to V$
            \item A morhism $\psi \colon W \to f^*W'$.
        \end{itemize}
    \end{itemize}
    In fact the category $\mathrm{Span}(\AVB^{\times_{\An} 2}, b.w., f.w.)$ is equivalent to 
    the category $\AVB^{\vop}\times_{\An}\AVB$. 
    We are more interested in the category 
    \[ 
        \cD(n) := \mathrm{Span}(\AWB(n), q^*b.w., q^*f.w.).
    \]
    The objects $(X,V, W, \alpha \colon (X, V \oplus W) \to (\pt, \R^n))$ of $\cD(n)$ have an 
    additional datum of a map from the sum of the vector bundles to the object $(\pt, \R^n)$.
    As mentioned earlier this is equivalent to the datum of a map 
    \[
    V \oplus W \to \R^n_X .  
    \]
    The morphisms $(X, V, W, \alpha) \to (Y, V', W', \beta)$ of $\cD(n)$ contain an additional 
    compatibility datum 
    \[\begin{tikzcd}
        & {f^*V' \oplus W} \\
        {V\oplus W} && {f^*V' \oplus f^*W'} \\
        & {\R^n_X}
        \arrow["{\phi \oplus W}"', from=1-2, to=2-1]
        \arrow["{f^*V' \oplus\psi}", from=1-2, to=2-3]
        \arrow["\alpha"', from=2-1, to=3-2]
        \arrow["{f^*\beta}", from=2-3, to=3-2]
    \end{tikzcd}\]
    Finally let $\AVB^\vop(n) \subset \cD(n)$ be the full subcategory on those objects 
    \[
        (X,V,W,\alpha \colon (X, V \oplus W) \to (\pt, \R^n))  
    \] where either 
    \begin{itemize}
        \item $\alpha$ is a Cartesian morphism or 
        \item $V = W = \emptyset_X$.
    \end{itemize}
    In the first case, $\alpha$ identifies $W$ as an orthogonal complement bundle of $V$ inside of 
    $\R^n_X$.
    By construcion we have a functor 
    \[
        \AVB^\vop(n) \subset \cD(n) \to \AVB^\vop \times_{\An} \AVB  \to \AVB^\vop \times \AVB
    \]
    Let its projections be called 
    \begin{eqnarray*}
        \overleftarrow{\pr} & \colon & \AVB^\vop(n)  \to \AVB^\vop \\
        \overrightarrow{\pr} & \colon & \AVB^\vop(n)  \to  \AVB
    \end{eqnarray*}

    Moreover the functor 
    \[
        \id_{\AVB} \times_{\id_{\An}} (- \oplus \R) \colon \AVB \times_{\An} \AVB \to \AVB \times_{\An} \AVB        
    \]
    lifts to a functor $- \oplus \overrightarrow{\R} \colon \AWB(n) \to \AWB(n + 1)$ under $q$, 
    such that $- \oplus \overrightarrow{\R}$ is compatible with the adequate triple structure. 
    This induces a functor 
    $- \oplus \overrightarrow{\R} \colon \cD(n) \to \cD(n + 1)$ which restricts to a functor 
    \[
        - \oplus \overrightarrow{\R} \colon \AVB^\vop(n) \to \AVB^\vop(n + 1).
    \] 

\end{construction}

    It is evident by construction that the functor $-\oplus \overrightarrow{\R}$
    renders the following diagram commutative 
    \[\begin{tikzcd}
        {\AVB^\vop} & {\AVB^\vop(n)} & \AVB \\
        {\AVB^\vop} & {\AVB^\vop(n + 1)} & \AVB
        \arrow["{ -\oplus\overrightarrow{\R}}"', dashed, from=1-2, to=2-2]
        \arrow["{\overrightarrow{\pr}}", from=1-2, to=1-3]
        \arrow["{\overrightarrow{\pr}}"', from=2-2, to=2-3]
        \arrow["{ - \oplus \R}", from=1-3, to=2-3]
        \arrow["{\overleftarrow{\pr}}"', from=1-2, to=1-1]
        \arrow["{\overleftarrow{\pr}}", from=2-2, to=2-1]
        \arrow[Rightarrow, no head, from=1-1, to=2-1].
    \end{tikzcd}\]

\subsubsection{Proof of Proposition~\ref{prop: AVB(n)} Part 1}

    We now want to show that the induced functor 
    \[
    \overleftarrow{\pr} \colon \AVB^\vop(\infty) := \colim_n \AVB^\vop(n) \to \AVB^\vop   
    \]
    induces an equivalence 
    \[
    \overleftarrow{\pr}_{\An^\omega} \colon \AVB^\vop(\infty) \times_{\An} \An^\omega \to \AVB^\vop \times_{\An} \An^\omega.
    \]
\begin{remark}
    It is not true that $\overleftarrow{\pr}$ is an equivalence. First our argument 
    wont offer more than an equivalences over finite anima as it uses that 
    filtered colimits in $\catinfty$ commute with finite limits and not all limits. Second it is 
    not even true that $\overleftarrow{\pr}$ is essentially surjective as this would imply that 
    all vector bundles admit a finite dimensional orthogonal complement, which is absurd when one considers 
    the universal rank one bundle $V_{\mathrm{uni}} \to BO(1)$.
    One can verify by a calculation in characteristic classes that a potential complement of 
    $V_\mathrm{uni}$ cannot have a bounded rank.
\end{remark}

    We observe that our construction of $\AVB^\vop(n)$ is compatible with its parametrization 
    over $\An$, that means that we can construct $\AVB^\vop(n)$ as the unstraightening of a functor 
    that sends $X$ to the functor category $\Fun(X, \AffLin^\op(n))$. Where $\AffLin^\op(n)$ is some 
    category constructed out of $\AffLin$ in a similar fashion to how we have constructed $\AVB^\vop(n)$
    out of $\AVB$. 

\begin{construction}
    Let $\mathcal W(n)$ be the pullback of categories 
    \[\begin{tikzcd}
        {\mathcal W(n)} & {\AffLin/\R^n} \\
        {\AffLin^{\times 2}} & \AffLin
        \arrow["{\mathrm{source}}", from=1-2, to=2-2]
        \arrow["\oplus"', from=2-1, to=2-2]
        \arrow[from=1-1, to=2-1]
        \arrow[from=1-1, to=1-2]
        \arrow["\lrcorner"{anchor=center, pos=0.125}, draw=none, from=1-1, to=2-2]
    \end{tikzcd}\]
    Let $\tilde{r.e.}, \tilde{l.e.}$ be defined as 
    \begin{eqnarray*} 
        \tilde{r.e.} & := & \left(\AffLin \times \AffLin^\simeq \right) \times_{\AffLin} \AffLin/\R^n \\
        \tilde{l.e.} & := & \left(\AffLin^\simeq \times \AffLin \right) \times_{\AffLin} \AffLin/\R^n.
    \end{eqnarray*}
    The categories of morphisms $\tilde{r.e.}, \tilde{l.e.}$ endow $\mathcal W(n)$ with the structure of 
    an adequate triple. 
    Let $\AffLin^\op(n)$ be the full subcategory of
    \[
        \Span(\mathcal W(n), \tilde{r.e.}, \tilde{l.e.}) 
    \]
    on those objects $(V,W, \alpha \colon  V \oplus W \to \R^n)$ where either 
    \begin{itemize}
        \item $V = W = \emptyset$ or 
        \item $\alpha$ is an equivalence.
    \end{itemize}

    A typical morphism of $\AffLin^\op(n)$ is of the form 
          \[
            (V,W,\alpha\colon  V\oplus W \xto{\simeq} \R^n) \xot{f}
            (S,U,\beta \colon S \oplus U \to \R^n) \xto{g}
            (V',W'\alpha' \colon V'\oplus W' \xto{\simeq} \R^n)  
          \]
          In particular it gives us a commutative diagram 
          \[\begin{tikzcd}
              && {S \oplus U} \\
              {V\oplus W} && {\R^n} && {V'\oplus W'}
              \arrow["{(\overleftarrow{\pr}g,\overrightarrow{\pr}g)}", from=1-3, to=2-5]
              \arrow["{(\overleftarrow{\pr}f\overrightarrow{\pr}f)}"', from=1-3, to=2-1]
              \arrow["{\alpha }"', from=2-1, to=2-3]
              \arrow[from=1-3, to=2-3]
              \arrow["{\alpha'}", from=2-5, to=2-3]
          \end{tikzcd}\]
          where $\overrightarrow{\pr}f$ and $\overleftarrow{\pr}g$ are equivalences or $V=W=\emptyset$.
          Hence, we can think of the data of a morphism in $\AffLin^\op(n)$ to 
          consist of  
          \begin{itemize}
              \item A map $\phi \colon V' \to V$,
              \item A map $\psi \colon W \to W'$,
              \item A commutative square 
              \[\begin{tikzcd}
                  {V'\oplus W} & {V'\oplus W'} \\
                  {V\oplus W} & {\R^n}
                  \arrow["\alpha"', from=2-1, to=2-2]
                  \arrow["{\alpha'}", from=1-2, to=2-2]
                  \arrow["{\phi\oplus W}"', from=1-1, to=2-1]
                  \arrow["{V'\oplus \psi}", from=1-1, to=1-2]
              \end{tikzcd}\]
          \end{itemize} 
\end{construction}

\begin{lemma}
    Let $X \in \An$ and let $(\cC, b.w._{\cC}, f.w._{\cC})$ be an adequate triple. 
    The functor category $\cC^X$ is an adequate triple by setting
    \[
    b.w._{\cC^X} := (b.w._{\cC})^X    
    \]    
    and likewise with $f.w._{\cC^X}$.
    Then the natural map into the limit 
    \[
        \Span(\cC^X, b.w._{\cC^X}, f.w._{\cC^X}) \to \Span(\cC,b.w._{\cC}, f.w._{\cC})^X
    \]
    is an equivalence.
\end{lemma}

\begin{proof}
    To show that this functor is an equivalence we show that the corresponding map 
    \[
        \left( \Span(\cC^X, b.w._{\cC^X}, f.w._{\cC^X})^{\Delta^\bullet}\right)^\simeq \to \left(\Span(\cC,b.w._{\cC}, f.w._{\cC})^{X \times \Delta^\bullet}\right)^\simeq
    \]
    of Segal anima is an equivalence.
    For $\bullet = 0$ this follows from the fact that the core of a category $\cD$ is equivalent to the core of its span category 
    \[
        \cD^\simeq \subset \left(\Span(\cD, b.w._{\cD}, f.w._{\cD})\right)^\simeq \subset \left(\Span(\cD)\right)^\simeq \simeq \cD^\simeq 
    \]
    and the fact that the core is right adjoint to the inclusion of anima into categories
    \[
      \left(\cD^X\right)^\simeq \simeq (\cD^\simeq)^X.
    \]
    For $\bullet \geq 1$ we can reduce to $\bullet = 1$ by the Segal property of the span category construction as a complete Segal anima.
    In that case we find
    \begin{eqnarray*}
        \left(\Span(\cC^X, b.w._{\cC^X}, f.w._{\cC^X} )^{\Delta^1}\right)^\simeq & \simeq & \left(b.w._{\cC^X} \times_{\cC^X}f.w._{\cC^X}\right)^\simeq \\
        % & \simeq & (\left(b.w. \times_{\cC} f.w. \right)^X)^\simeq \\
        & \simeq & \left((b.w. \times_{\cC} f.w.)^X\right)^\simeq \\
        & \simeq & \left(\Span(\cC,f.w._{\cC}, b.w._{\cC})^{X \times \Delta^1}\right)^\simeq
    \end{eqnarray*}

\end{proof}

\begin{proposition}\label{prop: asunst}
    % The exists a diagram where the vertical arrows are equivalences 
    % \tiny
    % \[\begin{tikzcd}[column sep=tiny]
    %     {\AVB^\vop(n)} & {\Span(\left( \int_{X \in \An^\op} \AffLin^X\right)^{\times 2}\times_{\left( \int_{X \in \An^\op} \AffLin^X\right)}\left( \int_{X \in \An^\op} \AffLin^X/\mathrm{const}_{\R^n}\right), r.e., l.e.)} \\
    %     {\int_{X \in \An^\op} \AffLin^\op(n)^X} & {\int_{X \in \An^\op} \Span(\mathcal W(n), \tilde{r.e.}, \tilde{l.e.})^X}
    %     \arrow[hook, from=1-1, to=1-2]
    %     \arrow[Rightarrow, no head, from=1-2, to=2-2]
    %     \arrow[Rightarrow, no head, from=1-1, to=2-1]
    %     \arrow[hook, from=2-1, to=2-2]
    % \end{tikzcd}\]
    % \normalsize

There exists an equivalence of categories 
\[
    \cD(n) = \Span(\AWB(n), r.e., l.e.) \simeq \int_{X \in \An^\op}\Span(\mathcal W(n), \tilde{r.e.}, \tilde{l.e.})^X.
\]
Furthermore over the category of compact anima
the full subcategories 
\[
    \AVB^\vop(n) \times_{\An}\An^\omega \subset \cD(n) \times_{\An}\An^\omega
\]
and 
\[
    \int_{X \in (\An^\omega)^\op} \AffLin^\op(n)^X \subset \int_{X \in (\An^\omega)^\op}\Span(\mathcal W(n), \tilde{r.e.}, \tilde{l.e.})^X
\]
coincide under this identification.


\end{proposition}
\begin{proof}
    By using the equivalence 
    \[
    \AVB/(\pt, \R^n) \simeq \int_{X \in \An^\op} (\AffLin/\R^n)^X   
    \]
    we find that $\AWB(n)$ is a pullback of Cartesian fibrations over $\An$ along functors that 
    are functors of Cartesian fibrations.
    As such it is a Cartesian fibration over $\An$ too, namely
    \[
    \AWB(n) \simeq \int_{X \in \An^\op} (\AffLin^X)^{\times 2} \times_{\AffLin^X} \times (\AffLin/\R^n)^X \simeq \int_{X \in \An^\op} \mathcal W(n)^X
    \]
    By CITE FABIAN!!!, we have 
    \[
    \cD(n) = \Span(\AWB(n), r.e., l.e.) \simeq \int_{X \in \An^\op} \Span(\mathcal W(n)^X, \tilde{r.e.}^X, \tilde{l.e.}^X)    
    \]
    The last lemma shows 
    \[
        \int_{X \in \An^\op} \Span(\mathcal W(n)^X, \tilde{r.e.}^X, \tilde{l.e.}^X) \simeq \int_{X \in \An^\op} \Span(\mathcal W(n), \tilde{r.e.}, \tilde{l.e.})^X  
    \]
    \dots


\end{proof}    
    

    We conclude that in particular $\overleftarrow{\pr}_{\An^\omega}$ is a functor of Cartesian fibrations over $\An^\omega$
    and the straightend functors $(\An^\omega)^\op \to \catinfty$ of its source and target 
    are both finite limit preserving 
    and thus it is enough to show that $\overleftarrow{\pr}$ is an equivalence over the point, i.e.
    \[ 
        \AffLin^\op(\infty) := \colim_n \AffLin^\op(n) \to \AffLin^\op
    \]
    is an equivalence. 

  \begin{lemma}\label{colimitlemma}
    The functor $-\oplus \overrightarrow{\R} \colon {\cO}^{\times 2} \to {\cO}^{\times 2}
      \colon (V,W) \mapsto (V, W\oplus \R)$ induces a functor
    $-\oplus \overrightarrow{\R} \colon \cO^\op(n)\to \cO^\op(n+1)$,
    compatible with the functors
    $\cO^\op(n) \to \Span({\mathcal W(n)}, r.e., l.e. ) \xto{(pr_1)_*} \Span(\cO,
      {\cO}, {\cO}^\simeq) \simeq ({\cO})^\op$
    so that we have an equivalence
    \[
      \AffLin^\op(\infty) := \colim (\cO^\op(0) \xto{-\oplus \overrightarrow{\R}} \cO^\op(1)
      \xto{-\oplus \overrightarrow{\R}} \dots) \xrightarrow{\simeq} {\cO}^\op 
      .\]
  \end{lemma}
  \begin{proof}
    The existence of the induced functor
    $\cO(n)^\op \to \cO(n+1)^\op$ and the compatibility is clear.
    We show that the functor
    \[
     \AffLin^\op(\infty) \to ({\cO})^\op
    \]
    is fully faithful and essentially surjective.
    The essential surjectivity is clear as every object of $\AffLin$ is 
    equivalent to $\R^k$ or $\emptyset$ for some $k$.
    And these are evidently in the image. 
    We now show that the functor is fully faithful.
    Fix objects $(V,W,\alpha \colon V \oplus W \xto{\simeq} \R^n),
      (V',W',\alpha' \colon V' \oplus W' \xto{\simeq} \R^n) \in \cO^\op(n)$ and let
    \[
      M := \Map((V,W,\alpha \colon V \oplus W \xto{\simeq} \R^{n}),
      (V',W',\alpha' \colon V \oplus W' \xto{\simeq} \R^{n}))
      .\]
    In the case $\emptyset \in \{V,W,V',W'\}$ we easily compute
    $ M \simeq \pt \simeq \Map_{(\cO)^\op}(V',\emptyset) \simeq
      \Map_{(\cO)^\op}(\emptyset, V)$ and the claim follows from the fact
    $(\emptyset,\emptyset) \oplus \overrightarrow{\R} \simeq
      (\emptyset,\emptyset) \in \cO(n+1)$ as $\emptyset \in \cO$
    is an absorbing element with respect to $\oplus$.
  
    So suppose $V,W,V',W' \neq \emptyset$. By unraveling the construction of
    $\cO^\op(n)$ we have that $M$ is equivalent to the core of the
    full subcategory of
    \[
      \Fun(\Lambda^2_0, {\mathcal W(n)}) \times_{(\ev_1,\ev_2),
        {\mathcal W(n)}\times {\mathcal W(n)}} \{(V,W,\alpha),(V',W',\alpha')\}
    \]
    on those wedges
    \[
      (V,W,\alpha) \xleftarrow{f} (A,B,\beta) \xto{g} (V',W',\beta)
    \]
    for which $f$ is a right equivalence and $g$ is a left equivalence.
    By writing $ \Fun(\Lambda^2_0, {\mathcal W(n)}) $ as the
    pullback ${\mathcal W(n)}^{\Delta^1} \times_{s,{\mathcal W(n)},s} {\mathcal W(n)}^{\Delta^1}$
    and commuting limits with limits we find that $M$ is equivalent to the core of
    \[
      r.e./(V,W,\alpha) \times_{{\mathcal W(n)}} l.e./(V',W',\alpha').
    \]
    The category of right equivalences, for example,
    is itself a pullback of categories.
    % , see \eqref{eq:defnre}.
    Therefore, we can compute $r.e./(V,W,\alpha)$, after
    commuting limits again, as the pullback of
    \[
      \cO/V \times (\cO)^\simeq/W \xto{(\oplus, \R^n)} \cO/V\oplus W \times
      \cO/\R^n \xot{(s,t)} (\cO)^{\Delta^1}/\alpha
    \]
    The category $(\cO)^\simeq/W$ is contractible, so we can omit it in the above wedge.
    Moreover, the above wedge is of the form
    $A \times_{\pt} \pt \to B \times_{\pt} C \leftarrow D \times_D D$,
    so we can compute its pullback $P$ as the
    pullback of  $A \times_B D \to D \leftarrow \pt \times_C D$.
    Let us first compute
    \begin{eqnarray*}
      \pt \times_C D &=& \{\id_{\R^n}\} \times_{\cO/\R^n,t} (\cO)^{\Delta^1}/\alpha \\
      & \simeq & \{\id_{\R^n}\} \times_{(\cO)^{\Delta^1}, \ev_{\{1\} \subset \{1,2\} }}
      \Fun(\Pow(\langle 2 \rangle), \cO)
      \times_{(\cO)^{\Delta^1}, \ev_{\{2\} \subset \{1,2\} }} \{\alpha \} \\
      & \simeq & \cO / V \oplus W  = B .
    \end{eqnarray*}
    One can check that the composition $B = \pt \times_C D \to D \to B$ is the identity,
    and that makes the map from $P$ to $A = \cO/V$ an equivalence.
    Similarly one can compute that $l.e./(V',W',\alpha')$ is equivalent to
    $\cO/W'$.
    If we put these results together we can identify $M$ with the core of the pullback
    \[
      \cO/V \to {\mathcal W(n)} \leftarrow \cO/W'
    \]
    where the left map sends a map $f\colon U \to V$ to the triple
    $(U,W, U\oplus W \xto{f \oplus W} V \oplus W \xto{\alpha} \R^n)$ and
    the right map sends $g \colon U \to W'$ to the triple
    $(V',U,V' \oplus U \xto{V' \oplus g} V' \oplus W' \xto{\alpha'} \R^n)$.
  
    There is an evident map from
    $\Map(V',V) \times_{\Map(V'\oplus W,\R^n)} \Map(W,W')$
    into $\cO/V \times_{{\mathcal W(n)}} \cO/W'$. In fact it is obtained
    as the pullback:
    \[\begin{tikzcd}[column sep=tiny]
        {\Map(V',V)\times_{\Map(V'\oplus W,\R^n)}\Map(W,W')} && {\cO/V \times_{{\mathcal W(n)}}\cO/W} \\
        \\
        \\
        {\pt \times_{\pt} \pt} && {\cO \times_{\id\times\{W\},\cO \times \cO,\{V'\}\times\id}\cO}
        \arrow[from=1-1, to=4-1]
        \arrow["{(V',W)}", from=4-1, to=4-3]
        \arrow["{s\times_ss}", from=1-3, to=4-3]
        \arrow[from=1-1, to=1-3]
        \arrow["\lrcorner"{anchor=center, pos=0.125}, draw=none, from=1-1, to=4-3]
      \end{tikzcd}\]
    After taking cores the lower map becomes an equivalence, which shows that
    $M$ is equivalent to $\Map(V',V) \times_{\Map(V'\oplus W,\R^n)} \Map(W,W')$.
    Under this identification the forgetful functor $\cO(n) \to (\cO)^\op$
    induces the projection $\Map(V',V) \times_{\Map(V'\oplus W,\R^n)} \Map(W,W') \to \Map(V',V)$
    and the functor $-\oplus \overrightarrow{\R} \colon \cO(n) \to \cO(n+1)$
    can be identified with the map
    \[
      \Map(V',V) \times_{\Map(V'\oplus W,\R^n)} \Map(W,W') \to
      \Map(V',V) \times_{\Map(V'\oplus W \oplus \R,\R^{n+1})} \Map(W\oplus \R,W'\oplus \R)
    \]
    Now, it becomes clear, that the induced functor
    $\colim \cO(n) \to (\cO)^\op$ is fully faithful,
    whence we have proven that the map
    \begin{eqnarray*}
      \Map(W\oplus \R^k, W' \oplus \R^k) &\to& \Map(V' \oplus W \oplus \R^k, \R^{n+k}) \\
      \phi &\mapsto & \alpha' \circ (\id_{V'} \oplus \phi)
    \end{eqnarray*}
    becomes an equivalence in the colimit for $k \to \infty$.
    Since $\alpha'$ is an equivalence, we may as well show the claim for the system
    \begin{eqnarray*}
      \Map(W\oplus \R^k, W' \oplus \R^k) &\to&
      \Map(V' \oplus W \oplus \R^k, V' \oplus W' \oplus \R^k) \\
      \phi &\mapsto & \id_{V'} \oplus \phi
    \end{eqnarray*}
  
    Both anima of affine linear embeddings are Thom anima
    $M_i$ for vector bundles $\xi_i \to A_i$,  $i = 1,2$, 
    \begin{itemize}
      \item $A_1 = \mathrm{IsoEmb}({W \oplus \R^k},{W' \oplus \R^k})$
      \item $A_2 = \mathrm{IsoEmb}({V' \oplus W \oplus \R^k},{V' \oplus W' \oplus \R^k})$
    \end{itemize}
    of rank
    $\dim W' - \dim W$.
    Moreover the map between $M_1 \to M_2$ comes from a map of vector bundles
    $\xi_1 \to \xi_2$, which induces isomorphisms on fibers.
    Hence we can reduce to check that $A_1 \to A_2$ becomes an equivalence for 
    $k \to \infty$.
    After picking bases $W \cong \R^l, W' \cong \R^{l'}, V' \cong \R^{m'}$,
    we obtain equivalences $A_1 \simeq O(l' + k)/O(l' - l)$ and
    $A_2 \simeq O(m' + l' + k)/O(l' - l)$.
    Under these choices, 
    we can identify the map $A_1 \to A_2$ with the standard inclusion
    $O(l' + k)/O(l' - l) \subset O(m' + l' + k)/O(l' - l)$,
    which yields the desired equivalence on colimits for $k \to \infty$.
  \end{proof}

\subsubsection{Proof of Proposition~\ref{prop: AVB(n)} Part 2}

Similar to Construction~\ref{construction: mainfunctors} we will define analogues 
of the functors $\iota, S$ in the context of $\AVB^\vop(n)$.

\begin{construction}
    By Proposition~\ref{prop: asunst} $\AVB^\vop(n)$ is Cartesian over $\An$ 
    with fiber over the point given by $\AffLin^\op(n)$, so we have a pullback/restriction functor
    \[
        \res \colon \An \times \AffLin^\op(n) \to \AVB^\vop(n) \colon (X, Z) \mapsto \res^X_\pt(Z) = (X, r^*Z)
    \]
    where $r \colon X \to \pt$ is the unique map.
    For any $Z \in \AffLin^\op(n)$ let 
    \[
        \iota_Z(-) = \res^{-}_{\pt}(Z) \colon \An \to \AVB^\vop(n)    
    \]
\end{construction}

Consider the following objects of $\AffLin^\op(n)$.
\begin{itemize}
    \item $Z_1 = (\emptyset, \emptyset, \emptyset \oplus \emptyset = \emptyset \xto{\infty} \R^n)$
    \item $Z_2 = (\R^n, 0, \R^n \oplus 0 = \R^n)$.
\end{itemize}
One can check that $Z_1$ is the zero object of $\AffLin^\op(n)$ and that $Z_2$ corepresents the functor 
that sends $(V,W,\alpha)$ to $S^W$.
We set 
\begin{itemize}
    \item $\iota_{\overleftarrow{\emptyset},\overrightarrow{\emptyset}} := \iota_{Z_1} \colon \An \to \AVB^\vop(n)$
    \item $\iota_{\overleftarrow{\R^n},\overrightarrow{0}} := \iota_{Z_2} \colon \An \to \AVB^\vop(n)$.
\end{itemize}
The unique morphism $Z_2 \to Z_1$ induces a natural transformation \[
    \infty \colon \iota_{\overleftarrow{\R^n},\overrightarrow{0}} \Rightarrow 
\iota_{\overleftarrow{\emptyset},\overrightarrow{\emptyset}}.
\]
Now consider the projection 
\[\begin{tikzcd}
	{\AVB^\vop(n)} && \AVB \\
	& \An
	\arrow["{\overrightarrow{\pr}}", from=1-1, to=1-3]
	\arrow["\pr"', from=1-1, to=2-2]
	\arrow["\pr", from=1-3, to=2-2]
\end{tikzcd}\]
and set 
\[
S^\perp = S \circ \overrightarrow{\pr} \colon \AVB^\vop(n) \to \AVB \to \An.  
\]
The natural transformation $\sigma_\infty \colon \pr \Rightarrow S$ induces a 
natural transformation $\pr \Rightarrow S^\perp$ which we also call $\sigma_\infty$.

\begin{lemma}The following statements are true
    \begin{enumerate}
        \item The functor $\pr \colon \AVB^\vop(n) \to \An$ is corepresented by $\iotaemptyempty(\pt)$.
        \item The functor $\iotaemptyempty$ is left and right adjoint to $\pr$.
        \item The functor $S^\perp$ is corepresented by $\iotarnzero(\pt)$.
        \item The functor $\iotarnzero$ is left adjoint to $S^\perp$.
    \end{enumerate}\label{lemma:favouritelemma}
\end{lemma}

\begin{proof}
    We strictly follow the proof of Lemma~\ref{lemma:insight}.
    The fiber of the map 
    \[
    \Map_{\AVB^\vop(n)}((X,\overleftarrow{V},\overrightarrow{W}, \alpha), (Y,\overleftarrow{V'},\overrightarrow{W'},\beta)) \to \Map(X,Y)   
    \]
    at $f \colon X \to Y$ can be identified with 
    \[
    M := \Map_{\AffLin^\op(n)^X}((\overleftarrow{V},\overrightarrow{W}, \alpha), (\overleftarrow{f^*V'},\overrightarrow{f^*W'},f^*\beta)).
    \]
    The proof of Lemma~\ref{colimitlemma} shows that $M$ can be computed as the pullback 
    \[
      \Map_{\AffLin^X}(f^*V', V) \times_{\Map_{\AffLin^X}(f^*V' \oplus W, \R^n_X)} \Map_{\AffLin^X}(W,f^*W').
    \] 
    For $V = W = \emptyset$ this is contractible, which shows 1. and 2.
    For $V = \R^n_X, W = 0, \alpha = \mathrm{triv}$, we immediately see that the pullback is equivalent to 
    \[
      \Map_{\AffLin^X}(0,f^*W').  
    \]
    From here we can follow the rest of the proof of Lemma~\ref{lemma:insight}.
\end{proof}

\begin{construction}
    Let
    \[
        \theta \colon \iotarnzero S^\perp \Rightarrow \id_{\AVB^\vop(n)}    
    \]
    be the counit of the adjunction 
    $\iotarnzero \dashv S^\perp$. 
    Let 
    \[
      \infty \colon \iotaemptyempty \pr \Rightarrow \id_{\AVB^\vop(n)}  
    \]
    be the counit of the adjunction 
    $\iotaemptyempty \dashv \pr$.
\end{construction}

\begin{proposition}
    We have a unique (up to an anima of contractable choices) homotopy
    \[
      \theta \circ \iotarnzero(\sigma_\infty) \simeq \infty , 
    \]
    or equivalently the following square has a unique (up to an anima of contractable choices) filler 
    \[\begin{tikzcd}
        {\iotaemptyempty \pr} & {\id_{\AVB^\vop(n)}} \\
        {\iotarnzero \pr} & {\iotarnzero S^\perp}
        \arrow["{\iotarnzero(\sigma_\infty)}"', Rightarrow, from=2-1, to=2-2]
        \arrow["\theta"', Rightarrow, from=2-2, to=1-2]
        \arrow["\infty", Rightarrow, from=2-1, to=1-1]
        \arrow["\infty", Rightarrow, from=1-1, to=1-2]
    \end{tikzcd}\label{diag:AVB(n)square}\]\label{prop: thomatinftyinavb(n)}
\end{proposition}
\begin{proof}
    With the help of Lemma~\ref{lemma:favouritelemma} we compute 
    \begin{eqnarray*}
        & & \Nat_{\Fun(\AVB^\vop(n), \AVB^\vop(n))}(\iotarnzero \pr, \id_{\AVB^\vop(n)}) \\ 
        & \simeq & \Nat_{\Fun(\AVB^\vop(n), \An)}(\pr, S^\perp)\\
        & \simeq & \Nat_{\Fun(\AVB^\vop(n), \An)}(\Map(\iotaemptyempty(\pt), -), S^\perp) \\
        & \simeq & S^\perp( \iotaemptyempty(\pt)) = \pt 
    \end{eqnarray*}
\end{proof}

\subsubsection{Proof of Proposition~\ref{prop: AVB(n)} Part 3}
We end this section by showing that the functor $\overleftarrow{\pr}$ maps the square~\ref{fig:endosquareavb(n)}
to a diagram of distinguised squares.

\begin{lemma}
    Let $(\overleftarrow{V},\overrightarrow{W},\alpha) \in \AffLin^\op(n)$. Then we have equivalences of functors 
    \begin{enumerate}
        \item \[
            \overleftarrow{\pr} \circ \iota_{(\overleftarrow{V},\overrightarrow{W},\alpha)} \simeq \iota_V \colon \An \to \AVB^\vop
        \]
        \item \[
            \overrightarrow{\pr} \circ \iota_{(\overleftarrow{V},\overrightarrow{W},\alpha)} \simeq \iota_W \colon \An \to \AVB
        \]
        
    \end{enumerate}
\label{lemma:iotasagree}
\end{lemma}
\begin{proof}
    Since $\overleftarrow{\pr},\overrightarrow{\pr}$ are functors of Cartesian fibrations, we only have to provide equivalences 
    of the values of the point.
    But by construction the left hand and right hand sides agree on the point. 
\end{proof}

By the last lemma we compute that the postcomposition of square~\ref{fig:endosquareavb(n)} with $\overleftarrow{\pr}$ is equivalent 
to the square 
\[\begin{tikzcd}
	{\iota_{\emptyset}\pr} & {\overleftarrow{\pr}} \\
	{\iota_{\R^n}\pr} & {\iota_{\R^n}S^\perp}
	\arrow["\infty", Rightarrow, from=2-1, to=1-1]
	\arrow["\infty", Rightarrow, from=1-1, to=1-2]
	\arrow["{\iota_{\R^n}\sigma^\infty}"', Rightarrow, from=2-1, to=2-2]
	\arrow["{\overleftarrow{\pr}(\theta)}"', Rightarrow, from=2-2, to=1-2]
\end{tikzcd}\]

Consider the counit $\epsilon$ of the adjunction $\iota_0 S \dashv \id_{\AVB}$ of Construction~\ref{constr:thomdiag1}.
On an object $(X,W)$ the morphism $\epsilon_{(X,W)}$ consists of the data of a projection of anima 
\[
\pi \colon S^W \to X    
\]
and an affine linear morphism of vector bundles over $S^W$:
\[
\vartheta \colon 0 \to \pi^*W.    
\]

We want to show that the value of the above square on an object $(X,V,W,\alpha \colon V \oplus W \to \R^n_X)$ is distinguished.
The missing piece is to identify the value of $\overleftarrow{\pr}(\theta)$.
It consists of the information of a morphism of anima 
\[
\pi \colon S^W \to X    
\]
and an affine linear morphism of vector bundles over $S^W$:
\[
\phi \colon \pi^*V \to \R^n_{S^W}.    
\]
In fact $\theta$ itself consists of an additional data of 
an affine linear morphism of vector bundles over $S^W$:
\[
\psi \colon 0 \to \pi^*W
\]
and a compatibility datum 
\[\begin{tikzcd}
	{\pi^*V \oplus 0} & {\pi^*V \oplus \pi^*W} \\
	{\R^n_{S^W} \oplus 0} & {\R^n_{S^W}}
	\arrow["{\pi^*V \oplus \psi}", from=1-1, to=1-2]
	\arrow["{\pi^*\alpha}", from=1-2, to=2-2]
	\arrow["{\mathrm{triv}}"', from=2-1, to=2-2]
	\arrow["{\phi\oplus 0}"', from=1-1, to=2-1]
\end{tikzcd}.\]
To show that the square in question is distinguished we
can assume that $\phi$ is given by $\pi^*V \oplus \psi$.
By \`subtracting' $V$ we only need to show that the squares 
\[\begin{tikzcd}
	{(X,\emptyset)} & {(X,0)} \\
	{(X,W)} & {(S^W,W)}
	\arrow["\infty", from=1-1, to=1-2]
	\arrow["{(\sigma_\infty,\id)}"', from=2-1, to=2-2]
	\arrow["\infty", from=2-1, to=1-1]
	\arrow["{(\pi,\psi)}"', from=2-2, to=1-2]
\end{tikzcd}\]
and 
\[\begin{tikzcd}
	{(X,\emptyset)} & {(X,0)} \\
	{(X,W)} & {(S^W,W)}
	\arrow["\infty", from=1-1, to=1-2]
	\arrow["{(\sigma_\infty,\id)}"', from=2-1, to=2-2]
	\arrow["\infty", from=2-1, to=1-1]
	\arrow["{(\pi,\vartheta)}"', from=2-2, to=1-2]
\end{tikzcd}\]
are equivalent. In other words we need to show
the above square is simple distinguished. 
The following lemma answers this affirmatively.
With Lemma~\ref{lemma:iotasagree} and the definition $S^\perp = S\circ \overrightarrow{\pr}$ in mind we have 
\begin{lemma}
    The natural transformations 
    \[
    \overrightarrow{\pr}(\theta)   \colon \iota_n S^\perp \Rightarrow \overrightarrow{\pr} 
    \]
    and 
    \[
    \epsilon_{\overrightarrow{\pr}} \colon \iota_n S^\perp \Rightarrow \overrightarrow{\pr} 
    \]
    are equivalent.
\end{lemma}

\begin{proof}
    By Lemma~\ref{lemma:iotasagree} we see that $\overrightarrow{\pr}(\theta)$ is a natural transformation 
    \[
      \overrightarrow{\pr} \circ \iotarnzero \circ S^\perp \Rightarrow \overrightarrow{\pr}.  
    \]
    Under the adjunction $\iota_n = \overrightarrow{\pr} \circ \iotarnzero \dashv S$ its adjoint transformation 
    \[
    (\overrightarrow{\pr}(\theta))^\flat \colon S^\perp \Rightarrow S \circ \overrightarrow{\pr} = S^\perp   
    \]
    is computed as the composite
    \[\begin{tikzcd}
        {S^\perp} & {S\circ\overrightarrow{\pr} \circ \iotarnzero \circ S^\perp} & {S \circ \overrightarrow{\pr}} & {S^\perp}
        \arrow["{\eta_{S^\perp}}", Rightarrow, from=1-1, to=1-2]
        \arrow["{S(\overrightarrow{\pr}(\theta))}", Rightarrow, from=1-2, to=1-3]
        \arrow[Rightarrow, no head, from=1-3, to=1-4],
    \end{tikzcd}\]
    where 
    $ \eta_{S^\perp}$ is the value of the unit    
    \[
        \eta \colon \id_{\An} \Rightarrow S\iota_0
    \]
    of the adjunction $\iota_n = \overrightarrow{\pr} \circ \iotarnzero \dashv S$
    on the values of the functor $S^\perp$.
    We first compute $\eta$.
    We claim that $\eta_X$ is given by the $0$-section 
    \[
    X \to S^0_X = S^0 \times X = \{0, \infty\} \times X. 
    \]
    By Yoneda it is enough to check this for $X = \pt$.
    Since $S$ is corepresented by $\iota_0(\pt)$ the value $\eta_\pt$ is determined by the functor $\iota_0$:
    \[\begin{tikzcd}
        \pt & {\Map(\pt,\pt)} \\
        {S(\iota_0(\pt))} & {\Map((\pt,0),(\pt,0))}
        \arrow["{\iota_0}", from=1-2, to=2-2]
        \arrow["{\eta_\pt}", from=1-1, to=2-1]
        \arrow["\simeq", from=1-1, to=1-2]
        \arrow["\simeq"', from=2-1, to=2-2]
    \end{tikzcd}\]
    Under the lower horizontal identification the morphism $\id_{(\pt,0)}$ corresponds to $0 \in S^0_\pt \simeq \{0,\infty\}$.

    We claim that $(\overrightarrow{\pr}(\theta))^\flat$ is the identity transformation $S^\perp \Rightarrow S^\perp$.
    Since $S^\perp$ is corepresented by $\iotarnzero(\pt)$ we only have to verify this on $\iotarnzero(\pt)$, that is 
    we want to understand the map
    \[
    S(S^0, 0) \xrightarrow{S(\overrightarrow{\pr}(\theta_{\iotarnzero(\pt)}))} S(\pt, 0)
    \]
    over the zero section 
    \[
    S^0 \to S(S^0, 0).
    \]
    To pick a point $a$ in the zero section of $S(S^0, 0)$ we need to pick a point $a \in S^0$
    and consider the induced morphism 
    \[
        (\pt,0) \xto{\iota_0(a)} (S^0,0)
    \]
    under $\iota_0$.
    To understand the composite 
    \[
        (\pt,0) \xto{\iota_0(a)} (S^0,0) \xto{\overrightarrow{\pr}(\theta)} (\pt, 0)
    \]
    we observe that we can lift $\iota_0(a)$ along $\overrightarrow{\pr}$, namely by 
    \[
      \iotarnzero(a) \colon \iotarnzero(\pt) \to \iotarnzero(S^0).  
    \]
    The composite $\overrightarrow{\pr}(\theta) \circ \iota_0(a)$ is then equivalent to the image 
    of $\theta \circ \iotarnzero(a)$ under $\overrightarrow{\pr}$.
    But 
    \[
        \theta \circ \iotarnzero(a) \colon \iotarnzero(\pt) \to \iotarnzero(\pt) 
    \]
    is the left adjoint morphism $a^\sharp$ of 
    \[
      a \colon \pt \to S^\perp(\iotarnzero(\pt)) = S^0.  
    \]
    Since $\overrightarrow{\pr}$ is an equivalence between the mapping anima $\Map(\iotarnzero(\pt)), \iotarnzero(\pt)$ 
    and $\Map(\iota_0(\pt), \iota_0(\pt))$, we can recover $a$ out of $\overrightarrow{\pr}(a^\sharp)$.
    That is we have shown that the natural transformation $(\overrightarrow{\pr}(\theta))^\flat$ is invertible. 
    But there is only one invertible natural transformation $S^\perp \Rightarrow S^\perp$ namely the identity.
    The left adjoint natural transformation $\phi^\sharp \colon \iota_n S^\perp \Rightarrow \overrightarrow{\pr}$ of
    a transformation $\phi \colon S^\perp \Rightarrow S \circ \overrightarrow{\pr}$ is computed by the composite 
    \[\begin{tikzcd}
        {\iota_n\circ S^\perp} & {\iota_n \circ S \circ \overrightarrow{\pr}} & {\overrightarrow{\pr}}
        \arrow["{\iota_n \phi}", Rightarrow, from=1-1, to=1-2]
        \arrow["{\epsilon_{\overrightarrow{\pr}}}", Rightarrow, from=1-2, to=1-3]
    \end{tikzcd}\]
    Thus 
    \begin{eqnarray*}
        \overrightarrow{\pr}(\theta) & \simeq & ((\overrightarrow{\pr}(\theta))^\flat)^\sharp \\
        & \simeq & \id^\sharp \\
        & \simeq & \epsilon_{\overrightarrow{\pr}} \circ \iota_n(\id) \\
        & \simeq & \epsilon_{\overrightarrow{\pr}} \circ \id \\ 
        & \simeq & \epsilon_{\overrightarrow{\pr}}
    \end{eqnarray*}  
\end{proof}


% \section{The Category of Spectra generated by Animae and Vector Bundles}\label{section:spectraGenerated}


\begin{definition}
    % Here definition of pushouts with constant coeffficients

\end{definition}

\begin{lemma}
    The functor $\Th \colon \AVB \to \An_*$ sends split pushouts with constant coefficients to pushouts.    
\end{lemma}

\begin{proposition}
    The functor $\mathrm{Th}^- \colon \AVB^\vop \to \Sp$ that sends a pair $(X,V)$ to the 
    Thom spectrum $X^{-V}$ of the negative bundle of $V$ is a functor that sends distinguished squares to pushouts.
\end{proposition}

\begin{proof}
    We have to show that 
    \[\begin{tikzcd}
        0 & {X^{-V}} \\
        {X^{-V\oplus W}} & {(S^{W})^{-V\oplus W}}
        \arrow[to=1-2, from=1-1]
        \arrow[to=1-1, from=2-1]
        \arrow[to=1-2, from=2-2]
        \arrow[to=2-2, from=2-1]
    \end{tikzcd}\]
    is a pushout.
    This square is the image of the square in $T\An$:
    \[\begin{tikzcd}
        {0_X} & {\mathbb S^{-V}_X} \\
        {\mathbb S_X^{-V\oplus W}} & {\pi_! \pi^* \mathbb S_X^{-V\oplus W}}
        \arrow[to=1-2, from=1-1]
        \arrow[to=1-1, from=2-1]
        \arrow[to=1-2, from=2-2]
        \arrow[to=2-2, from=2-1]
    \end{tikzcd}\]
    under the colimit preserving functor $T\An \to \Sp$ that `carries out the colimit'.
    By the projection formula this square is equivalent to the square 
    \[\begin{tikzcd}
        {0_X} & {\mathbb S^{-V}_X} \\
        {\mathbb S_X^{-V} \otimes \mathbb S_X^{-W}} & {\mathbb S_X^{-V} \otimes \mathbb S_X^{-W}\otimes\pi_! \pi^* \mathbb S_X}
        \arrow[to=1-2, from=1-1]
        \arrow[to=1-1, from=2-1]
        \arrow[to=1-2, from=2-2]
        \arrow[to=2-2, from=2-1]
    \end{tikzcd}.\]
    Since the functor $\AVB^\vop \to T\An$ is symmetric monoidal we see that the above square is the tensor product of the image of 
    the simple distinguished square with $\mathbb S_X^{-V}$. So we can assume $V = 0$. So we want to prove that 
\[\begin{tikzcd}
	{0_X} & {\mathbb S_X} \\
	{\mathbb S_X^{-W}} & {\mathbb S_X^{-W}\otimes\pi_! \pi^* \mathbb S_X}
	\arrow[to=1-2, from=1-1]
	\arrow[to=1-1, from=2-1]
	\arrow[to=1-2, from=2-2]
	\arrow[to=2-2, from=2-1]
\end{tikzcd}\]
is a pushout square in $T\An$. This is a pushout if and only if for each point inclusion $x \colon \pt \to X$ the pullback 
\[\begin{tikzcd}
	{x^*0_X} & {x^*\mathbb S_X} \\
	{x^*\mathbb S_X^{-W}} & {x^*(\mathbb S_X^{-W}\otimes\pi_! \pi^* \mathbb S_X)}
	\arrow[to=1-2, from=1-1]
	\arrow[to=1-1, from=2-1]
	\arrow[to=1-2, from=2-2]
	\arrow[to=2-2, from=2-1]
\end{tikzcd}\]
is a pushout. 
So we can assume that $X = \pt$. That is we have to show that 
\[\begin{tikzcd}
	0 & {\mathbb S} \\
	{\mathbb S^{-W}} & {\mathbb S^{-W}\otimes \colim_{S^W} \mathbb S}
	\arrow[to=1-2, from=1-1]
	\arrow[to=1-1, from=2-1]
	\arrow[to=1-2, from=2-2]
	\arrow[to=2-2, from=2-1]
\end{tikzcd}\]
is a pushout in spectra for each sphere $S^W$.
Let us now try to understand the non-trivial maps in that diagram.
The bottom map comes from the point inclusion at $\infty \in S^W$, namely its the map 
\[
\mathbb S^{-W} \otimes \colim_{\pt} \mathbb S \xto{\mathbb S^{-W}\otimes \colim_{\infty} \mathbb S} \mathbb S^{-W} \otimes \colim_{S^W} \mathbb S.
\]
and the right map composes as 
\[
\mathbb S^{-W} \otimes \colim_{S^W} \mathbb S \xto{\mathbb S^{-W}\otimes \colim_{S^W} \Sigma^\infty \theta} \mathbb S^{-W} \otimes \colim_{S^W} \mathbb S^W \simeq \colim_{S^W}\mathbb S \xto{\colim_\pi \mathbb S} \colim_\pt \mathbb S    
\]
By tensoring the above square with the invertible object $\mathbb S^W$ we can instead show that
\[\begin{tikzcd}
	0 & {\mathbb S^W} \\
	{\colim_\pt \mathbb S} & {\colim_{S^W} \mathbb S}
	\arrow[from=1-1, to=1-2]
	\arrow[from=2-1, to=1-1]
	\arrow["{\colim_\infty \mathbb S}"', from=2-1, to=2-2]
	\arrow["{\colim_{\pi}\Sigma^\infty\theta}"', from=2-2, to=1-2]
\end{tikzcd}\]
is a pushout square.
This square is the image under the functor $\Sigma^\infty$ of the square 
\[\begin{tikzcd}
	\pt & {S^W} \\
	{\pt_+} & {S^W_+}
	\arrow["\infty", from=1-1, to=1-2]
	\arrow[from=2-2, to=1-2]
	\arrow[from=2-1, to=1-1]
	\arrow["{\infty_+}"', from=2-1, to=2-2]
\end{tikzcd}\]
where the right map is the composite 
\[
S^W_+ \xto{\theta} S^W_+ \wedge S^W \xto{\pi \wedge S^W} \pt_+ \wedge S^W.    
\]
We claim that $\theta$ is given by the diagonal 
\[
S^W_+ \xto{\Delta} S^W_+ \times S^W_+ \xto{\mathrm{can}} S^W_+ \wedge S^W.
\]
This clearly shows that the right map of the square above is given by the identity on the $S^W$ component, which 
makes the square a pushout.
By tracing back the definition we find that $\theta$ is the canonical map 
\[
  \colim^{\in \An_*}_{\Map((\pt,0),(\pt,W))}  \Map(((\pt,0),(\pt,0))) \xto{\colim_f (g \mapsto f \circ g)} \colim^{\in \An_*}_{\Map((\pt,0),(\pt,W))} \Map(((\pt,0),(\pt,W)))
\]
which is equivalent to the map above.
\end{proof}

\begin{remark} 
    The last proposition shows that we have natural cofiber sequences 
    \[
      X^{-V \oplus W} \xto{ \sigma_\infty^{- V \oplus W}} (S^W)^{-V \oplus W} \xto{\pi^{-\theta}} X^{-V}.
    \]
    These split natural as the left map admits a left inverse via 
    \[
    \pi^{-V\oplus W} \colon (S^W)^{-V \oplus W} \to X^{-V \oplus W}.
    \]
    So we have a natural decomposition 
    \[
        (S^W)^{-V \oplus W} \simeq  X^{-V \oplus W} \oplus X^{-V}.
    \]
    Suppose $V \oplus W \simeq \R^n_X$ is a trivial bundle, than we have a square in $\AVB^\vop$
    \[\begin{tikzcd}
        {(X,\R^n_X)} & {(S^W,\R^n_{S^W})} \\
        {(\pt, \R^n)} & {(\mathrm{Th}(W),\R^n_{\mathrm{Th}(W)})}
        \arrow["{(r,\id)}"', from=1-1, to=2-1]
        \arrow["{(\infty,\id)}"', from=2-1, to=2-2]
        \arrow["{(\sigma_\infty, \id)}", from=1-1, to=1-2]
        \arrow["{(\mathrm{can}, \id)}", from=1-2, to=2-2]
    \end{tikzcd}\]
    which get send to a pushout via $\Th^-$ since it is a pushout in the anima coordinate and constant in the vector bundle coordinate.
    Thus combining with the previous result we have a natural equivalence 
    \[
      X^{-V} \to \Sigma^{\infty - n}\Th(W)  
    \]
    once we fix a choice of a complement bundle $W$ of $V$. 
\end{remark}

We have seen that $\Th^-$ respects pushout squares in the anima coordinate and sends distinguished squares to 
pushouts. The next theorem will show that it is the universal functor that does this at least over compact anima. 

\begin{definition}
    Let $\cD$ be a category. We call a pushout square
    \[\begin{tikzcd}
        A & B \\
        C & D
        \arrow["f", from=1-1, to=1-2]
        \arrow["g"', from=1-1, to=2-1]
        \arrow[from=2-1, to=2-2]
        \arrow[from=1-2, to=2-2]
        \arrow["\lrcorner"{anchor=center, pos=0.125, rotate=180}, draw=none, from=2-2, to=1-1]
    \end{tikzcd}\]
    in $\cD$ \emph{split} if at least one of the maps $f$ or $g$ admits a retraction.
    We say that $\cD$ \emph{has all split pushouts} if for every span $B \xot{f} A \xto{g} C$, where either 
    $f$ or $g$ is a retraction, the pushout $B \coprod_A C$ exists.
\end{definition}

\begin{theorem}\label{thrm:dream}
    
    For any pointed $\infty$-cagegory $\cD$
    with split pushouts the restriction along the functor
    $\Th^- \colon \cX|_{\An^\omega} \to \Sp^\omega, (X,V) \mapsto X^{-V}$
    induces a functor
    $ (\Th^-)^* \colon \Fun(\Sp^\omega, \cD) \to \Fun(\cX|_{\An^\omega} , \cD)$
    which is fully faithful when restricted to the subcategory consisting of those
    functors $\Sp^\omega \to \cD$ which preserve split pushouts and are reduced.
    The essential image of that subcategory consists precisely of those functors
    $\Phi \colon \cX|_{\An^\omega}  \to \cD$ which
    \begin{enumerate}
      \item send distinguished squares to pushouts,
      \item send $(X,\emptyset)$ for all compact anima $X$ to the zero object,
      \item send squares which are constant in the bundle coordinate and split pushouts in the anima variable to pushouts. \footnote{By this we strictly mean a square that is a Cartesian lift.}
    \end{enumerate}
\end{theorem}

\begin{lemma}\label{lemma:reduction}
    The following diagram commutes 
    \[\begin{tikzcd}
        {\AVB^\vop(n)|_{\An^\omega}} && {} && {\An_*^\omega} \\
        & {\AVB^\vop|_{\An^\omega}} & {\Sp^\omega} \\
        {\AVB^\vop(n+1)|_{\An^\omega}} &&&& {\An_*^\omega}
        \arrow["{\Th^\perp}", from=1-1, to=1-5]
        \arrow["{\Th^\perp}"', from=3-1, to=3-5]
        \arrow["{\oplus \overrightarrow{\R}}"', from=1-1, to=3-1]
        \arrow["{\Th^-}", from=2-2, to=2-3]
        \arrow["\Sigma", from=1-5, to=3-5]
        \arrow["{\Sigma^{\infty-n}}", from=1-5, to=2-3]
        \arrow["{\Sigma^{\infty-(n+1)}}", from=3-5, to=2-3]
        \arrow["{\overleftarrow{\pr}}"', from=1-1, to=2-2]
        \arrow["{\overleftarrow{\pr}}", from=3-1, to=2-2]
    \end{tikzcd}\]
\end{lemma}

\begin{proof}
    We show that the outer and the upper square commute. The rest is clear.
    \begin{itemize} 
        \item 
        We start by showing the commutativity of the outer square.
        We observe that we can factor the vertical functors as follows 
        \[
            \AVB^\vop(n)|_{\An^\omega} \xto{\overrightarrow{\pr}} \AVB|_{\An^\omega} \to \int_{X \in \An^\omega} \Fun(X,\An_*^\omega) \xto{\colim} \An_*^\omega 
        \]
    \end{itemize} 
    Where the middle functor is induced from 
    \[
        S = \Map((\pt,0), -) \colon \AffLin \to \An_*^\omega.  
    \]  
    Since $\overrightarrow \pr \circ (\oplus \overrightarrow{\R}) \simeq  (\oplus \R )\circ \overrightarrow{\pr}$ 
    and $\oplus \R$ is also induced from its restriction to $\AffLin$ we only need to show that 
    \[\begin{tikzcd}
        \AffLin & {\An^\omega_*} \\
        \AffLin & {\An^\omega_*}
        \arrow["{\oplus \R}"', from=1-1, to=2-1]
        \arrow["\Sigma", from=1-2, to=2-2]
        \arrow["S", from=1-1, to=1-2]
        \arrow["S"', from=2-1, to=2-2]
    \end{tikzcd}\]
    commutes, which it oviously does.
    \item Now we show that the upper square commutes. 
    A reduction argument like before shows that we only need to show that the following diagram commutes
    \[\begin{tikzcd}
        && {\AffLin^\op(n)} \\
        {\AffLin^\op} &&&& \AffLin \\
        & {{\Sp^\omega}{^\op}} && {\Sp^\omega}
        \arrow["{\overleftarrow{\pr}}"', from=1-3, to=2-1]
        \arrow["{\overrightarrow{\pr}}", from=1-3, to=2-5]
        \arrow["{\Sigma^{\infty - n} S}", from=2-5, to=3-4]
        \arrow["{\mathbb D}"', from=3-2, to=3-4]
        \arrow["{\Sigma^\infty S}"', from=2-1, to=3-2]
    \end{tikzcd}\]
    that is we have to provide an equivalence between the functors 
    \[
    (\overleftarrow V, \overrightarrow{W}, \alpha \colon V \oplus W \to \R^n) \mapsto \mathbb D \Sigma^\infty S^V    
    \]
    and 
    \[
    (\overleftarrow V, \overrightarrow{W}, \alpha \colon V \oplus W \to \R^n) \mapsto \Sigma^{\infty - n} S^W.
    \]
    The morphism $\alpha$ induces a natural map 
    \[
    S^W \to \Map(S^V, S^n)    
    \]
    which induces a natural equivalence 
    \[
    \Sigma^\infty S^W \to \mathrm{map}(\Sigma^\infty S^V, \Sigma^\infty S^n)    
    \]
    which after delooping yields the desired equivalence.
\end{proof}

\begin{proposition}\label{prop:dream2}
    For any pointed $\infty$-cagegory $\cD$
    with split pushouts the restriction along the functor
    $\Th^\perp \colon \cX(n)|_{\An^\omega} \to \An_*^\omega, (X,\overleftarrow{V},\overrightarrow{W}, \alpha) \mapsto \Th_X(W)$
    induces a functor
    $ (\Th^\perp)^* \colon \Fun(\An_*^\omega, \cD) \to \Fun(\cX(n)|_{\An^\omega} , \cD)$
    which is fully faithful when restricted to the subcategory consisting of those
    functors $\An_*^\omega \to \cD$ which preserve split pushouts and are reduced.
    The essential image of that subcategory consists precisely of those functors
    $\Phi \colon \cX(n)|_{\An^\omega}  \to \cD$ which
    \begin{enumerate}
      \item send distinguished squares to pushouts,
      \item send $(X,\emptyset,\emptyset, \infty)$ for all compact anima $X$ to the zero object,
      \item send squares which are constant in the bundle coordinate and split pushouts in the anima variable to pushouts.
    \end{enumerate}
\end{proposition}

\begin{proof}[Proof of Theorem~\ref{thrm:dream}]
    \begin{eqnarray*}   
        \Fun^{\mathrm{d.s,red.,s.p.}}(\AVB^\vop, \mathcal D) & \stackrel{\mathrm{Lemma}~\ref{lemma:reduction}}{\simeq} & \lim_n 
        \Fun^{\mathrm{d.s,red.,s.p.}}(\AVB^\vop(n), \mathcal D) \\
        & \stackrel{\mathrm{Prop.}~\ref{prop:dream2}}{\simeq} & \lim_n \Fun^{\mathrm{s.p.}}(\An_*^\omega, \mathcal D) \\
        & \simeq & \Fun^{\mathrm{s.p.}}(\Sp^\omega, \mathcal D)
    \end{eqnarray*} 
\end{proof}
\begin{proof}[Proof of Proposition~\ref{prop:dream2}]
    Given a functor $\psi \colon \AVB^\vop(n) \to \mathcal D$ we define 
    the functor 
    \[
    \psi|_{\An_*^\omega} \colon \An_*^\omega \to \mathcal D    
    \]
    by letting it send a pointed anima $\pt \to X$ to the value 
    \[
    \psi|_{\An_*^\omega}(X) := \mathrm{cof}\left( \psi(\iotarnzero(\pt))\to \psi(\iotarnzero(X))\right).  
    \]
    We claim that the functor $\res_{\An_*^\omega} \colon \psi \mapsto \psi|_{\An_*^\omega}$ is inverse to the functor $(\Th^\perp)^*$ when restricted to 
    the full subcategories 
    $\Fun^{\mathrm{d.s,red.,s.p.}}(\AVB^\vop(n), \mathcal D)$ and 
    $\Fun^{\mathrm{red.,s.p.}}(\An_*^\omega, \cD)$ of the respective functor categories.

    We start with a reality check by showing the inclusions 
    
        \[ 
            \res_{\An_*^\omega}(\Fun^{\mathrm{red.,s.p.}}(\AVB^\vop(n), \mathcal D)) \subset \Fun^{\mathrm{red.,s.p.}}(\An_*^\omega, \cD),
        \]
         \[
            (\Th^\perp)^*(\Fun^{\mathrm{red.,s.p.}}(\An_*^\omega, \cD)) \subset \Fun^{\mathrm{d.s,red.,s.p.}}(\AVB^\vop(n), \mathcal D):
        \]


    \begin{itemize}
        \item The inclusion 
        \[
            \res|_{\An_*^\omega}(\Fun) \subset \Fun^\mathrm{red.}
        \]
        follows immediately from the definitions.
        \item The inclusion 
        \[
            \res|_{\An_*^\omega}(\Fun^\mathrm{s.p.}) \subset \Fun^{\mathrm{s.p.}}
        \]
        folows from the fact that the functor $\iotarnzero$ sends split pushouts to split pushouts that are constant in the vector bundle 
        coordinates.
        \item The inclusion 
        \[
            (\Th^\perp)^*(\Fun^\mathrm{red.}) \subset \Fun^\mathrm{red.}
        \]
        follows from the fact that $\Th_X(\emptyset) \simeq \pt$ for every anima $X$.
        \item The inclusion 
        \[
            (\Th^\perp)^*(\Fun^\mathrm{s.p.}) \subset \Fun^\mathrm{s.p., d.s.}
        \]
        follows immediately when we have shown that the functor $\Th \colon \AVB \to \An_*$ sends distinguished squares and squares that are split pushouts 
        with constant coefficients to pushouts. 
        Let 
        \[\begin{tikzcd}
            {(X,\emptyset)} & {(X,W)} \\
            {(X,0)} & {(S^W_X,0)}
            \arrow["{(\id,\infty)}", from=2-1, to=1-1]
            \arrow["{(\id,\infty)}", from=1-1, to=1-2]
            \arrow["{(\sigma_\infty,\id)}"', from=2-1, to=2-2]
            \arrow["{(\pi,\theta)}"', from=2-2, to=1-2]
        \end{tikzcd}\]
        be a simple distinguished square in $\AVB$.
        The image of this square under $S$ is given by the square 
        \[\begin{tikzcd}
            X & {S^W_X} \\
            {X \coprod X} & {S^W_X \coprod S^X_X}
            \arrow["{\mathrm{fold}}", from=2-1, to=1-1]
            \arrow["{\sigma_\infty}", from=1-1, to=1-2]
            \arrow["{\sigma_\infty \coprod \sigma_\infty}"', from=2-1, to=2-2]
            \arrow["{(\id, \sigma_\infty \pi)}"', from=2-2, to=1-2]
        \end{tikzcd}\]\footnote{The zigzag identity of the adjunction $\iota_0 \dashv S$ shows that the first coordinate of the right arrow is the identity.}
        After quotienting out the added points at $\infty$ we are left with the square 
        \[\begin{tikzcd}
            \pt & {\Th_X(W)} \\
            {X \coprod \pt} & {S^W_X \coprod \pt}
            \arrow[from=2-1, to=1-1]
            \arrow["\infty", from=1-1, to=1-2]
            \arrow["{\sigma_\infty \coprod \id}"', from=2-1, to=2-2]
            \arrow["{(\mathrm{can},\infty)}"', from=2-2, to=1-2]
        \end{tikzcd}\]
        which is obviously a pushout.

        Now let 
        \[\begin{tikzcd}
            {(A,W)} & {(B,W)} \\
            {(C,W)} & {(D,W)}
            \arrow[from=1-1, to=1-2]
            \arrow[from=2-1, to=2-2]
            \arrow[from=1-2, to=2-2]
            \arrow[from=1-1, to=2-1]
        \end{tikzcd}\]
        be a split pushout with constant coefficients. We want to show that 
        \[\begin{tikzcd}
            {\Th_A(W)} & {\Th_B(W)} \\
            {\Th_C(W)} & {\Th_D(W)}
            \arrow[from=1-1, to=1-2]
            \arrow[from=2-1, to=2-2]
            \arrow[from=1-2, to=2-2]
            \arrow[from=1-1, to=2-1]
        \end{tikzcd}\]
        is a pushout. By pulling back along points of $\cD$ we can assume $D \simeq \pt$. Then $W \simeq \R^n$ for some $n$.
        That is we look at the square 
        \[\begin{tikzcd}
            {(A,\R^n)} & {(B,\R^n)} \\
            {(C,\R^n)} & {(\pt,\R^n)}
            \arrow[from=1-1, to=1-2]
            \arrow[from=2-1, to=2-2]
            \arrow[from=1-2, to=2-2]
            \arrow[from=1-1, to=2-1].
        \end{tikzcd}\]
        The image under $\Th$ is 
        \[\begin{tikzcd}
            {\Sigma^n A_+} & {\Sigma^n B_+} \\
            {\Sigma^nC_+} & {\Sigma^nD_+}
            \arrow[from=1-1, to=1-2]
            \arrow[from=2-1, to=2-2]
            \arrow[from=1-2, to=2-2]
            \arrow[from=1-1, to=2-1]
        \end{tikzcd}\]
        which is clearly a pushout. 
        % Todo: Prove that in a lemma
    \end{itemize}

    To show that $\res|_{\An_*^\omega}$ is inverse to $(\Th^\perp)^*$ we construct functors $\mathcal F \colon \Fun(\An_*^\omega, \mathcal D) \to \Fun(\An_*^\omega, \mathcal D)$
    and $\mathcal G, \mathcal H \colon \Fun(\AVB^\vop(n), \mathcal D) \to \Fun(\AVB^\vop(n),\mathcal D)$ and natural transformations 
    \[
    \res|_{\An_*^\omega} (\Th^\perp)^* \xrightarrow{\rho_1} \mathcal F \xleftarrow{\rho_2}   \id_{\Fun(\An_*^\omega, \mathcal D)} 
    \]
    \[
        (\Th^\perp)^* \res|_{\An_*^\omega} \xleftarrow{\eta_1} \mathcal G \xto{\eta_2} \mathcal H \xot{\eta_3} \id_{\Fun(\AVB^\vop(n), \mathcal D)}   
    \]
    so that the natural transformations $\rho_i$ evaluete to equivalences on the full subcategory 
    $\Fun^{\mathrm{red.,s.p.}}(\An_*^\omega, \cD)$ and 
    likewise $\eta_i$ evaluate to equivalences on the subcategory 
    $\Fun^{\mathrm{d.s,red.,s.p.}}(\AVB^\vop(n), \mathcal D)$.

    Let $\mathcal F$ be the localization functor corresponding to the full subcategory of reduced functors. That is 
    \[
      \mathcal F (\phi) := \phi^{\mathrm{red}}\colon X \mapsto \mathrm{cof}\left(\phi(\pt) \to \phi(X)\right).
    \]
    It comes with a natural transformation $\rho_2 \colon \id \to \mathcal F$, that evaluates as the canonical map 
    $\phi(X) \to \phi^\mathrm{red}(X)$ into the cofiber.
    A simple calculation shows that 
    \[
    (\phi \circ \Th^\perp)|_{\An_*^\omega} (X) \simeq \mathrm{cof}(\phi(\pt_+) \to \phi(X_+)).
    \]
    The functorial split pushout 
    \[\begin{tikzcd}
        {\pt_+} & {X_+} \\
        \pt & X
        \arrow[from=1-1, to=1-2]
        \arrow[from=1-1, to=2-1]
        \arrow[from=2-1, to=2-2]
        \arrow[from=1-2, to=2-2]
        \arrow[bend right=30, dotted, from=1-2, to=1-1]
        \arrow[bend left=30, dotted, from=2-2, to=2-1]
    \end{tikzcd}\]
    induces a natural transformation 
    $ \rho_1 \colon \res|_{\An_*^\omega} (\Th^\perp)^* \to \mathcal F$.

    Let $\mathcal G$ be the functor that sends a functor $\psi \colon \AVB^\vop(n) \to \mathcal D$ to the functor 
    \[
    \mathrm{cof}\left(\psi\iotarnzero\pr \xto{\psi\iotarnzero\sigma_\infty}\psi\iotarnzero S^\perp\right)    
    \]
    and let $\mathcal H$ send $\psi$ to 
    \[
    \psi^\mathrm{red} := \mathrm{cof}(\psi(\infty) \colon \psi \iotaemptyempty \pr \to \psi).
    \]
    The evident map $\psi \to \psi^\mathrm{red}$ refines to a natural transformation 
    $\eta_3 \colon \id \to \mathcal H$.

    The square~\ref{fig:endosquareavb(n)} of endofucntors in Proposition~\ref{prop: AVB(n)} yields a functorial square 
    \[\begin{tikzcd}
        {\psi \iotaemptyempty} & \psi \\
        {\psi \iotarnzero \pr} & {\psi \iotarnzero S^\perp}
        \arrow[from=1-1, to=1-2]
        \arrow[from=2-2, to=1-2]
        \arrow[from=2-1, to=1-1]
        \arrow[from=2-1, to=2-2]
    \end{tikzcd}\]
    that induces a natural map $\mathcal G(\psi) \to \psi^\mathrm{red}$ that is a transformation $\eta_2 \colon \mathcal G \to \mathcal H$.
    Finally the split pushout square of functors 
    \[\begin{tikzcd}
        \pr & {S^\perp} \\
        {\mathrm{const}_{\pt}} & {\Th^\perp}
        \arrow["\sigma_\infty"', from=1-1, to=1-2]
        \arrow[from=1-2, to=2-2]
        \arrow[from=1-1, to=2-1]
        \arrow[from=2-1, to=2-2]
        \arrow[bend right=30, dotted, from=1-2, to=1-1]
        \arrow[bend left=30, dotted, from=2-2, to=2-1]
    \end{tikzcd}\]
    induces a square 
    \[\begin{tikzcd}
        {\psi \iotarnzero \pr} & {\psi \iotarnzero S^\perp} \\
        {\psi \iotarnzero \mathrm{const}_{\pt}} & {\psi \iotarnzero \Th^\perp}
        \arrow[from=1-1, to=1-2]
        \arrow[from=1-2, to=2-2]
        \arrow[from=1-1, to=2-1]
        \arrow[from=2-1, to=2-2]
    \end{tikzcd}\]
    that ultimately leads to a natural map $\mathcal G(\psi) \to \psi|_{\An_*^\omega} \circ \Th^\perp$ that refines 
    to a natural transformation 
    $ \eta_1 \colon \mathcal G \to (\Th^\perp)^* \res|_{\An_*^\omega}$.

    The subcategories of the functor categories in question are chosen precisely such that all of the transformations 
    $\eta_i, \rho_j$ are equivalences when restricted to them, that is:
    \begin{itemize}
        \item $\rho_1 \colon (\phi \circ \Th^\perp)|_{\An_*^\omega} \to \phi^\mathrm{red}$ is an equivalence if $\phi$ preserves split pushouts,
        \item $\rho_2 \colon \phi \to \phi^\mathrm{red}$ is an equivalence if $\phi$ is reduced,
        \item $\eta_1 \colon \mathcal G(\psi) \to \psi_{\An_*^\omega} \circ \Th^\perp$ is an equivalence if $\psi$ preserves split pushouts in the anima variable that are 
        constant on the vector bundle coordinates, 
        \item $\eta_2 \colon \mathcal G(\psi) \to \psi^\mathrm{red}$ is an equivalence if $\psi$ sends distinguished squares to pushouts,
        \item $\eta_3 \colon \psi \to \psi^\mathrm{red}$ is an equivalence if $\psi$ sends objects of the form $\iotaemptyempty(X)$ to the zero object.
    \end{itemize}

\end{proof}

\begin{remark}
    This proof also shows that 
    \[\begin{tikzcd}
        {\Fun^{\mathrm{red.,s.p.}}(\An_*^\omega,\mathcal D)} & {\Fun^{\mathrm{red.,s.p.}}(\AVB^\vop(n)|_{\An^\omega},\mathcal D)}
        \arrow[""{name=0, anchor=center, inner sep=0}, "{(\Th^\perp)^*}"', shift right=2, from=1-1, to=1-2]
        \arrow[""{name=1, anchor=center, inner sep=0}, "{\res|_{\An_*^\omega}}"', shift right=2, from=1-2, to=1-1]
        \arrow["\dashv"{anchor=center, rotate=90}, draw=none, from=0, to=1]
    \end{tikzcd}\]    
is a Bousfield co-localization. The category of colocal objects is the full subcategory 
\[
\Fun^{\mathrm{d.s,red.,s.p.}}(\AVB^\vop(n), \mathcal D) \subset \Fun^{\mathrm{red.,s.p.}}(\AVB^\vop(n), \mathcal D) 
\]
of functors that additionally send distinguished squares to pushouts.
\end{remark}


% \section{Genuine Cohomology Theories}\label{section:Genuinecohomologytheories}

\begin{definition}[Genuine Cohomology Theories]
  
    A \emph{genuine cohomology theory} consists of a collection
  $E^V(X)$ of abelian groups, where $X$ runs trough all compact anima and
  $V$ is a vector bundle over $X$.
  For each map $f \colon X \to Y$ and a vector bundle $W$ over $Y$ we have induced
  maps
  $f^W \colon E^{W}(Y) \to E^{f^*W}(X)$.
  For each map $\phi \colon V \to W$ of vector bundles over $X$ we have
  induced maps
  $\phi_X \colon E^{V}(X) \to E^{W}(X)$.
  The usual functoriality conditions hold, that is
  \begin{itemize}
    \item If $f, \phi$ above are the identity maps then they induce the identity.
    \item We have $(g \circ f)^V = f^{g^*V} \circ g^V$ and
    $(\phi \circ \psi)_X = \phi_X \circ \psi_X$.
  \end{itemize}
  
  We require these collection of abelian groups and induced maps to satisfy some
  properties
  
  \begin{enumerate}
    \item[(i)]\label{homotopyinvariance} (Homotopy Invariance)\\
     If $\phi \simeq \psi$ are homotopic morphisms of vector bundles $V \to W$ over $X$, then 
     the induced morphisms $\phi_X = \psi_X$ are equal.
     If $H \colon f \simeq g$ is a homotopy, then it induces an equivalence 
     $\phi_H \colon f^*V \simeq g^*V$ between vector bundles over $X$.
     We require that $({\phi_H})_X \circ f^V = g^V$, i.e. the following commutes 
     \[\begin{tikzcd}
      && {E^{f^*V}(X)} \\
      {E^V(X)} \\
      && {E^{g^*V}(X)}
      \arrow["{({\phi_H})_X}", from=1-3, to=3-3]
      \arrow["{f^V}", from=2-1, to=1-3]
      \arrow["{g^V}"', from=2-1, to=3-3]
    \end{tikzcd}\]
    \iffalse
    \item[(ii)]\label{additivity} (Additivity)\\
    For each collection $(X_i,V_i)_{i \in I}$
    of anima $X_i$ and vector bundles $V_i$ over $X_i$ the induced homomorphism
    \[
      E^{\coprod_i V_i}(\coprod_i X_i) \to \prod_i E^{V_i}(X_i)
      \]
      is an isomorphism.
    \fi 
      \item[(ii)]\label{beckchavalley} (Beck-Chevalley)\\
      Let $f \colon X \to Y$ be
      a map of anima and $\phi \colon V \to W$ be a map of vector bundles over $Y$.
      Let $f^*\phi$ be the induced map $f^* V \to f^* W$ of vector bundles over $X$.
      Then the following square commutes
      \[\begin{tikzcd}
        {E^{V}(Y)} & {E^{f^*V}(X)} \\
        {E^W(Y)} & {E^{f^*W}(X)}
        \arrow["{\phi_Y}"', from=1-1, to=2-1]
        \arrow["{f^W}"', from=2-1, to=2-2]
        \arrow["{f^V}", from=1-1, to=1-2]
        \arrow["{(f^*\phi)_X}", from=1-2, to=2-2]
      \end{tikzcd}\]
  
    \item[(iii)]\label{reduced} (Reducedness)\\
      Let $X$ be a compact anima and 
      let $\emptyset$ be the empty bundle over $X$. Then 
      $E^\emptyset(X) \cong 0$.
    \item[(iv)]\label{mayervietoris} (Mayer-Vietoris)\\
    For each pushout square of anima
    \[\begin{tikzcd}
      {X_1} & {X_2} \\
      {X_3} & {X_4}
      \arrow["{f_1}", from=1-1, to=1-2]
      \arrow["{f_3}", from=1-2, to=2-2]
      \arrow["{f_2}"', from=1-1, to=2-1]
      \arrow["{f_4}"', from=2-1, to=2-2]
      \arrow["\lrcorner"{anchor=center, pos=0.125, rotate=180}, draw=none, from=2-2, to=1-1]
      \arrow["H"{description}, Rightarrow, from=2-1, to=1-2]
    \end{tikzcd}\]
    and vector bundle $V$ over $X_4$ the induced sequence
    \[
      E^{V}(X_4) \xto{(f_3^V,f_4^V)} E^{f_3^*V}(X_2)\oplus E^{f_4^*V}(X_3)
      \xto{f_1^{f_3^*V} - ({\phi_H})_X \circ f_2^{f_4^*V}} E^{f_1^*f_3^*V}(X_1)
      \]
    is exact in the middle.
    \item[(v)]\label{thomiso}(The Thom Isomorphism)\\
    Let $V,W$ be two vector bundles over $X$. Let
    \[E^{V\oplus W}(S_X^V,X) =
    E^{p^*(V\oplus W)}(S_X^V) \ominus E^{V\oplus W}(X)
    \] be the complement of
    the direct summand inside $E^{p^*(V\oplus W)}(S_X^{V})$ of $E^{V\oplus W}(X)$
    due to the retraction $\sigma^{p^*(V\oplus W)} \circ p^{V\oplus W} = \id$.
    Then the map
    \[
    E^W(X) \xto{p^W} E^{p^*W}(S_X^V) \xto{(\theta_{(X)}^{(V,W)})_{S_X^V}} E^{p^*(V \oplus W)}(S_X^V) \to E^{V\oplus W}(S_X^V,X)
    \]
    is an isomorphism.
  \end{enumerate}
  
  \end{definition}
  
  We want to make the above definition rigorous, by observing that some of the statet 
  axioms of a genuine cohomology theory are implemented by a functor 
  out of $h\AVB^\op$. In general we can describe the homotopy category of the 
  total category of a Cartesian fibration as follows:
  
  \begin{remark}
  Let $p \colon \cC \to \mathcal I$ be a Cartesian fibration. We want to give a generators and 
  relation discription 
  of the homotopy category of $\cC$ in terms of the functor ${\mathrm{d}} p \colon \mathcal I^\op \to \catinfty$ that 
  classifies $p$. 
  First, consider the functor $h {\mathrm{d}} p \colon \mathcal I^\op \to \catinfty$, that 
  we obtain by postcomposing ${\mathrm{d}} p$ with the functor $h \colon \catinfty \to \catinfty$
  that sends an $\infty$-category $\cD$ to its homotopy category $h\cD$.
  The natural transformation $\id_{\catinfty} \to h$ induces a functor 
  $h\cC \to h\int_{\mathcal I^\op} h {\mathrm{d}} p$.
  We claim that this functor induces an equivalence between ordinary categories.
  TODO 
  Since $h {\mathrm{d}} p$ maps into the full subcategory $\mathcal{C}\mathrm{at}_1$ of 
  $\catinfty$ of ordinary categories, which is a $(2,1)-category$, the functor $h {\mathrm{d}} p$
  factorizes over the homotopy-$2$ category $\tau_{\leq 2} \mathcal I$. 
  In fact, the induced functor $\int_{\mathcal I^\op} h {\mathrm{d}} p 
  \to \int_{\mathcal \tau_{\leq 2}I^\op} h {\mathrm{d}} p $ is an 
  equivalence of categories.
  We conclude, that we can compute the homotopy category $h\cC$ by computing 
  $h \int_{\tau_{\leq 2}} h {\mathrm{d}} p$, i.e. the homotopy category 
  of the Grothendieck construction of the $2$-functor from the homotopy-$2$ category of 
  $\mathcal I^\op$ to the $(2,1)$-category of $1$-categories, given by the 
  fiber wise homotopy category of the straightening of $p$. 
  Luckily, there is a formula for that construction:
  
  An object of $h\cC$ is a pair $(i, [c])$ consisting of an object $i \in \mathcal I$
  and an equivalence class of objects $[c] \in h {\mathrm{d}} p(i)$.
  
  A morphism $(i, [c]) \to (j, [d])$ is an equivalcence class represented by a pair $(f, [\phi])$ consisting 
  of a morphism 
  $f \colon j \to i$ in $\mathcal I$ and homotopy class of morphisms $[\phi \colon c \to f^*d]$. 
  Two of such pairs $(f, [\phi])$ and $(g, [\psi])$ get identified if there exists 
  a homotopy $H \colon f \to g$ in $\mathcal I$ such that with the induced natural 
  isomorphism $H^* \colon f^* \simeq g^*$ between functors 
  $h {\mathrm{d}} p(j) \to h {\mathrm{d}} p(i)$
  the post composition $c \xto{[\phi]} f^*d \xto{H^*} g^*d$ is equal to the class 
  $[\psi]$.
  We can even further simplify the the discription of $h\cC$ by abusing the fact 
  that we can factorize every morphism in $\cC$ in a fiber wise morphism followed by a Cartesian 
  morphism. Let $(f,[\phi]) \colon (i,c) \to (j, d)$ represent a morphism in $h \cC$.
  Then we can write it as the composite $(f,[\phi]) = (f, [\id]) \circ (\id, [\phi])$
  \[\begin{tikzcd}
      {(i,c)} && {(j,d)} \\
      & {(i,f^*d)}
      \arrow["{(f,[\phi])}", from=1-1, to=1-3]
      \arrow["{(\id,[\phi])}"', from=1-1, to=2-2]
      \arrow["{(f,\id)}"', from=2-2, to=1-3].
  \end{tikzcd}\]
  The composition of two such decompositions $(f,[\phi]) = (f, [\id]) \circ (\id, [\phi])$ and 
  $(g,[\psi]) = (g, [\id]) \circ (\id, [\psi])$ is given by the decomposition $(\id, [\psi \phi])\circ (\id, gf)$
  \[\begin{tikzcd}
      {(i,c)} \\
      {(i,f^*d)} & {(j,d)} \\
      {(i,f^*g^*e)} & {(j, g^*e)} & {(k,b)}
      \arrow["{(\id,[\phi])}"', from=1-1, to=2-1]
      \arrow["{(f,\id)}"', from=2-1, to=2-2]
      \arrow["{(\id,[\psi])}", from=2-2, to=3-2]
      \arrow["{(g,\id)}"', from=3-2, to=3-3]
      \arrow["{(\id,[\psi])}"', dashed, from=2-1, to=3-1]
      \arrow["{(f,\id)}"', dashed, from=3-1, to=3-2]
  \end{tikzcd}\].
  Thus a functor $F \colon h\cC \to \cD$ into an ordinary category consists of the data 
  \begin{itemize}
    \item A specified object $F(i,c) \in \cD$ for each pair $i \in \mathcal I$ and $c \in  h {\mathrm{d}} p(i)$.
    \item A morphism $F(f) \colon F(i,f^*d) \to F(j,d)$ for every morphism $f \colon i \to j$ in $\cC$ and fixed object $d \in h {\mathrm{d}} p(j)$.
    \item A morphism $F([\phi]) \colon F(i, c) \to F(i, d)$ for every homotopy class of morphisms $[\phi \colon c \to d]$ in $ h {\mathrm{d}} p(i)$.
    \item A specified natural isomorphism $\epsilon_{H,d} \colon F(i,f^*d) \xto{\cong} F(i, g^*d)$ for every homotopy $H \colon f \simeq g$ between morphisms 
    $ i \to j $ in $\cC$ and object $d \in h {\mathrm{d}} p(j)$
  \end{itemize}
  such that 
  \begin{itemize}
    \item (Beck-Chavalley) Both composites agree $F(\phi) \circ F(f) = F(f) \circ F(f^*\phi)$
    \[\begin{tikzcd}
      {F(i,f^*c)} & {F(j, c)} \\
      {F(i,f^*d)} & {F(j,d)}
      \arrow["{F([f^*\phi])}"', from=1-1, to=2-1]
      \arrow["{F(f)}"', from=2-1, to=2-2]
      \arrow["{F(f)}", from=1-1, to=1-2]
      \arrow["{F([\phi])}", from=1-2, to=2-2]
    \end{tikzcd}\]
    \item $F(f) = F(g) \circ \epsilon_{H,d}$
    \[\begin{tikzcd}
      {F(i,f^*d)} \\
      && {F(j,d)} \\
      {F(i,g^*d)}
      \arrow["{\epsilon_{H,d}}"', from=1-1, to=3-1]
      \arrow["{F(f)}", from=1-1, to=2-3]
      \arrow["{F(g)}"', from=3-1, to=2-3]
    \end{tikzcd}\]
  \end{itemize}
  \end{remark}
  With the last remark we can now give an equivalent definition of a 
  genuine cohomology theory. 
  
  \begin{definition}[Genuine Cohomology Theories II]\label{truedef}  
    A \emph{genuine cohomology theory} $E$ is a functor $E \colon h (\AVB^\op) \to \mathrm{Ab} \colon (X,V) \mapsto E^V(X)$
    that satisfies the following axioms 
    \begin{enumerate}
      \item[(iii)](Reducedness)\\
      Let $X$ be a compact anima and 
      let $\emptyset$ be the empty bundle over $X$. Then 
      $E^\emptyset(X) \cong 0$.
    \item[(iv)](Mayer-Vietoris)\\
    For each pushout square of anima
    \[\begin{tikzcd}
      {X_1} & {X_2} \\
      {X_3} & {X_4}
      \arrow["{f_1}", from=1-1, to=1-2]
      \arrow["{f_3}", from=1-2, to=2-2]
      \arrow["{f_2}"', from=1-1, to=2-1]
      \arrow["{f_4}"', from=2-1, to=2-2]
      \arrow["\lrcorner"{anchor=center, pos=0.125, rotate=180}, draw=none, from=2-2, to=1-1]
      \arrow["H"{description}, Rightarrow, from=2-1, to=1-2]
    \end{tikzcd}\]
    and vector bundle $V$ over $X_4$ the induced sequence
    \[
      E^{V}(X_4) \xto{(f_3^V,f_4^V)} E^{f_3^*V}(X_2)\oplus E^{f_4^*V}(X_3)
      \xto{f_1^{f_3^*V} - ({\phi_H})_X \circ f_2^{f_4^*V}} E^{f_1^*f_3^*V}(X_1)
      \]
    is exact in the middle.
    \item[(v)](Thom Isomorphism)\\
    Let $V,W$ be two vector bundles over $X$.
    \iffalse 
     Let
    \[E^{V\oplus W}(S_X^V,X) =
    E^{p^*(V\oplus W)}(S_X^V) \ominus E^{V\oplus W}(X)
    \] be the complement of
    the direct summand inside $E^{p^*(V\oplus W)}(S_X^{V})$ of $E^{V\oplus W}(X)$
    due to the retraction $\sigma^{p^*(V\oplus W)} \circ p^{V\oplus W} = \id$.
    \fi 
    Then the map
    \[
    E^W(X) \xto{p^W} E^{p^*W}(S_X^V) \xto{(\theta_{(X)}^{(V,W)})_{S_X^V}} E^{p^*(V \oplus W)}(S_X^V) \to E^{V\oplus W}(S_X^V,X)
    \]
    is an isomorphism.
    \end{enumerate}
  \end{definition}  
  
  \begin{remark}
    Our motivation for this specific equivalent formutlation of the Eilenberg-Steenrod axioms for cohomology theories is, that it is generic enough, to 
    translate it to different contexts, e.g. equivariant homotopy theory, but it is specific enough to model the `correct' notion of cohomology theories
    instead of naive ones. We devoted the last chapter to propose a definition for equivariant genuine cohomology theories.
  \end{remark}
  
  \begin{definition}
    A \emph{cohomology theory of finite spectra} $E$ is a functor 
    $ (h \Sp^\omega)^\op \to \mathrm{Ab}$ that satisfies the following axioms:
    \begin{enumerate}
      \item[(i)](Reducedness) $E(0) \simeq 0$
      \item[(ii)](Exactness in the middle) Let $X \xto{f} Y \xto{g} Z$ be 
      a fiber sequence of finite spectra and let $a \in \ker E(f)$, 
      then there exists a $b \in E(Z)$ such that $E(g)(b) = a$.
    \end{enumerate}
  \end{definition}
  
  \begin{proposition}[Adams version of Brown Representability]
    Let $E$ be a cohomology theory of finite spectra. Then there exists a spectrum 
    $\mathcal E$ and a natural isomorphism $[-, \mathcal E] \xto{\simeq} E$ of functors 
    $(h \Sp^\omega)^\op \to \mathrm{Ab}$.
    It follows that the spectrum $\mathcal E$ is necesseraly unique.
  \end{proposition}

  \begin{proposition}
    The functor $E \mapsto E \circ \Th^-$ is an equivalence between the category
    \[\{\mathrm{Cohomology\ Theories\ on\ }\Sp^\omega\}\] 
    of cohomology theories on compact spectra and 
    the category \[
        \{\mathrm{Genuine\ Cohomology\ Theories}\}
       \] of genuine cohomology theories.
    In particular exists vor every genuine cohomology theory $E$ a unique spectrum $\mathcal E$ and a natural 
    equivalence 
    \[
      E^V(X) \simeq [X^{-V}, \mathcal E]  .
    \]
  \end{proposition}

\begin{proof}
    The category of genuine cohomology theories is modeled 
    by the full subcategory 
    \[
      \Fun^{\mathrm{M.V., d.s., red.}}(\AVB^\vop|_{\An^\omega}, \mathrm{Ab}^\op)^\op  
    \]
    of the functor category $\Fun(\AVB^\vop|_{\An^\omega}, \mathrm{Ab}^\op)^\op$ on those functors 
    $E \colon \AVB^\vop|_{\An^\omega} \to \mathrm{Ab}^\op$ that satisfies the axioms of Definition~\ref{truedef}.
    By Theorem~\ref{thrm:dream} the functor $E \mapsto E \circ \Th^-$ identifies the category of genuine cohomology 
    theories with a full subcategory of 
    \[
        \Fun^{\mathrm{add.}}(\Sp^\omega, \mathrm{Ab}^\op)^\op.
    \]
    We claim that this subcategory is precisely the one which is spanned by the cohomology theories.
    To show this we have to prove two things 
    \begin{itemize}
        \item Let $\mathcal E$ be a spectrum, then the functor 
            \[
                [\Th^-(-), \mathcal E] \colon \AVB^\vop|_{\An^\omega} \to \mathrm{Ab}^\op
            \]
            satisfies the Mayer-Vietoris axiom of Definition~\ref{truedef}.
        \item Let $E$ be a genuine cohomology and let $\tilde E \colon \Sp^\omega \to \Ab^\op$
              be the unique functor such that $E \simeq \tilde E \circ \Th^-$ holds, according to Theorem~\ref{thrm:dream}.
              Then $\tilde E$ is a cohomology theory.

    \end{itemize}
    The first claim follows from the fact that the functor $\Th^-$ sends pushout squares with constant coefficients 
    to pushout squares in spectra. 
    % \[\begin{tikzcd}
    %     {(A,V)} & {(B,V)} \\
    %     {(C,V)} & {(D,V)}
    %     \arrow[from=1-1, to=1-2]
    %     \arrow[from=1-2, to=2-2]
    %     \arrow[from=1-1, to=2-1]
    %     \arrow[from=2-1, to=2-2]
    % \end{tikzcd}\]
    The second claim follows from the fact that any pushout square
    \[\begin{tikzcd}
        {\mathcal A} & {\mathcal B} \\
        {\mathcal C} & {\mathcal D}
        \arrow[from=1-1, to=1-2]
        \arrow[from=1-2, to=2-2]
        \arrow[from=1-1, to=2-1]
        \arrow[from=2-1, to=2-2]
        \arrow["\lrcorner"{anchor=center, pos=0.125, rotate=180}, draw=none, from=2-2, to=1-1]
    \end{tikzcd}\]
     is of the form 
    \[\begin{tikzcd}
        {\Sigma^{\infty - n}A} & {\Sigma^{\infty - n}B} \\
        {\Sigma^{\infty - n}C} & {\Sigma^{\infty - n}D}
        \arrow[from=1-1, to=1-2]
        \arrow[from=1-2, to=2-2]
        \arrow[from=1-1, to=2-1]
        \arrow[from=2-1, to=2-2]
        \arrow["\lrcorner"{anchor=center, pos=0.125, rotate=180}, draw=none, from=2-2, to=1-1],
    \end{tikzcd}\]
    where 
    \[\begin{tikzcd}
        A & B \\
        C & D
        \arrow[from=1-1, to=1-2]
        \arrow[from=1-2, to=2-2]
        \arrow[from=1-1, to=2-1]
        \arrow[from=2-1, to=2-2]
        \arrow["\lrcorner"{anchor=center, pos=0.125, rotate=180}, draw=none, from=2-2, to=1-1]
    \end{tikzcd}\]
    is a pushout in $\An_*^\omega$.
    Consider the following pushout in $\AVB^\vop|_{\An^\omega}$:
    \[\begin{tikzcd}
        {(A,\R^n)} & {(B,\R^n)} \\
        {(C,\R^n)} & {(D,\R^n)}
        \arrow[from=1-1, to=1-2]
        \arrow[from=1-2, to=2-2]
        \arrow[from=1-1, to=2-1]
        \arrow[from=2-1, to=2-2]
        \arrow["\lrcorner"{anchor=center, pos=0.125, rotate=180}, draw=none, from=2-2, to=1-1]
    \end{tikzcd}.\]
    By the (Mayer-Vietoris) axiom it gets send to a sequence 
    \[
    E^n(D) \to E^n(B) \oplus E^n(C) \to E^n(A)    
    \]
    which is exact in the middle. Therefore also the sequence 
    \[
    \tilde E(\Sigma^{\infty - n} D) \to \tilde E(\Sigma^{\infty - n} B) \oplus \tilde E(\Sigma^{\infty - n} C)  \to \tilde E(\Sigma^{\infty - n} A)     
    \]
    is exact in the middle since we have 
    \[
    \tilde E(\Sigma^{\infty -n} X) \simeq \fib( E^n X \to E^n \pt)
    \]
    for any pointed finite space $X$.
\end{proof}


\section{Equivariant Cohomology Theories}

Throughout this section fix a compact Lie group $G$.

\begin{definition}
We consider the category of $G$-anima which is defined as the the presheaf category 
\[
\An_G := \Psh(\mathrm{Orb}_G).
\]
For a closed subgroup $H \leq G$ we let $\AffLin^H$ be the category consisting of finite dimensional euclidean vector spaces with an 
$H$-action through linear isometries. A morphism $f \colon V \to W$ is an affine linear morphism that 
decomposes as an equivariant linear isometric embedding $\iota \colon V \hookrightarrow W$ followed by a translation along a vector 
$w \in (\iota(V)^\perp)^H$ out of the $H$ fixed points of the orthogonal complement of $\iota(V)$.
Thus $\AffLin^{\{e\}} \simeq \AffLin$.
The assignment $G/H \mapsto \AffLin^H$ makes $\AffLin^{-}$ into a $G$-category, that is a functor 
$\mathrm{Orb}_G^\op \to \catinfty$.
Let $X \mapsto \AffLin^X$ the unique limit preserving extension $\An_G^\op \to \catinfty$ of $\AffLin^{-}$.
Let 
\[
  \AVB^\vop_G := \int_{X \in \An_G^\op} (\AffLin^{X})^\op   
\]
be the unstraightening of the functor $\AffLin^{-}$.
The category $\AVB^\vop_G$ is the vertical opposite of the \emph{category of (equivariant) affine vector bundles over $G$} 
\[
  \AVB_G := \int_{X \in \An_G} \AffLin^X.
\]
For every $G$-representation $V \in \AffLin^G$ there is a functor 
\[
\iota_V \colon \An_G \to \AVB_G    
\]
that equips a $G$-anima with the trivial $V$ vector bundle over it. One can construct $\iota_V$ 
as in the non-equivariant case via the source of a cartesian lift 
\[\begin{tikzcd}
	\pt && {\Fun(\An_G,\AVB_G)} \\
	{\Delta^1} && {\Fun(\An_G,\An_G)}
	\arrow["{\pr_*}", from=1-3, to=2-3]
	\arrow["{\id \to \mathrm{const}_{G/G}}"', from=2-1, to=2-3]
	\arrow["{\mathrm{const}_{(G/G,V)}}", from=1-1, to=1-3]
	\arrow["{\mathrm{target}}"', from=1-1, to=2-1]
	\arrow[dashed, from=2-1, to=1-3].
\end{tikzcd}\]

As in the case $G = \{e\}$ we have distinguished squares in the category $\AVB^\vop_G$




Let $(\An_G)^\omega_*$ be the category of pointed objects in the category compact $G$-anima.
A \emph{$G$-cohomology theory $E$} is a functor $E \colon ((\An_G)^\omega_*)^\op \times \AffLin_G^\simeq \to \Ab$
together with natural equivalences 
\[
    \sigma_V \colon E(X,W) \to E(S^V \wedge X, V \oplus W)
\]
such that 
$E(-, \R^\bullet_{\mathrm{triv}} \oplus V)$ is a cohomology theory for every $V$ and the following diagrams commute
\[\begin{tikzcd}
	& {E(X,V)} \\
	{E(S^W\wedge X,W\oplus V)} && {E(S^U\wedge X, U \oplus V)} \\
	{E(S^U\wedge S^W \wedge X, U \oplus W \oplus V)} && {E(S^W \wedge S^U \wedge X, W \oplus U \oplus V)}
	\arrow["{\sigma_W}", from=1-2, to=2-1]
	\arrow["{\sigma_U}"', from=1-2, to=2-3]
	\arrow["{\sigma_U}", from=2-1, to=3-1]
	\arrow["{\sigma_W}"', from=2-3, to=3-3]
	\arrow["\simeq", from=3-1, to=3-3]
\end{tikzcd},\]

\[\begin{tikzcd}
	{E(X;W)} & {E(S^V \wedge X; V \oplus W)} \\
	{E(S^{U \oplus V} \wedge X; U \oplus V \oplus W)} & {E(S^U \wedge S^V \wedge X;U \oplus V \oplus W)}
	\arrow["{\sigma_V}", from=1-1, to=1-2]
	\arrow["\simeq"', from=2-1, to=2-2]
	\arrow["{\sigma_U}", from=1-2, to=2-2]
	\arrow["{\sigma_{U\oplus V}}"', from=1-1, to=2-1].
\end{tikzcd}\]

%TODO define genuine cohomology theory 

\end{definition}

\begin{definition}
    Let $E \colon (\AVB^\vop_G|_{\An_G^\omega})^\op \to \Ab$ be a genuine cohomology theory, $Y \to Z$ a map of $G$-anima and fix $V \in \AffLin^G$, then we define 
    the $V$-th $E$-cohomology group of $Z$ relative to $Y$ as
    \[
      E^V(Z,Y) := E(Z,Y;V) := \ker ( E(\iota_V Z \to \iota_V Y)). 
    \]
    For a pointed $G$-anima $X$ we set 
    \[
      \tilde E^V(X) := \tilde E(X;V) := E(X, \pt; V).
    \]
\end{definition}
\begin{lemma}

    The diagram 
    \[\begin{tikzcd}
        {E(\pt, W)} & {E(X,W)} \\
        {E(S^V\vee X,V \oplus W)} & {E(S^V \times X, V \oplus W)} \\
        {E(\pt, V \oplus W)} & {E(S^V \wedge X, V \oplus W)}
        \arrow[from=1-2, to=1-1]
        \arrow[from=1-1, to=2-1]
        \arrow[from=2-2, to=2-1]
        \arrow[from=1-2, to=2-2]
        \arrow[from=3-2, to=2-2]
        \arrow[from=3-1, to=2-1]
        \arrow[from=3-2, to=3-1]
    \end{tikzcd}\]
    induces on horizontal kernels a zigzag 
    \[
    \tilde E(X,W) \to E(S^V \times X, S^V \vee X ; V \oplus W) \leftarrow \tilde E(S^V \wedge X, V \oplus W),
    \]
    where both morphisms are isomorphisms.
\end{lemma}

\begin{proof}
By the snake-lemma the following ladder diagram 
    \[\begin{tikzcd}
        & 0 & {E(S^V\times X;V\oplus W)} & {E(S^V\times X;V\oplus W)} & 0 \\
        0 & {\tilde E(S^V;V\oplus W)} & {E(S^V\vee X;V\oplus W)} & {E(X;V\oplus W)} & 0
        \arrow[from=2-2, to=2-3]
        \arrow[from=2-3, to=2-4]
        \arrow[from=2-4, to=2-5]
        \arrow[from=2-1, to=2-2]
        \arrow[from=1-3, to=2-3]
        \arrow[from=1-2, to=1-3]
        \arrow[from=1-2, to=2-2]
        \arrow[from=1-3, to=1-4]
        \arrow[from=1-4, to=1-5]
        \arrow[two heads, from=1-4, to=2-4]
    \end{tikzcd}\]
induces a long exact sequence
\[
0 \to E(S^V \times X, S^V \vee X; V \oplus W) \to E(X, W) \to \tilde E(S^V; V \oplus W) \to \dots 
\]
The Thom-isomorphism axiom shows that $\tilde E(S^V; V\oplus W) \simeq E(\pt, W)$. 
Hence \[E(S^V \times X, S^V \vee X; V \oplus W) \simeq \tilde E(X;W).\]
A careful examination shows that the left arrow of the span estabishes this isomorphism. The right arrow of the span 
is an isomorphism because of the Mayer-Vietoris axiom.
\end{proof}
\begin{definition}
    We define 
    \[
    \sigma_V \colon \tilde  E^W(X) \to \tilde E^{V\oplus W}(S^V \wedge X)    
    \]
    to be the resulting isomorphism.
\end{definition}

\begin{lemma}
    Let $E \colon (\AVB^\vop_G|_{\An_G^\omega})^\op \to \Ab$ be a genuine cohomology theory.
    Then $\tilde E$ is a $G$-cohomology theory.
\end{lemma}
\begin{proof}
    % TODO Proof
    By definition of the Mayer-Vietoris axiom we have that $\tilde E(-, \R^\bullet \oplus V)$ is an ordinary cohomomology theory 
    for every $V$.
    We will shows that $\sigma_{U\oplus V} \simeq \sigma_U \circ \sigma_V$ holds. Then we are finished since the axiom 
    $\sigma_U \circ \sigma_V \simeq \sigma_V \circ \sigma_U$ then follows from this and the functoriality of 
    $\sigma_{-}$ in $\AffLin^G$.
    The essential part in proving $\sigma_{U\oplus V} \simeq \sigma_U \circ \sigma_V$  comes from the commutativity of the square 
    \[\begin{tikzcd}
        {(X;W)} & {(S^V\times X;V \oplus W)} \\
        {(S^{U\oplus V}\times X;U\oplus V\oplus W)} & {(S^U\times S^V \times X;U\oplus V\oplus W)}
        \arrow["{(\pr,\theta_V)}"', from=1-2, to=1-1]
        \arrow["{(\pr,\theta_{U\oplus V})}", from=2-1, to=1-1]
        \arrow["{(q\times X;\id)}", from=2-2, to=2-1]
        \arrow["{(\pr,\theta_U)}"', from=2-2, to=1-2]
    \end{tikzcd}\]
    where $q\colon S^U \times S^V \to S^{U\oplus V}$ is the quotient map.
\end{proof}


\begin{definition}
Let $\cC$ be a symmetric monoidal category. Let $\cC //^\mathrm{lax} \cC^\simeq$ be the lax 
quotient of the action of the core $\cC^\simeq$ on $\cC$, which is defined as the unstraightening of the 
functor 
\[
B\cC^\simeq \to \catinfty \colon \pt \mapsto \cC.
\]
One can think of an object of $\cC //^\mathrm{lax} \cC^\simeq$ as an object of $\cC$, while a morphism 
\[
  c \to d  
\]
consists of the data of an object $e \in \cC$ and a morphism 
\[
  c \otimes e \to d  
\]
of $\cC$.
\end{definition}
\begin{definition}
Since every pair of objects $c,e \in \cC$ defines a functor 
\[
  \cC^\op(c) \xto{-\otimes \overrightarrow{e}} \cC^\op(c \otimes e)  
\]
and and every morphism $c \to d$ defines a morphism 
\[
  \cC^\op(c) \to \cC^\op(d)  
\]
in a compatible way we find that the assignment
\[
    c \mapsto \cC^\op(c)
\]
defines a functor 
\[
 \cC^\op(-) \colon \cC //^\mathrm{lax} \cC^\simeq \to \catinfty. 
\]
Since the forgetful functors $\cC^\op(c) \to \cC^\op$ are compatible with the structure maps of the diagram $\cC^\op(-)$ we obtain 
a natural transformation 
\[\cC^\op(-) \Rightarrow \mathrm{const_{\cC^\op}}.\]
\end{definition}

\begin{lemma}

\end{lemma}


% \section{The General Philosophy}

Let $\cal X$ be a topos and let $(\cal V, \oplus)$ be a pointed symmetric monoidal $\cal X$-category together with a 
symmetric monoidal $\cal X$-fucntor $S \colon \cal V \to \cal X^\omega_*$, this in particular consists 
of a family of symmetric monoidal functors 
\begin{eqnarray*}
({\cal V}^X,\oplus) &\to& (\cal X_{X/\cdot /X}, \wedge_X)    \\
V & \mapsto & S^V.
\end{eqnarray*}
Furthermore, we assume that $S$ is corepresented by the 
unit $0 \in \cal V^\pt$.

\begin{proposition}
Let $\cD$ be an idempotent complete pointed category. Then there is 
an equivalence of categories 
\[
  \mathrm{Gen}^\mathrm{split}(\int_{\cal X^\omega} {\cal V}^{\op}(V), \cD) \simeq \mathrm{Exc}_*^\mathrm{split}(\cal X_*^\omega, \cD)  
\]
\end{proposition}

\begin{proposition} We have an equivalence of $\cal X$-categories 
    \[
        \colim_{V \in \cal V //^\mathrm{lax} \cal V^\simeq} {\cal V}^{\op}(V) \simeq \cal V^\op   
    \]
    where the colimit is taken in the catgory of $\cal X$-categories.
\end{proposition}

\begin{proposition} Let $\cal V \to \cal C$ be a symmetric monoidal $\cal X$-functor.
    Then 
    \[
    \colim^{\in \Gamma}_{V \in \cal V //^\mathrm{lax} \cal V^\simeq} \cC   
    \]
    is a model for $\mathcal{C}[\mathcal{V}^{- 1}]$ in $\mathrm{CAlg(\Gamma)}$, where 
    $\Gamma \in \{\cal X -\catinfty, \cal{X} - \mathrm{Pr}^\mathrm{L} \}$
\end{proposition}

\begin{corollary}
Let $\cD$ be an idempotent complete pointed category. Then there is an 
equivalence of categories 
\[
    \mathrm{Gen}^\mathrm{split}(\int_{\cal X ^\omega} \cal V ^\op, \cD) \simeq \mathrm{Exc}^\mathrm{split}_*(\mathcal{X}^\omega_*[\mathcal{V}^{- 1}], \cD)
\]
\end{corollary}
\begin{corollary}
    Let $\cD$ be an idempotent complete pointed category. Then there is an 
    equivalence of categories 
    \[
        \mathrm{Gen}(\int_{\cal X ^\omega} \cal V ^\op, \cD) \simeq \mathrm{Exc}_*(\mathcal{X}^\omega_*[\mathcal{V}^{- 1}], \cD)
    \]
    \end{corollary}


%   \begin{remark}
%     A functor $F \colon \cC \to \cD$ which preserves split puhsouts also preserves
%     cofibers of split monomorphisms as it is a special case.
%     If $\cC$ and $\cD$ are semi-additive $\infty$-categories,
%     then we observe that tfae
%     \begin{itemize}
%       \item $F$ preserves split pushouts.
%       \item $F$ preserves cofibers of split monomorphisms.
%       \item $F$ is additive.
%     \end{itemize}
%     since we can classify split pushouts in a semi-additive $\infty$-category to be of the form
%     \[
%       \begin{tikzcd}
%         A & A \oplus X \\
%         B & B \oplus X
%         \arrow[from=1-1,to=1-2]
%         \arrow[from=1-1,to=2-1]
%         \arrow[from=2-1,to=2-2]
%         \arrow[from=1-2,to=2-2]
%       \end{tikzcd}.
%     \]
%   \end{remark}
% This is a special case of a more general lemma 
% \begin{lemma}
%     Let 
%     \[\begin{tikzcd}
%         \cC & \cD & \cE & \cC
%         \arrow["F", from=1-1, to=1-2]
%         \arrow["G", from=1-2, to=1-3]
%         \arrow["L", from=1-3, to=1-4]
%     \end{tikzcd}\]
%     be functors together with adjunction data 
%     $FL \dashv G$ with counit $\epsilon \colon FLG \Rightarrow \id$ and 
%     $L \dashv GF$ with counit $\theta \colon LGF \Rightarrow \id$ such that
%     both units 
%     $\id \Rightarrow FLG$ are mutually equivalent.
%     Then there exists an auto equivalence $\Phi \colon F \Rightarrow F$ such that 
%     \[
%     F(\theta) \simeq \Phi \circ \epsilon_F     
%     \]
%     are equivalent transformations $FLGF \Rightarrow F$.
% \end{lemma}
% \begin{proof}
%     Consider the following diagram 
%     \[\begin{tikzcd}
%         & {\Map(-,GF-)} \\
%         {\Map(L-,-)} & {\Map(L-,LGF-)} \\
%         {\Map(FL-,F-)} & {\Map(FL-,FLGF-)} & {\Map(FL-,F-)}
%         \arrow["{F(\theta)_*}", from=3-2, to=3-1]
%         \arrow["{(\epsilon_F)_*}"', from=3-2, to=3-3]
%         \arrow["{\theta_*}"', from=2-2, to=2-1]
%         \arrow["F"', from=2-1, to=3-1]
%         \arrow["F"', from=2-2, to=3-2]
%         \arrow["L"', from=1-2, to=2-2]
%         \arrow["\simeq"', from=1-2, to=2-1]
%         \arrow["\simeq", from=1-2, to=3-3]
%     \end{tikzcd}\]
% \end{proof}
% \begin{construction}\label{cX(n)}
%     Let $\mathbb W(n)$ be the pullback of categories 
%     \[\begin{tikzcd}
%         {\mathbb W(n)} & {\AffLin/\R^n} \\
%         {\AffLin^{\times 2}} & \AffLin
%         \arrow["{\mathrm{source}}", from=1-2, to=2-2]
%         \arrow["\oplus"', from=2-1, to=2-2]
%         \arrow[from=1-1, to=2-1]
%         \arrow[from=1-1, to=1-2]
%         \arrow["\lrcorner"{anchor=center, pos=0.125}, draw=none, from=1-1, to=2-2]
%     \end{tikzcd}\]
%     Let $l.e.$ ($r.e.$) denote the non-full subcategory of the slice category
%     ${\mathbb W(n)} := (\oplus \colon (\cO)^{\times 2} \to \cO)_{/\R^n}$
%     given by the class of morphisms which get send to equivalences under the
%     projections ${\mathbb W(n)} \to (\cO)^{\times 2} \xto{\pr_j} \cO$ for
%     $j = 1$ ($j = 2$ resp.).
%     For example, $r.e.$ is the pullback of
%     \begin{equation}\label{eq:defnre}
%       \cO \times (\cO)^\simeq \xto{(\oplus, \R^n)} \cO
%       \times \cO \xleftarrow{(s,t)} \cO^{\Delta^1}
%     \end{equation}
%     We define $\cO(n)$ to be the full subcategory of the span category
%     associated to the triple $({\mathbb W(n)}, r.e., l.e.)$ on the objects
%     $(V,W, \alpha \colon V\oplus W \to \R^n)$ where $\alpha$ is either an
%     equivalence or the $\infty$-map, but in that case we require $V = W = \emptyset$.
  
%     A typical morphism of $\AffLin(n)$ is of the form 
%           \[
%             (V,W,\alpha\colon  V\oplus W \xto{\simeq} \R^n) \xot{f}
%             (S,U,\beta \colon S \oplus U \to \R^n) \xto{g}
%             (V',W'\alpha' \colon V'\oplus W' \xto{\simeq} \R^n)  
%           \]
%           In particular it gives us a commutative diagram 
%           \[\begin{tikzcd}
%               && {S \oplus U} \\
%               {V\oplus W} && {\R^n} && {V'\oplus W'}
%               \arrow["{(\overleftarrow{\pr}g,\overrightarrow{\pr}g)}", from=1-3, to=2-5]
%               \arrow["{(\overleftarrow{\pr}f\overrightarrow{\pr}f)}"', from=1-3, to=2-1]
%               \arrow["{\alpha }"', from=2-1, to=2-3]
%               \arrow[from=1-3, to=2-3]
%               \arrow["{\alpha'}", from=2-5, to=2-3]
%           \end{tikzcd}\]
%           where $\overrightarrow{\pr}f$ and $\overleftarrow{\pr}g$ are equivalences or $V=W=\emptyset$.
%           Hence, we can think of the data of a morphism in $\AffLin(n)$ to 
%           consist of  
%           \begin{itemize}
%               \item A map $\phi \colon V' \to V$,
%               \item A map $\psi \colon W \to W'$,
%               \item A commutative square 
%               \[\begin{tikzcd}
%                   {V'\oplus W} & {V'\oplus W'} \\
%                   {V\oplus W} & {\R^n}
%                   \arrow["\alpha"', from=2-1, to=2-2]
%                   \arrow["{\alpha'}", from=1-2, to=2-2]
%                   \arrow["{\phi\oplus W}"', from=1-1, to=2-1]
%                   \arrow["{V'\oplus \psi}", from=1-1, to=1-2]
%               \end{tikzcd}\]
%           \end{itemize}
  
%     Moreover, let $\cX(n)$ be the total space of the Cartesian fibration 
%     associated to $\underline{\cO(n)}$. 
%     % supposedly the span category construction takes care of the vop 
%   \end{construction}
  
%   \begin{remark}\label{spansandunstraightening}
%     We want to apply Theorem 3.8 in \cite{hebestreit2020orthofibrations} to 
%     the functor $G \colon (\spaces^\omega)^\op \to \mathrm{AdTrip}$ which sends 
%     $X$ to the adequate triple $(\Fun(X,{\mathbb W(n)}), \Fun(X,r.e.), \Fun(X,l.e.))$.
%     This will yield a discription of $\cX(n)$ as a full subcategory of a 
%     span category. 
%     Since we are particularly interested in the Cartesian fibration associated to 
%     $G$ we want to compare our situation to the content of remark 3.16 of \cite{hebestreit2020orthofibrations}.
%     In order to not confuse ourselfs in the comparison between our notation and the 
%     notation in paper \cite{hebestreit2020orthofibrations} we give a little 
%     dictionary:
    
%     \begin{tabular}{l|r}
%       This Paper & \cite{hebestreit2020orthofibrations} \\
%       \hline\hline
%       backwards arrow & egressive morphism \\
%       \cline{1-2}
%       forwards arrow & ingressive morphism \\
%       \cline{1-2}
%       adequate Triple & adequate Triple \\ 
%       (category, backwards arrows, forwards arrows) &
%       (category, forwards arrows, backwards arrows) \\
%     \end{tabular}
    
%     Let $X:= \spaces^\omega$ and $Y := \int_{X \in {\spaces^\omega}^\op} \Fun(X,{\mathbb W(n)})$.
%     Then we have $Y^{\mathrm{eg}} = \int_{X \in {\spaces^\omega}^\op} \Fun(X,r.e.)$,
%     $Y_{\mathrm{in}} = \int_{X \in {\spaces^\omega}^\op} \Fun(X,l.e.) $ and 
%     \[
%       Y^{\mathrm{eq}}_{\mathrm{fw}} = \coprod_{X \in \spaces^\omega} r.e.^X 
%       := \left[ \int_{X \in (\spaces^\omega)^\op} \underline{r.e.} \right]
%       \times_{\spaces^\omega} (\spaces^\omega)^\simeq.
%     \] 
%     Remark 3.16 of \cite{hebestreit2020orthofibrations} tells us now that we 
%     have an equivalence of $\infty$-categories
%     \[
%       \int_{x \in {\spaces^\omega}^\op} \Span({\mathbb W(n)}^X,r.e.^X,l.e.^X) \simeq 
%       \Span \left(\int_{x \in {\spaces^\omega}^\op}{\mathbb W(n)}^X,\coprod_X r.e.^X , 
%       \int_{x \in {\spaces^\omega}^\op}l.e.^X \right).    
%     \]
%     We sum up: 
%     $\cX(n)$ is a full subcategory of the span category
%     of $\int_X \underline{{\mathbb W(n)}}$ with backwards arrows given by $\coprod_X r.e.^X$
%     and forward morphisms given by $\int_X l.e.^X$.
%   \end{remark}
  
  
%   \begin{lemma}\label{colimitlemma}
%     The functor $-\oplus \overrightarrow{\R} \colon {\cO}^{\times 2} \to {\cO}^{\times 2}
%       \colon (V,W) \mapsto (V, W\oplus \R)$ induces a functor
%     $-\oplus \overrightarrow{\R} \colon \cO(n)\to \cO(n+1)$,
%     compatible with the functors
%     $\cO(n) \to \Span({\mathbb W(n)}, r.e., l.e. ) \xto{(pr_1)_*} \Span(\cO,
%       {\cO}, {\cO}^\simeq) \simeq ({\cO})^\op$
%     so that we have an equivalence
%     \[
%       \colim (\cO(0) \xto{-\oplus \overrightarrow{\R}} \cO(1)
%       \xto{-\oplus \overrightarrow{\R}} \dots) \xrightarrow{\simeq} {\cO}^\op 
%       .\]
%   \end{lemma}
%   \begin{proof}
%     The existence of the induced functor
%     $\cO(n) \to \cO(n+1)$ and the compatibility is clear by inspection.
%     We show that the functor
%     \[
%       \colim (\cO(0)^X \xto{-\oplus \overrightarrow{\R}} \cO(1)^X
%       \xto{-\oplus \overrightarrow{\R}} \dots) \to ({\cO^X})^\op
%     \]
%     is fully faithful and essentially surjective for every compact space $X$.
%     Since $X$ is a compact space, we can pull it out the colimit diagram and 
%     see that we can reduce the claim to the case $X \simeq \pt$.
%     The essential surjectivity is clear as every object of $\AffLin$ is 
%     equivalent to $\R^k$ or $\emptyset$ for some $k$.
%     And these are evidently in the image. 
%     We now show that the functor is fully faithful.
%     Fix objects $(V,W,\alpha \colon V \oplus W \xto{\simeq} \R^n),
%       (V',W',\alpha' \colon V' \oplus W' \xto{\simeq} \R^n) \in \cO(n)$ and let
%     \[
%       M := \Map((V,W,\alpha \colon V \oplus W \xto{\simeq} \R^{n}),
%       (V',W',\alpha' \colon V \oplus W' \xto{\simeq} \R^{n}))
%       .\]
%     In the case $\emptyset \in \{V,W,V',W'\}$ we easily compute
%     $ M \simeq \pt \simeq \Map_{(\cO)^\op}(V',\emptyset) \simeq
%       \Map_{(\cO)^\op}(\emptyset, V)$ and the claim follows from the fact
%     $(\emptyset,\emptyset) \oplus \overrightarrow{\R} \simeq
%       (\emptyset,\emptyset) \in \cO(n+1)$ as $\emptyset \in \cO$
%     is an absorbing element with respect to $\oplus$.
  
%     So suppose $V,W,V',W' \neq \emptyset$. By unraveling the construction of
%     $\cO(n)$ we have that $M$ is equivalent to the core of the
%     full subcategory of
%     \[
%       \Fun(\Lambda^2_0, {\mathbb W(n)}) \times_{(\ev_1,\ev_2),
%         {\mathbb W(n)}\times {\mathbb W(n)}} \{(V,W,\alpha),(V',W',\alpha')\}
%     \]
%     on those wedges
%     \[
%       (V,W,\alpha) \xleftarrow{f} (A,B,\beta) \xto{g} (V',W',\beta)
%     \]
%     for which $f$ is a right equivalence and $g$ is a left equivalence.
%     By writing $ \Fun(\Lambda^2_0, {\mathbb W(n)}) $ as the
%     pullback ${\mathbb W(n)}^{\Delta^1} \times_{s,{\mathbb W(n)},s} {\mathbb W(n)}^{\Delta^1}$
%     and commuting limits with limits we find that $M$ is equivalent to the core of
%     \[
%       r.e./(V,W,\alpha) \times_{{\mathbb W(n)}} l.e./(V',W',\alpha').
%     \]
%     The category of right equivalences, for example,
%     is itself a pullback of categories, see \eqref{eq:defnre}.
%     Therefore, we can compute $r.e./(V,W,\alpha)$, after
%     commuting limits again, as the pullback of
%     \[
%       \cO/V \times (\cO)^\simeq/W \xto{(\oplus, \R^n)} \cO/V\oplus W \times
%       \cO/\R^n \xot{(s,t)} (\cO)^{\Delta^1}/\alpha
%     \]
%     The category $(\cO)^\simeq/W$ is contractible, so we can omit it in the above wedge.
%     Moreover, the above wedge is of the form
%     $A \times_{\pt} \pt \to B \times_{\pt} C \leftarrow D \times_D D$,
%     so we can compute its pullback $P$ as the
%     pullback of  $A \times_B D \to D \leftarrow \pt \times_C D$.
%     Let us first compute
%     \begin{eqnarray*}
%       \pt \times_C D &=& \{\id_{\R^n}\} \times_{\cO/\R^n,t} (\cO)^{\Delta^1}/\alpha \\
%       &\simeq & \{\id_{\R^n}\} \times_{(\cO)^{\Delta^1}, \ev_{\{1\} \subset \{1,2\} }}
%       \Fun(\Pow(\langle 2 \rangle), \cO)
%       \times_{(\cO)^{\Delta^1}, \ev_{\{2\} \subset \{1,2\} }} \{\alpha \} \\
%       & \simeq & \cO / V \oplus W  = B .
%     \end{eqnarray*}
%     One can check that the composition $B = \pt \times_C D \to D \to B$ is the identity,
%     and that makes the map from $P$ to $A = \cO/V$ an equivalence.
%     Similarly one can compute that $l.e./(V',W',\alpha')$ is equivalent to
%     $\cO/W'$.
%     If we put these results together we can identify $M$ with the core of the pullback
%     \[
%       \cO/V \to {\mathbb W(n)} \leftarrow \cO/W'
%     \]
%     where the left map sends a map $f\colon U \to V$ to the triple
%     $(U,W, U\oplus W \xto{f \oplus W} V \oplus W \xto{\alpha} \R^n)$ and
%     the right map sends $g \colon U \to W'$ to the triple
%     $(V',U,V' \oplus U \xto{V' \oplus g} V' \oplus W' \xto{\alpha'} \R^n)$.
  
%     There is an evident map from
%     $\Map(V',V) \times_{\Map(V'\oplus W,\R^n)} \Map(W,W')$
%     into $\cO/V \times_{{\mathbb W(n)}} \cO/W'$. In fact it is obtained
%     as the pullback:
%     \[\begin{tikzcd}[column sep=tiny]
%         {\Map(V',V)\times_{\Map(V'\oplus W,\R^n)}\Map(W,W')} && {\cO/V \times_{{\mathbb W(n)}}\cO/W} \\
%         \\
%         \\
%         {\pt \times_{\pt} \pt} && {\cO \times_{\id\times\{W\},\cO \times \cO,\{V'\}\times\id}\cO}
%         \arrow[from=1-1, to=4-1]
%         \arrow["{(V',W)}", from=4-1, to=4-3]
%         \arrow["{s\times_ss}", from=1-3, to=4-3]
%         \arrow[from=1-1, to=1-3]
%         \arrow["\lrcorner"{anchor=center, pos=0.125}, draw=none, from=1-1, to=4-3]
%       \end{tikzcd}\]
%     After taking cores the lower map becomes an equivalence, which shows that
%     $M$ is equivalent to $\Map(V',V) \times_{\Map(V'\oplus W,\R^n)} \Map(W,W')$.
%     Under this identification the forgetful functor $\cO(n) \to (\cO)^\op$
%     induces the projection $\Map(V',V) \times_{\Map(V'\oplus W,\R^n)} \Map(W,W') \to \Map(V',V)$
%     and the functor $-\oplus \overrightarrow{\R} \colon \cO(n) \to \cO(n+1)$
%     can be identified with the map
%     \[
%       \Map(V',V) \times_{\Map(V'\oplus W,\R^n)} \Map(W,W') \to
%       \Map(V',V) \times_{\Map(V'\oplus W \oplus \R,\R^{n+1})} \Map(W\oplus \R,W'\oplus \R)
%     \]
%     Now, it becomes clear, that the induced functor
%     $\colim \cO(n) \to (\cO)^\op$ is fully faithful,
%     whence we have proven that the map
%     \begin{eqnarray*}
%       \Map(W\oplus \R^k, W' \oplus \R^k) &\to& \Map(V' \oplus W \oplus \R^k, \R^{n+k}) \\
%       \phi &\mapsto & \alpha' \circ (\id_{V'} \oplus \phi)
%     \end{eqnarray*}
%     becomes an equivalence in the colimit for $k \to \infty$.
%     Since $\alpha'$ is an equivalence, we may as well show the claim for the system
%     \begin{eqnarray*}
%       \Map(W\oplus \R^k, W' \oplus \R^k) &\to&
%       \Map(V' \oplus W \oplus \R^k, V' \oplus W' \oplus \R^k) \\
%       \phi &\mapsto & \id_{V'} \oplus \phi
%     \end{eqnarray*}
  
%     Both spaces of affine linear embeddings are Thom spaces
%     $M_i$ for vector bundles $\xi_i \to A_i$,  $i = 1,2$, 
%     \begin{itemize}
%       \item $A_1 = \isoemb{W \oplus \R^k}{W' \oplus \R^k}$
%       \item $A_2 = \isoemb{V' \oplus W \oplus \R^k}{V' \oplus W' \oplus \R^k}$
%     \end{itemize}
%     of rank
%     $\dim W' - \dim W$.
%     Moreover the map between $M_1 \to M_2$ comes from a map of vector bundles
%     $\xi_1 \to \xi_2$, which induces isomorphisms on fibers.
%     Hence we can reduce to check that $A_1 \to A_2$ becomes an equivalence for 
%     $k \to \infty$.
%     After picking bases $W \cong \R^l, W' \cong \R^{l'}, V' \cong \R^{m'}$,
%     we obtain equivalences $A_1 \simeq O(l' + k)/O(l' - l)$ and
%     $A_2 \simeq O(m' + l' + k)/O(l' - l)$.
%     Under these choices, 
%     we can identify the map $A_1 \to A_2$ with the standard inclusion
%     $O(l' + k)/O(l' - l) \subset O(m' + l' + k)/O(l' - l)$,
%     which yields the desired equivalence on colimits for $k \to \infty$.
%   \end{proof}
  
%   \begin{corollary}\label{filterAVB}
%     The
%     functors $- \oplus (0, \R) \colon \cO(n) \to \cO(n + 1)$ induce an
%     equivalence 
%       \[
%         \colim (\cX(0) \xto{-\oplus \overrightarrow{\R}} \cX(1)
%         \xto{-\oplus \overrightarrow{\R}} \dots) \xrightarrow{\simeq} {\cX}
%       .\]
%   \end{corollary}
  
%   \begin{proof}
%     Apply the last lemma together with the fact that the
%     functor $\catinfty \to \catinfty$, which sends 
%     a category $\cD$ to the underying category of the 
%     cartesian unstraightening of $\underline{\cD}$
%     preserves filtered colimits.
%   \end{proof}
  


\begin{thebibliography}{99}
    % BibTeX-Einträge
\end{thebibliography}

\end{document}